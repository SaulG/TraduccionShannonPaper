\chapter*{Introducci\'{o}n}
\addcontentsline{toc}{chapter}{Introducci\'{o}n}

El reciente desarrollo de varios m\'{e}todos de modulaci\'{o}n como
PCM y PPM el cual intercambio el ancho de banda por se\~{n}al a ruido
ha intensificado el inter\'{e}s general de la teor\'{\i}a de la
comunicaci\'{o}n. El fundamento de esta teor\'{\i}a se encuentra en
los art\'{\i}culos importantes por \citet{Nyquist1} y \citet{Hartley}
sobre este tema. En el presente art\'{\i}culo se extiende la
teor\'{\i}a a incluir un n\'{u}mero de nuevos factor, en particular el
efecto del sonido en un canal, y los ahorros posibles debido a la
estructura estad\'{\i}stica del mensaje original y a la naturaleza del
destino final de la informaci\'{o}n. 

El problema fundamental de la comunicaci\'{o}n es el de reproducir en
un momento exacto o aproximado un mensaje seleccionado en otro
punto. Frecuentemente los mensajes tienen un significado; es decir que
refieren o est\'{a}n correlacionados de acuerdo con alg\'{u}n sistema
con ciertas entidades f\'{\i}sicas o conceptuales. Estos aspectos
semanticos de la comunicaci\'{o}n son irrelevantes a el problema de
ingenier\'{\i}a. El significante aspecto es que el actual mensaje es
uno de los seleccionados desde un conjunto de posibles mensajes. El
sistema debe estar dise\~{n}ado para operar por cada selecci\'{o}n
posible, no solamente por el \'{u}nico que ha sido escogido ya que es
desconocido en el momento del dise\~{n}o.

Si el n\'{u}mero de mensajes en el conjunto es finito entonces el
numero o cualquier funci\'{o}n mon\'{o}tico de este n\'{u}mero puede
ser considerado como una medida de la informaci\'{o}n producida cuando
un mensaje es escogido de un conjunto, todas las opciones son
igualmente probables. Como se ha se\~{n}alado por Hartley la
opci\'{o}n mas natural es la funci\'{o}n logar\'{\i}tmica. Aunque esta
definici\'{o}n debe ser generalizado consideradamente cuando se cuenta
la influencia de las estad\'{\i}sticas del mensaje y cuando tenemos un
rango continuo de mensajes, vamos a utilizar en todos los casos una
medida esencialmente logar\'{\i}tmica.

La medida logar\'{\i}tmica es m\'{a}s conveniente por varias razones:
\begin{enumerate}
\item{Es m\'{a}s \'{u}til en la pr\'{a}ctica. Par\'{a}metros
  importantes utilizados en la ingenier\'{\i}a como el tiempo, ancho
  de banda, n\'{u}mero de pasos, etc., tiende a variar linealmente con
  el logar\'{\i}tmo de n\'{u}mero de posibilidades. Por ejemplo,
  agregando un paso a un grupo dobla el n\'{u}mero de posible estado
  de pasos. Eso agrega 1 a la base 2 del logar\'{\i}tmo de este
  n\'{u}mero. Duplicando el tiempo aproximadamente hace que el
  n\'{u}mero de cuadrados posibles, o duplicando el logar\'{\i}tmo,
  etc.}
\item{Es m\'{a}s cerca a nuestro sentimiento intuitivo en cuanto a la
  medidad adecuada. Esto es cercanamente relacionado desde que medimos
  intuitivamente entidades por una comparaci\'{o}n lineal con
  est\'{a}ndares com\'{u}nes \cite{Nyquist1,Nyquist2}. Un sentimiento,
  por ejemplo, cuando perforas dos cartas deber\'{\i}an tener el doble
  de la capacidad de uno para la transmisi\'{o}n de informaci\'{o}n
  \cite{Hartley}.}
\item{Es matem\'{a}ticamente m\'{a}s adecuado. Muchas de las
  operaciones que limitan son simples en t\'{e}rminos de el
  logar\'{\i}tmo pero requerir\'{\i}a reemplantamiento descuidado en
  t\'{e}rminos del n\'{u}mero de posibilidades.}
\end{enumerate}

La selecci\'{o}n de la base del logar\'{\i}tmo corresponde a la
selecci\'{o}n de la unidad para medir la informaci\'{o}n. S\'{\i} la
base 2 es usado, las unidades resultantes podr\'{\i}an ser llamados
digitos binarios, o m\'{a}s brevemente bits, una palabra sugerida por
J.\ W.\ Tukey. Un dispositivo con dos posiciones estables, como lo es
un n\'{u}mero de paso o un circuito flip-flop, pueden almacenar un bit
de informaci\'{o}n. $N$ dispositivos pueden almacenar $N$ bits, desde
que el n\'{u}mero total de posibles estados es $2^N$ y $\log_2 * 2^N =
N$. Si la base $10$ es usada las unidades podr\'{\i}an ser llamados
d\'{\i}gitos decimales. Desde
\begin{equation}
\log_2 M = \log_10 M / \log_10^2 = 3.32 \log_10 M,
\end{equation}
un d\'{\i}gito decimal es alrededor de $3 \frac{1}{3}$ bits. Una rueda
d\'{\i}gital en una computadora de escritorio tiene $10$ posiciones
estables y por lo tanto, tiene una capacidad de almacenamiento de un
d\'{\i}gito decimal. En trabajo analitico donde la integraci\'{o}n y
la diferenciaci\'{o}n estan envueltos en base e es algunas veces
conveniente. Las unidades resultantes de informaci\'{o}n ser\'{\i}an
llamados unidades naturales. El cambio de base $a$ a base $b$
simplemente requiere la multiplicaci\'{o}n por $\log_b a$.

\begin{figure}[!ht]
\centerline{\includegraphics[width=120mm]{fig1shannon_esp.pdf}}
\caption{Diagrama esquem\'{a}tico de un sistema de comunicaciones
  general.}
\label{fig:1}
\end{figure}

Por un sistema de comunicaci\'{o}n se quiere decir que es un sistema
de un tipo indicado esquem\'{a}ticamente en la figura \ref{fig:1}. Eso
consiste de cinco partes esenciales:

\begin{enumerate}
\item{La fuente de informaci\'{o}n el cual produces el mensaje o
  secuencia de mensajes a ser comunicado a la terminal que recibe. El
  mensaje puede ser de varios tipos:
  \begin{enumerate}[(a)]
  \item{Una secuencia de letras como un sistema de telegrafo de
    teletipo;}
  \item{Una sola funci\'{o}n de tiempo $f(t)$ como un radio o
    telefon\'{\i}a;}
  \item{Una funci\'{o}n de tiempo y otras variables como
    televisi\'{o}n en blanco y negro --- aqu\'{\i} el mensaje
    podr\'{\i}a pasar a traves de una funci\'{o}n $f(x,y,t)$ de dos
    espacios de coordenadas y tiempo, la intensidad de la luz en el
    punto $(x, y)$ y el tiempo $t$ sobre un tubo en la placa;}
  \item{Dos o m\'{a}s funciones de tiempo, digames $f(t)$, $g(t)$,
    $h(t)$ --- este es el caso en ``tres dimensional'' transmisi\'{o}n
    de sonido o si el sistema est\'{a} dise\~{n}ado para dar servicio
    a varios canales individuales en multiplex;}
  \item{Varias funciones de varias variables --- en la televisi\'{o}n
    a color el mensaje consiste en tres funciones $f(x,y,t)$,
    $g(x,y,t)$, $h(x,y,t)$ definidas en tres dimensiones continuas ---
    tambi\'{e}n se podr\'{\i}a pensar que esas tres funciones como
    componentes de un campo vectorial definido en una regi\'{o}n -- de
    manera similar, distintas fuentes de televisiones en blanco y
    negro podr\'{\i}an producir ``mensajes'' que consisten en un
    n\'{u}mero de funciones de tres variables;}
  \item{Diversas combinaciones tambi\'{e}n se producen , por ejemplo,
    en la televisi\'{o}n con un canal de audio asociado.}
\end{enumerate}}

\item{Un transmisor que opera en el mensaje de alguna manera para
  producir una se\~{n}al adecuada para la transmisi\'{o}n sobre el
  canal. En la telefon\'{\i}a esta operaci\'{o}n consiste simplemente
  en el cambio de presi\'{o}n de sonido en una corriente el\'{e}ctrica
  proporcional. En la telegraf\'{\i}a tenemos una operaci\'{o}n de
  codificaci\'{o}n que produce una secuencia de puntos, guiones y
  espacios en el canal correspondiente al mensaje. En un sistema
  multiplex PCM las diferentes funciones de voz deben tomar muestras,
  compresiones, cuantificada y codificada, y finalmente intercalados
  adecuadamente para construir la se\~{n}al. Sistemas de vodocoder, la
  televisi\'{o}n y la modulaci\'{o}n de frecuencia son ejemplos de
  operaciones complejas aplicadas a los mensajes para obtener la
  se\~{n}al.}
\item{El canal es simplemente el medio usado para transmitir la
  se\~{n}al desde un transmitor a un receptor. Eso podr\'{\i}a ser un
  par de cables, un cable coazial, una radio frecuencia, un rayo de
  luz, etc.}
\item{El receptor ordinalmente hace la operaci\'{o}n inversa de lo que
  ya est\'{a} hecho por el transmitor, reconstruye el mensaje desde la
  se\~{n}al.}
\item{El destino es la persona (o cosa) hacia quien el mensaje es
  enviado.}
\end{enumerate}

Se desear\'{\i}a considerar ciertos problemas generales envueltos en
el sistema de comunicaci\'{o}n. Para hacer esto es necesario
representar varios elementos envueltos como lo son las entidades
matematicas, adecuadamente idealizada desde sus contrapartes
f\'{\i}sicas. A grandes rasgos se pueden clasificar los sistemas de
comunicaci\'{o}n en tres categor\'{\i}as principales: los discretes,
continuos y mixtos. Por un sistema discreto se quiere decir que en
tanto el mensaje como la se\~{n}al son una secuencia de simbolos
discretos. Un caso t\'{\i}pico es la telegraf\'{\i}a donde el mensaje
es una secuencia de letras y la se\~{n}al de una secuencia de puntos,
guiones y espacios. Un sistema continuo es aqul en el que est\'{a}n
tanto el mensaje como se\~{n}al tratadas como funciones continuas, por
ejemplo, la radio o la televisi\'{o}n. Un sistema mixto es una en la
que tanto las variables discretas y continuas aparecen, por ejemplo,
la transmisi\'{o}n PCM de voz. 

Consideremos en primer lugar el caso discreto. Este caso tiene
aplicaciones no s\'{o}lo en teor\'{\i}a de la comunicaci\'{o}n, sino
tambi\'{e}n en la teor\'{\i}a de las m\'{a}quinas de computaci\'{o}n,
el dise\~{n}o de las centrales telef\'{o}nicas y otros campos.
Adem\'{a}s el caso discreto crea una base para los casos continuos
mixtos que ser\'{a}n tratadas en la segunda mitad del art\'{\i}culo.

\clearpage

\part{Sistemas discretos silenciosos}
\label{part:1}

\chapter{El canal discreto silencioso}
\label{sec:1}

Teletipo y telegraf\'{\i}a son dos simples ejemplos de un canal
discreto para la transmisi\'{o}n de informaci\'{o}n. Generalmente, con
un canal discreto queremos decir que es un sistema, por lo cual una
secuencia de selecciones desde un conjunto de simbolos elementarios
$S_1,\ldots,$ puede ser transmitido desde un punto a otro. Cada uno de
los s\'{\i}mbolos $S_i$ es asumido a tener una cierta duraci\'{o}n en
tiempo ti segundos (no es necesariamente el mismo por cada $S_i$, por
ejemplo los puntos y guiones. Es no requerido que todas las posibles
secuencias de Si sean capaces de hacer transmisi\'{o}n en el sistema;
ciertas secuencias deben ser permitidas. Esos ser\'{\i}an posibles
se\~{n}ales por canal. Supongamos que en el telegrafo que los simbolos
son: 
\begin{enumerate}
\item{Un punto, que consta de un cierre de l\'{\i}nea para una
unidad de tiempo y, a continuaci\'{o}n de l\'{\i}nea abierta para una
unidad de tiempo;}
\item{Un gui\'{o}n, consta de tres unidades de tiempo
de cierre y una unidad abierta;}
\item{Un espacio de palabra de seis
unidades de l\'{\i}nea abierta. Se podr\'{\i}a colocar la
restricci\'{o}n de secuencias permisibles que no hay espacios siguen
uno a otro (por si dos espacios de letras son adyacentes, es
id\'{e}ntico con un espacio de palabra). La pregunta que ahora cuenta
es c\'{o}mo se puede medir la capacidad de un canal de transmitir
informaci\'{o}n.}
\end{enumerate}

En el caso del teletipo donde todos los simbolos son de la misma
duraci\'{o}n, y cualquier secuencia de 32 simbolos se permite y la
respuesta es f\'{a}cil. Cada simbolo representa cinco bits de
informaci\'{o}n. Si transmite el sistema de $N$ simbolos por segundo,
es natural decir que el canal tiene una capacidad de bits por segundo
$5 n$. Esto no significa que el canal del teletipo siempre ser\'{a} la
transmisi\'{o}n de informaci\'{o}n a este ritmo - esta es la tasa
m\'{a}xima posible y si o no la tasa real alcanza este m\'{a}ximo
depende de la fuente de informaci\'{o}n que alimenta el canal, como se
ver\'{a} m\'{a}s adelante. En el caso m\'{a}s general, con diferentes
longitudes de los simbolos y las limitaciones en las secuencias
permitidas, hacemos la siguiente definici\'{o}n.

\begin{definition}
La capacidad $C$ de un canal discreto es dado por
\begin{equation}
C = \lim_{T \rightarrow \infty} \frac{\log N(T)}{T},
\end{equation}
donde $N(T)$ es el numero de se\~{n}ales permitidas de duraci\'{o}n $T$.
\end{definition}

Es f\'{a}cilmente visto que en el caso de teletipo es reducido
hac\'{\i}a el resultado previo. Puede ser demostrado que el limite en
cuestion existe como un n\'{u}mero finito en la mayor\'{\i}a de los
casos de interes.  Supongamos que todas las secuencias de
s\'{\i}mbolos $S_1, \ldots, S_n$ son permitidos y esos simbolos tienen
una duraci\'{o}n $t_1, \ldots,t_n$. {\textquestiondown}Cu\'{a}l es la
capacidad del canal? Si $N(t)$ representa el n\'{u}mero de secuencias de
duraci\'{o}n $t$, nosotros tenemos:
\begin{equation}
N(t) = N(t-t_1) + N(t-t_2) + ... +N(t-t_n).
\end{equation}
El n\'{u}mero total es equivalente a la suma de n\'{u}mero de
secuencias terminando en $S_1, S_2, \ldots, S_n$ y esos son $N(t-t_1),
N(t-t_2), \ldots, N(t-t_n)$, respectivamente. Acorde a un buen
resultado en diferencias finitas, $N(t)$ es entonces asintotica para
un largo $t$ a $X_{t_0}$ donde $X_0$ es la soluci\'{o}n real de la
ecuaci\'{o}n caracter\'{\i}stica:
\begin{equation}
X-t_1 + X-t_2 + ... + X-t_n = 1.
\end{equation}

