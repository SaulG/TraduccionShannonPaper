Una Teoría matemática de la comunicación

por C.E. SHANNON

INTRODUCCION

El reciente desarrollo de varios metodos de modulación como PCM y PPM
el cual intercambio el ancho de banda por señal a ruido ha
intensificado el interes general de la teoría de la comunicación. A
base de esta teoría se encuentra en los papeles importantes
Nyquist_1 y Hartley_2 sobre este tema. 

En el presente artículo se extiende la teoría a incluir un número de
nuevos factor, en particular el efecto del sonido en un canal, y los
ahorros posibles debido a la estructura estadística del mensaje
original y a la naturaleza del destino final de la información. El
problema fundamental de la comunicación es el de reproducir en un
momento exacto o aproximado un mensaje seleccionado en otro punto.

Frecuentemente los mensajes tienen un significado; es decir que
refieren o están correlacionados de acuerdo con algún sistema con
ciertas entidades físicas o conceptuales. Estos aspectos semanticos de
la comunicación son irrelevantes a el problema de ingeniería. 

El significante aspecto es que el actual mensaje es uno de los
seleccionados desde un conjunto de posibles mensajes. El sistema debe
estar diseñado para operar por cada selección posible, no solamente
por el
único que ha sido escogido ya que es desconocido en el momento del
diseño.

Sí el número de mensajes en el conjunto es finito entonces el numero o
cualquier función monótico de este número puede ser considerado como
una medida de la información producida cuando un mensaje es escogido
de un conjunto, todas las opciones son igualmente probables. 

Como se ha señalado por Hartley la opción mas natural es la función
logarítmica. Aunque esta definición debe ser generalizado
consideradamente cuando se cuenta la influencia de las estadísticas
del mensaje y cuando tenemos un rango continuo de mensajes, vamos a
utilizar en todos los casos una medida esencialmente logarítmica.

La medida logarítmica es más conveniente por varias razones:

1. Es practicamente muy útil. Parámetros importantes utilizados en la
ingeniería como el tiempo, ancho de banda, número de pasos, etc.,
tiende a variar linealmente con el logarítmo de número de
posibilidades. Por ejemplo, agregando un paso a un grupo dobla el
número de posible estado de pasos. Eso agrega 1 a la base 2 del
logarítmo de este número. Duplicando el tiempo aproximadamente hace
que el número de cuadrados posibles, o duplicando el logarítmo, etc.

2. Es más cerca a nuestro sentimiento intuitivo en cuanto a la medidad
adecuada. Esto es cercanamente relacionado desde que medimos
intuitivamente entidades por una comparación lineal con estándares
comúnes.[?] Un sentimiento, por ejemplo, cuando perforas dos cartas
deberían tener el doble de la capacidad de uno para la transmisión de
información.[?]

3. Es matemáticamente más adecuado. Muchas de las operaciones que
limitan son simples en términos de el logarítmo pero requeriría
reemplantamiento descuidado en términos del número de posibilidades.


La selección de la base del logarítmo corresponde a la selección de la
unidad para medir la información. Sí la base 2 es usado, las unidades
resultantes podrían ser llamados digitos binarios, o más brevemente
bits, una palabra sugerida por J.W. Tukey. Un dispositivo con dos
posiciones estables, como lo es un número de paso o un circuito
flip-flop, pueden almacenar un bit de información. N dispositivos
pueden almacenar N bits, desde que el número total de posibles estados
es 2ˆN y log_2*2ˆN = N. Sí la base 10 es usada las unidades podrían
ser llamados dígitos decimales. Desde

    log_2 M = log_10 M / log_10ˆ2
          = 3.32 log_10 M,

[Imagen] Fig. 1 [Imagen]

un dígito decimal es alrededor de 3 1/3 bits. Una rueda dígital en una
computadora de escritorio tiene 10 posiciones estables y por lo tanto,
tiene una capacidad de almacenamiento de un dígito decimal. En trabajo
analitico donde la integración y la diferenciación estan envueltos en
base e es algunas veces conveniente. Las unidades resultantes de
información serían llamados unidades naturales. El cambio de base a a
base b simplemene requiere la multiplicación por log_b * a.

Por un sistema de comunicación se quiere decir que es un sistema de un
tipo indicado esquematicamente en la Fig. 1. Eso consiste de cinco
partes esenciales:

1. La fuente de información el cual produces el mensaje o secuencia de
mensajes a ser comunicado a la terminal que recibe. El mensaje puede
ser de varios tipos: (a) Una secuencia de letras como un sistem de
telegrafo de teletipo; (b) Una sola función de tiempo f(t) como un
radio o telefonía; (c) Una función de tiempo y otras variables como
televisión en blanco y negro --- aquí el mensaje podría pasar a traves
de una función f(x,y,t) de dos espacios de coordenadas y tiempo, la
intensidad de la luz en el punto (x, y) y el tiempo t sobre un tubo en
la placa; (d) Dos o más funciones de tiempo, digames f(t), g(t), h(t)
--  este es el caso en ``Tres dimensional'' transmisión de sonido o si
el
sistema está diseñado para dar servicio a varios canales individuales
en multiplex; (e) Varias funciones de varias variables -- en la
televisión a color el mensaje consiste en tres funciones f(x,y,t),
g(x,y,t),h(x,y,t) definido en tres dimensiones continuas -- también se
podría pensar que esas tres funciones como componentes de un campo
vectorial definido en una región -- de manera similar, distintas
fuentes de televisiones en blanco y negro podrían producir
``mensajes''
que consisten en un número de funciones de tres variables; (f)
Diversas combinaciones también se producen , por ejemplo, en la
televisión con un canal de audio asociado.

2. Un transmisor que opera en el mensaje de alguna manera para
producir una señal adecuada para la transmisión sobre el canal. En la
telefonía esta operación consiste simplemente en el cambio de presión
de sonido en una corriente eléctrica proporcional. En la telegrafía
tenemos una operación de codificación que produce una secuencia de
puntos, guiones y espacios en el canal correspondiente al mensaje. En
un sistema multiplex PCM las diferentes funciones de voz deben tomar
muestras, compresiones, cuantificada y codificada, y finalmente
intercalados adecuadamente para construir la señal. Sistemas de
vodocoder, la televisión y la modulación de frecuencia son ejemplos de
operaciones complejas aplicadas a los mensajes para obtener la señal.

3. El canal es simplemente el medio usado para transmitir la señal
desde un transmitor a un receptor. Eso podría ser un par de cables, un
cable coazial, una radio frecuencia, un rayo de luz, etc.

4. El receptor ordinalmente hace la operación inversa de lo que ya
esta hecho por el transmitor, reconstruye el mensaje desde la señal.

5. La destinación es la persona (o cosa) hacia quien el mensaje es
enviado.

 Se desearía considerar ciertos problemas generales envueltos en el
 sistema de comunicación. Para hacer esto es necesario representar
 varios elementos envueltos como lo son las entidades matematicas,
 adecuadamente idealizada desde sus contrapartes físicas.

//Checar
A grandes rasgos se pueden clasificar los sistemas de comunicación en
tres categorías principales: los discretes, continuos y mixtos. Por un
sistema discreto se quiere decir que en tanto el mensaje como la señal
son una secuencia de simbolos discretos. Un caso típico es la
telegrafía donde el mensaje es una secuencia de letras y la señal de
una secuencia de puntos, guiones y espacios. Un sistema continuo es
aqul en el que están tanto el mensaje como señal tratadas como
funciones continuas, por ejemplo, la radio o la televisión. Un sistema
mixto es una en la que tanto las variables discretas y continuas
aparecen, por ejemplo, la transmisión PCM de voz. Consideremos en
primer lugar el caso discreto. Este caso tiene aplicaciones no sólo en
teoría de la comunicación, sino también en la teoría de las máquinas
de computación, el diseño de las centrales telefónicas y otros campos.
Además el caso discreto crea una base para los casos continuos mixtos
que serán tratadas en la segunda mitad del papel.//Checar

PARTE I: Sistemas discretos silenciosos

1. El canal discreto silencioso

Teletipo y telegrafía son dos simples ejemplos de un canal discreto
para la transmisión de información. Generalmente, con un canal
discreto queremos decir que es un sistema, por lo cual una secuencia
de selecciones desde un conjunto de simbolos elementarios S1,..., Sn
puede ser transmitido desde un punto a otro. Cada uno de los simboos
Si es  asumido a tener una cierta duración en tiempo ti segundos (no
es necesariamente el mismo por cada Si, por ejemplo los puntos y
guiones. Es no requerido que todas las posibles secuencias de Si sean
capaces de hacer transmisión en el sistema; ciertas secuencias deben
ser permitidas. Esos serían posibles señales por canal. Supongamos que
en el telegrafo que los simbolos son: (1) Un punto, que consta de un
cierre de línea para una unidad de tiempo y, a continuación de línea
abierta para una unidad de tiempo; (2) Un guión, consta de tres
unidades de tiempo de cierre y una unidad abierta; (3) Un espacio de
palabra de seis unidades de línea abierta. Se podría colocar la
restricción de secuencias permisibles que no hay espacios siguen uno a
otro (por si dos espacios de letras son adyacentes, es idéntico con un
espacio de palabra). La pregunta que ahora cuenta es cómo se puede
medir la capacidad de un canal de transmitir información.

En el caso del teletipo donde todos los simbolos son de la misma
duración, y cualquier secuencia de 32 simbolos se permite y la
respuesta es fácil. Cada simbolo representa cinco bits de información.
Si transmite el sistema de N simbolos por segundo, es natural decir
que el canal tiene una capacidad de bits por segundo 5 n. Esto no
significa que el canal del teletipo siempre será la transmisión de
información a este ritmo - esta es la tasa máxima posible y si o no la
tasa real alcanza este máximo depende de la fuente de información que
alimenta el canal, como se verá más adelante. En el caso más general,
con diferentes longitudes de los simbolos y las limitaciones en las
secuencias permitidas, hacemos la siguiente definición.

Definición: La capacidada C de un canal discreto es dado por

            C = \im_{T \rightarrow \infty} frac{logN(T)}{T}
    
donde N(T) es el numero de señales permitidas de duración T.

Es facilmente visto que en el caso de teletipo es reducido hacía el
resultado previo. Puede ser demostrado que el limite en cuestion
existe como un numero finito en la mayoría de los casos de interes.
Supongamos que todas las secuencias de simbolos S1, ..., Sn son
permitidos y esos simbolos tienen una duración t1,...,tn. Cuál es la
capacidad del canal? Si N(t) representa el número de secuencias de
duración t, nosotros tenemos:

         N(t) = N(t-t1) + N(t-t2) + ... +N(t-tn).

El número tota es equivalente a la suma de número de secuencias
terminando en S1, S2,...,Sn y esos son N(t-t1), N(t-t2),...,N(t-tn),
respectivamente. Acorde a un buen resultado en diferencias finitas,
N(t) es entonces asintotica para un largo t a Xt_0 donde X0 es la
solución real de la ecuación característica:

         X-t1 + X-t2 + ... + X-tn = 1

\begin{equation}
8 \leq 10 \approx \, \text{True} \, \pi \Lambda
\end{equation}
