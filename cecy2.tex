
Deje un bloque de siete s\'imbolos que $X_{1}, X_{2},...,X_{7}$. De
estos $X_{3}, X_{5}, X_{6} y X_{7}$ son los mensajes de s\'{\i}mbolos
y eligidos arbitrariamente por la fuente. Los otros tres son
redundantes y se calculan como sigue:
\begin{center}
\begin{tabular}{c c c c}
$X_{4}$ & es elegido para hacer & $\alpha=X_{4}+X_{5}+X_{6}+X_{7}$ & par, \\
$X_{2}$ & `` `` `` & $\beta=X_{2}+X_{3}+X_{6}+X_{7}$ & `` `` ``, \\
$X_{1}$ & `` `` ``& $\gamma=X_{1}+X_{3}+X_{5}+X_{7}$ & `` `` ``.
\end{tabular}
\end{center}
Cuando un bloque de siete es recibido $\alpha, \beta$ y $\gamma$ son
calculados y si incluso llam\'o a cero, si un extra\~no llamado. El
n\'umero binario $\alpha \beta \gamma$ entonces da el sub\'indice de
la $X_{i}$ que no es correcto (si es 0 no hay error).

\clearpage

\begin{appendices}

\chapter{El crecimiento del n\'umero de bloques de s\'imbolos con una
condici\'on de estado finito}
\label{aini:1}

Siendo $N_{i}(L)$ el n\'umero de bloques de s\'imbolos de largo L
terminando en estado $i$.  Entonces tenemos
\begin{equation}
N_{j}(L)=\sum_{i,s} N_{i}(L-b_{ij}^{(s)})
\end{equation}
donde $b_{ij}^{1},b_{ij}^{2},...,b_{ij}^{m}$ el largo de los s\'imbolos los cuales pueden 
ser elegidos en estado $i$ y pasar a estado $j$. Esas son ecuaciones diferenciales lineales 
y el comportamiento como $L\rightarrow \infty $ debe ser de tipo
\begin{equation}
N_{j}=A_{j}W^{L}.
\end{equation}
Sustituyendo en la ecuaci\'on diferencial 
\begin{equation}
A_{j}W^{L} = \sum_{i,s}A_{i}W^{L-b_{ij}^{(s)}}
\end{equation}
o
\begin{equation}
\sum_{i} \left(\sum_{s}W^{b_{ij}^{(s)}}-\delta_{ij}\right)A_{i}=0.
\end{equation}
Para hacer posible esto el determinante
\begin{equation}
D(W)=\left| a_{ij}\right| =\left| \sum_{s}W^{-b_{ij}^{(s)}}-\delta_{ij}\right|
\end{equation}
debe desaparecer y esto determina $W$, que es, por supuesto, la mayor raíz real de $D = 0$.
La cantidad C est\'a dada entonces por
\begin{equation}
C = \lim_{L\rightarrow \infty}\frac{\log \sum A_{j}W^{L}}{L}=\log W
\end{equation}
y observamos tambi\'en que las mismas propiedades de crecimiento
resultan si nosotros requierimos que todos los bloques comiencen en el
mismo (elegido arbitrariamente) estado.

\clearpage

\chapter{Derivaci\'on de $H=-\sum p_{i}\log  p_{i}$}
\label{aini:2}

Siendo $H(\frac{1}{n},\frac{1}{n},...,\frac{1}{n})=A(n)$. Desde la
condici\'on (3) se puede descomponer una elecci\'on de $s^{m}$ con las
mismas probabilidades en una serie de $m$ elecciones de $s$ de
posiblemente igualmente probabilidades y obtener

\begin{equation}
A(s_{m})=mA(s).
\end{equation}

Igualmente

\begin{equation}
A(t^{n})=nA(t).
\end{equation}

Podemos elegir $n$ arbitrariamente grande y encontrar una $m$ para satisfacer

\begin{equation}
s^{m} \leq t^{n} < s^{(m+1)}.
\end{equation}

As\'i, tomando logaritmos y dividiendo en $n\log s$,

\begin{equation}
\frac{m}{n}\leq \frac{\log  t}{\log  s}\leq \frac{m}{n}+\frac{1}{n} \ o \ 
\left|\frac{m}{n}-\frac{\log  t}{\log  s}\right| < \epsilon 
\end{equation}

donde $\epsilon$ es arbitrariamente peque\~na.Ahora de la propiedad 
monot\'onica de $A(n)$,
\begin{equation}
A(s^{m})\leq A(t^{n})\leq A(s^{m+1})
\end{equation}
\begin{equation}
mA(s)\leq nA(t) \leq (m+1)A(s)
\end{equation}

Por lo tanto

\begin{equation}
H=K\left[\sum p_{i}\log \sum n_{i}-\sum p_{i}\log n_{i}\right]
\end{equation}
\begin{equation}
=-K\sum p_{i}\log \frac{n_{i}}{\sum n_{i}}=-K\sum p_{i}\log p_{i}.
\end{equation}

Si el $p_{i}$ es inconmensurables, que se puede aproximar por racionales y la misma 
expresi\'on debe contener por nuestra suposici\'on de continuidad. As\'i, la expresi\'on 
se mantiene en general. La elecci\'on del coeficiente $K$ es un asunto de conveniencia y 
asciende a la elecci\'on de una unidad de medida.

\clearpage

\chapter{Teoremas sobre fuentes erg\'odicas}
\label{aini:3}

Si es posible pasar de un estado con $P>0$ a cualquier otro a lo largo
de un camino de probabilidad $p>0$, el sistema es erg\'odico y una
fuerte ley de numeros largos es aplicada.  As\'i, el n\'umero de veces
que un camino dado $p_{ij}$ en una red se desplaza en una larga
secuencia de longitud $N$ es aproximadamente proporcional a la
probabilidad de estar en $i$, dice $P_{i}$, y escoge la ruta,
$P_{i}p_{ij}N$. Si $N$ es lo suficientemente larga la probabilidad de
porcentaje de error $\pm\delta$ es esto es menor que $\epsilon$ as\'i
que para todos, pero un conjunto de baja probabilidad de los n\'umeros
reales se encuentran dentro de los l\'imites
\begin{equation}
(P_{i}p_{ij}\pm \delta)N.
\end{equation}
Por lo tanto casi todas las secuencias tienen una probabilidad $p$ dada
por
\begin{equation}
p=\prod p_{ij}^{(P_{i}p_{ij}\pm \delta)N}
\end{equation}
y $\frac{\log  p}{N}$ es limitada por
\begin{equation}
\frac{\log  p}{N} = \sum (P_{i}p_{ij}\pm \delta)\log  p_{ij}
\end{equation}
o
\begin{equation}
\left|\frac{\log  p}{N} = \sum (P_{i}p_{ij}\pm \delta)\log p_{ij}\right|<\eta
\end{equation}

Esto demuestra el teorema \ref{th:3}. Teorema \ref{th:4} sigue
inmediatamente de esto en el c\'alculo de los l\'imites superior e
inferior para $n(q)$ basado en el rango posible de valores de $p$ en
el Teorema \ref{th:3}.  En el mixto (no erg\'{o}dico) caso si
\begin{equation}
L = \sum p_{i}L_{i}
\end{equation}
y las entropias de los componentes son $H_{1}\leq H_{2}\leq ... \leq
H_{n}$ tenemos :
\begin{theorem}
$\lim_{N\rightarrow \infty }\frac{\log n(q)}{N}=\varphi (q)$ es una
funci\'on de paso decreciente $\varphi (q)=H_{s}$ en el intervalo
$\sum_{1}^{s-1}\alpha_{i}<q<\sum_{1}^{s}\alpha_{i}$.
\label{nuevo}
\end{theorem}

Para demostrar teoremas \ref{th:5} y \ref{th:6}, primero note que
$F_{N}$ es mon\'otona decreciente porque el aumento de $N$ agrega un
sub\'indice a la entrop\'ia condicional. Una sencilla sustituci\'{o}n
de $p_{B_{i}}(S_{j})$ en la definici\'on de $F_{N}$ muestra que
\begin{equation}
F_{N} = NG_{N}-(N-1)G_{N}-1,
\end{equation}
y sumando esto por todas las $N$ dadas $G_{N}=\frac{1}{N}\sum
F{n}$. Por lo tanto $G_{N}\leq F_{N}$ y $G_{N}$ son mon\'{o}tonas
decrecientes. Tambi\'en se debe aproximar al mismo l\'imite. Usando el
teorema \ref{th:3} se observa que $\lim_{N\rightarrow \infty } G_{N} =
H$.

\clearpage

\chapter{Maximizar la tasa para un sistema de restricciones}
\label{a4}

Supongamos que tenemos un conjunto de restricciones sobre secuencias
de s\'imbolos que es del tipo de estado finito y puede ser
representado por una gr\'afica lineal. Siendo $\l_{ij}^{(s)}$ el largo
de varios s\'imbolos que pueden pasar de estado $i$ a estado $j$.  ?`
Qu\'e distribuci\'on de probabildades $P_{i}$ para los estados
diferentes y $p_{ij}^{(s)}$ para elegir un s\'imbolo $s$ en estado $i$
e ir a estado $j$ maximiza la tasa de generaci\'on de informaci\'on en
virtud de estas limitaciones? Las limitaciones defininen un canal
discreto y la velocidad m\'aximo debe ser menor o igual a la capacidad
$C$ de este canal, ya que si todos los bloques de longitud grande eran
igualmente probables, esta tasa dar\'ia lugar, y si es posible que
esto ser\'ia mejor. Se mostrar\'a que esta tasa se puede lograr
mediante la elecci\'on adecuada de $P_{i}$ y $p_{ij}^{(s)}$.  La tasa
es:
\begin{equation}
\frac{-\sum P_{i}p_{ij}^{(s)}\log p_{ij}^{(s)}}{\sum P_{i}p_{ij}^{(s)}l _{ij}^{(s)}} = \frac{N}{M}.
\end{equation}

Siendo $l_{ij}=\sum_{s}l _{ij}^{(s)}$. Evidentemente para un m\'aximo de 
$p_{ij}^{(s)}=k exp l_{ij}^{(s)}$. Las restricciones a la maximizaci\'on son 
$\sum P_{i} = 1$, $\sum P_{i}(p_{ij}-\delta_{ij})=0$. Por lo tanto podemos maximizar 
\begin{equation}
U=\frac{-\sum P_{i}p_{ij}\log p_{ij}}{\sum P_{i}p_{ij}l_{ij}}+\lambda \sum _{i}P_{i}+\sum\mu _{i}p_{ij}+\sum\eta _{j}P_{i}(p_{ij}-\delta_{ij})
\end{equation}

\begin{equation}
U=\frac{-\sum P_{i}p_{ij}\log p_{ij}}{\sum P_{i}p_{ij}l_{ij}}+\lambda \sum _{i}P_{i}+\sum\mu _{i}p_{ij}+\sum\eta _{j}P_{i}(p_{ij}-\delta_{ij})
\end{equation}

\begin{equation}
\frac {\partial U}{\partial p_{ij}}=-\frac{MP{i}(1+\log p_{ij})+NP_{i}l_{ij}}{M^{2}}+\lambda+\mu_{I}+\eta _{i}P_{i}=0.
\end{equation}

Resolviendo para $p_{ij}$
\begin{equation}
p_{ij}=A_{i}B_{j}D^{-l_{ij}}
\end{equation}

Desde

\begin{equation}
\sum _{j}p_{ij}=1,\ A_{i}^{-1}=\sum_{j}B_{j}D^{-l_{ij}}
\end{equation}
\begin{equation}
p_{ij}=\frac{B_{j}D^{-l_{ij}}}{\sum_{s}B_{s}D^{-l_{is}}}
\end{equation}

para luego

\begin{equation}
p_{ij}=\frac{B_{j}}{B_{i}}C^{-l_{ij}}
\end{equation}
\begin{equation}
\sum p_{i}\frac{B_{j}}{B_{i}}C^{-l_{ij}}=P_{j}
\end{equation}

o

\begin{equation}
\sum \frac{P_{i}}{B_{i}}C^{-l_{ij}}=\frac {P_{j}}{B_{j}}.
\end{equation}

Entonces si $\lambda_{i}$ satisface

\begin{equation}
\sum\gamma _{i}C^{-lij}=\gamma_{j}
\end{equation}
\begin{equation}
P_{i}=B_{i}\gamma _{i}.
\end{equation}

Tanto los conjuntos de ecuaciones para $B_{i}$ and $\gamma_{i}$ puede ser satisfecha ya 
que C es
\begin{equation}
\left | C^{-l_{ij}} - \delta _{ij}\right |=0.
\end{equation}

En este caso la tasa es
\begin{equation}
\sum \frac{P_{i}p_{ij}\log \frac{B_{j}}{B_{i}}C^{-l_{ij}}}{\sum P_{i}p_{ij}l_{ij}}=C-\frac{P_{i}p_{ij}\log \frac{B_{j}}{B_{i}}}{\sum P_{i}p_{ij}l_{ij}}
\end{equation}

pero

\begin{equation}
\sum P_{i}p_{ij}(\log B_{j}-\log B_{i})=\sum_{j}P_{j}\log B_{j}-\sum P_{i}\log B_{i}=0
\end{equation}

Por lo tanto la tasa es $C$ y como esto nunca se podr\'ia superar este es el 
m\'aximo, lo que justifica la soluci\'on supuesta.

\end{appendices}
