En caso de que existan restricciones en secuencias permitidas todav\'ia se puede 
obtener una ecuaci\'on diferencial de este tipo y encontrar C de la ecuaci\'on 
caracter\'istica. En el caso mencionado anteriormente de la telegraf\'ia

$N(t) = N(t-2)+N(t-4)+N(t-5)+N(t-7)+N(t-8)+N(t-10)$

como se ve por el c\'alculo de secuencia de los s\'imbolos de acuerdo al \'ultimo o 
al siguiente \'ultimo s\'imbolo ocurriendo.
Por lo tanto $-log\mu_{0}$ donde $\mu_{0}$ es la ra\'iz positiva de 
$1 = \mu^{2}+\mu^{4}+\mu^{5}+\mu^{7}+\mu^{8}+\mu^{10}$. Resoviendo eso encontramos 
que $C = 0.539$.

Una restricci\'on general se puede colocar en secuencias como la siguiente: Imagenemos
un numero de estados posibles $a_{1}, a_{2},...,a_{m}$. Por cada estado solo 
ciertos s\'imbolos del conjunto $S_{1},...S_{n}$ pueden ser transmitidos 
(diferentes subconjuntos por cada estado). Cuando uno de esos han sido transmitidos 
el estado cambia a un nuevo estado dependiendo de el viejo estado y del s\'imbolo en 
particular transmitido. Un ejemplo simple de esto es el caso del tel\'egrafo. Hay 
dos estados en funci\'on de si o no un espacio era el \'ultimo s\'imbolo transmtido. 
Si es as\'i, entonces s\'olo un punto o una raya puede ser enviado al lado y el 
estado cambia siempre. Si no, cualquier s\'imbolo puede ser transmitido y los 
cambios de estado si un espacio es enviado, de lo contrario sigue siendo el mismo. 
Las condiciones pueden ser indicadas en una gr\'afica lineal como se ve en la 
figura 2. Los puntos de uni\'on correspoden a los 


estados y las l\'ineas indicando los s\'imbolos posibles en un estado y el estado resultante. 
En el Ap\'endice 1 se muestran que si las condiciones en las secuencias 
permitidas puede ser descrita en la forma C puede existir y se puede calcular de 
acuerdo con el siguiente resultado:
\begin{theorem} Siendo $b_{ij}^{(s)}$ la duraci\'on del s\'imbolo $s^{th}$ el cual 
es permisible en el estado i y conduce al estado j. Despu\'es la capacidad del canal 
C es igual a $logW$ donde $W$ es la ra\'iz real m\'as larga de la ecuaci\'on:
\begin{equation}
\left|\sum_{s}W^{-b_{ij}^{(s)}}-\delta_{ij}\right|=0
\end{equation}
donde $\delta_{ij}=1 si i=j$ y es cero en cualquier otro caso
\end{theorem}
Por ejemplo en el caso del tel\'egrafo el determinante es:
\begin{equation}
\begin{vmatrix}
-1&(W^{-2}+W^{-2}) \\ 
 (W^{-3}+W^{-6})&(W^{-2}+W^{-2}-1) 
\end{vmatrix}=0
\end{equation}
Esta expansi\'on conduce a la ecuaci\'on dada anteriormente para este caso.

\section{LA FUENTE DISCRETA DE LA INFORMACI\'ON}
Hemos visto que bajo condiciones muy generales el logaritmo del n\'umero de se\~nales 
posibles en un canal discreto aumenta linealmente con el tiempo. La capacidad de 
transmisi\'on de informaci\'on se puede especificar dando a esta tasa de aumento, el n\'umero 
de bits por segundo necesarios para especificar la se\~nal particular utilizada.
Consideremos ahora la fuente de informaci\'on. ?`C\'omo es una fuente de informaci\'on a ser 
descrita matem\'aticamente, y la cantidad de informaci\'on en bits por segundo se produce en una 
fuente dada? El principal punto en cuesti\'on es la efecto de los datos estad\'isticos sobre la 
fuente en la reducci\'on de la capacidad requerida de la canal, por el uso de una correcta 
codificaci\'on de la informaci\'on. En la telegraf\'ia, por ejemplo, los mensajes para ser 
transmitidos consisten en secuencias de letras. Estas secuencias, sin embargo, no son 
completamente aleatorias. En general, las oraciones tienen una estructura estad\'istica de, 
por ejemplo, Ingl\'es. La letra E se produce con m\'as frecuencia que Q, la secuencia TH con m\'as 
frecuencia que XP, etc. La existencia de esta estructura permite hacer un ahorro en el tiempo (o 
capacidad de canal) para codificar correctamente las secuencias de mensajes en una secuencia 
de se\~nales. Esto ya se hace en una medida limitada en telegraf\'ia utilizando el s\'imbolo de canal 
m\'as corto, un punto, por la m\'as letra E que es la m\'as com\'un en Ingl\'es, mientras que las letras infrecuentes, 
Q, X, Z est\'an representados por secuencias m\'as largas de puntos y rayas. esta idea se lleva a\'un 
m\'as lejos en ciertos c\'odigos comerciales donde las palabras y frases comunes est\'an representados
por cuatro o cinco grupos de c\'odigo de letra con un considerable ahorro de tiempo promedio. 
Los telegramas est\'andar de saludo y aniversario se han usado tanto hasta el punto de 
que se codifican una o dos frases en una secuencia relativamente corta de n\'umeros.

Podemos pensar en una fuente discreta de generar el mensaje, s\'imbolo por s\'imbolo. Se elegir\'a 
una sucesi\'on de s\'imbolos de acuerdo con ciertas probabilidades dependiendo, en general, 
de las opciones anteriores, as\'i como los s\'imbolos en cuesti\'on. Un sistema f\'isico, o 
un modelo matem\'atico de un sistema que produce una secuencia de s\'imbolos gobernadas por un 
conjunto de probabilidades, se le conoce como un proceso estoc\'astico. 
Podemos considerar por lo tanto que una fuente discreta puede ser representada por un proceso 
estoc\'astico. A la inversa, cualquier proceso estoc\'astico que produce una secuencia discreta 
de s\'imbolos elegidos a partir de un conjunto finito puede ser 
considerado una fuente discreta. Esto incluye casos como:

\begin{enumerate}
\item Idiomas naturales escritos como el Ingl\'es, Alem\'an y Chino.
\item Fuentes de informaci\'on continuos que se han vuelto discreta por alg\'un proceso 
de cuantificaci\'on. Por ejemplo, la se\~nal de voz cuantizada desde un transmisor PCM, o 
una se\~nal de televisi\'on cuantizada.
\item Casos matem\'aticos en los que simplemente se definen abstractamente procesos estoc\'asticos 
que generan una secuencia de s\'imbolos. Los siguientes son ejemplos de este \'ultimo tipo de fuente.
\begin{enumerate}
\item Supongamos que tenemos cinco letras A, B, C, D, E, que se eligen cada una con probabilidad 0.2, 
elecciones sucesivas siendo independientes. Esto dar\'ia lugar a una secuencia, la siguiente 
es un t\'ipico ejemplo.

B D C B C E C C C A D C B D D A A E C E E A

A B B A E D E C A C E E E E B A C B C E A D

Este fue construido con el uso de una tabla de azar.
\item Utilizando las mismas cinco letras y siendo las probabilidades 0.4, 0.1, 
0.2, 0.2, 0.1, respectivamente, con decisiones independientes sucesivas. Un mensaje 
t\'ipico de esta fuente es entonces:

A A A C D C B D C E A A D A D A C E D A

E A D C A B E D A D E D C C A A A A A D.

\item Una estructura m\'as complicada se obtiene si los s\'imbolos no son elegidos 
de manera independiente pero sus probabilidades dependen de las letras anteriores. En 
el caso m\'as simple este tipo de elecci\'on depende solamente de la letra anterior y no
de las que estan antes. La estructura estad\'istica puede entonces ser descrita por un 
conjunto de probabilidades de transici\'on $p_{i}(j)$, la probabilidad de que la letra `$i$ 
es seguida por la letra $j$. Los \'indices de $i$ y $j$ se extienden sobre todos los 
s\'imbolos posibles. Una segunda manera equivalente de especificar la estructura es dar 
el "diagrama" de probabilidades $p(i,j)$, la frecuencia relativa de el diagrama $i j$.
La frecuencia de letras $p(i)$, (la probabilidad de la letra i), la transici\'on de 
probabilidades $p_{i}(j)$ y el diagrama de probabilidades $p(i,j)$ son descritos en las 
siguientes formulas:
\begin{equation}
p(i)=\sum_{j}p(i,j)=\sum_{j}p(j,i)=\sum_{j}p(j)p_{j}(i) 
\end{equation}
\end{enumerate}
\end{enumerate}
