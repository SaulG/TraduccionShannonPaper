\clearpage

\part{La tasa para una fuente continua}
\label{part:5}

\chapter{Las funciones de evaluaci\'{o}n de fidelidad}
\label{sec:27}

En el caso de una fuente discreta de informaci\'on nos fue posible
determinar una definida tasa de generaci\'on de informaci\'on, esta es
la entrop\'ia del proceso estoc\'astico subyacente. Con una continua
fuente, la situaci\'on es m\'as complicada. En primer lugar, una
cantidad continuamente variable puede ser asumida como un n\'umero
infinito de valores y por lo tanto requiere un n\'umero infinito de
d\'igitos binarios para su especificaci\'on exacta. Esto significa que
para transmitir la salida de una fuente continua con una {\em
 recuperaci\'on exacta} en el punto de recepci\'on, requiere
generalmente un canal de capacidad infinita (en bits por
segundo). Debido a que, ordinariamente, los canales tienen una cierta
cantidad de ruido, y por lo tanto una capacidad finita, la
transmisi\'on exacta es imposible.

Esto, aun as\'i, evade el problema real. De forma pr\'actica, nosotros
no estamos interesados en transmisi\'on exacta cuando tenemos una
fuente continua, sino solamente en la transmisi\'on dentro de una
cierta tolerancia. La cuesti\'on es si podemos asignar una tasa
definida a una fuente continua cuando requerimos solamente una cierta
fidelidad de recuperaci\'on, medida en una forma adecuada. Claro, a
como los requerimientos de fidelidad sean incrementados la tasa se
incrementar\'a de igual manera. Ser\'a mostrado que podemos, en casos
muy generales, definir tal tasa, teniendo la propiedad de que es
posible, propiamente mediante la codificaci\'on de la informaci\'on
para transmitirla a otro canal cuya capacidad sea igual a la tasa en
cuesti\'on, y as\'i satisfacer los requerimientos de fidelidad. Un
canal de menor capacidad es insuficiente.

Primero es necesario dar la formulaci\'on matem\'atica general de la
idea de fidelidad de transmisi\'on. Considera el conjunto de mensajes
de larga duraci\'on, digamos $T$ segundos. La fuente es descrita dando
la densidad de probabilidad en el espacio asociado, as\'i que la
fuente seleccione el mensaje en cuesti\'on $P(x)$. Un cierto sistema
de comunicaci\'on es descrito(desde el punto de vista externo) dando
la probabilidad condicional $P_{x}(y)$ as\'i que si el mensaje $x$ es
producido por la fuente, el mensaje recuperado en el punto de
recepci\'on ser\'a $y$. El sistema como un todo(incluyendo la fuente y
el sistema de transmisi\'on) es descrito por la funci\'on de
probabilidad $P(x, y)$, probabilidad de tener mensaje $x$ y salida
final $y$. Si esta funci\'on es conocida, las caracter\'isticas
completas del sistema desde el punto de vista de fidelidad son
conocidas. Cualquier evaluaci\'on de fidelidad debe corresponder
matem\'aticamente a una operaci\'on aplicada a $P(x, y)$. Esta
operaci\'on debe tener por lo menos las propiedades de un simple
ordenamiento de sistemas, por ejemplo, debe ser posible decir que dos
sistemas representados por $P_{1}(x, y)$ Y $P_{2}(x, y)$ que, de
acuerdo a nuestro criterio de fidelidad cumpla con ya sea (1) el
primero tiene una fidelidad m\'as alta, (2) el segundo tiene una
fidelidad m\'as alta, o (3) cuentan con una fidelidad
equivalente. Esto significa que el criterio de fidelidad puede ser
representado mediante una funci\'on n\'umericamente valuada.
\begin{equation} v(P(x,y)) \end{equation}
cuyos argumentos van m\'as all\'a de las posibles funciones de
probabilidad $P(x,y)$. Ahora mostraremos que bajo suposiciones muy
generales y razonables, la funci\'on $v(P(x,y))$ puede ser escrita en
una forma aparentemente mucho m\'as especializada, esta siendo un
promedio de una funci\'on $\rho(x,y)$ sobre el conjunto de valores
posibles de $x$ y $y$:
\begin{equation} 
v(P(x,y)) = \int \int P(x,y) \rho(x,y) \diff x  \diff y . 
\end{equation}
Para obtener esto necesitamos solamente asumir (1) que la fuente y el
sistema son erg\'odicos as\'i que una muestra muy larga ser\'a,
probablemente cercana a 1, t\'ipicamente del conjunto, y (2) que la
evaluaci\'on es razonable en el sentido que es posible, mediante la
observaci\'on de una tipica entrada y salida $x_{1}$ y $y_{1}$, formar
la evaluaci\'on tentativa en la base de esas muestra; y si estas
muestras son incrementadas en duraci\'on la evaluaci\'on tentativa,
con probabilidad 1, se acercar\'a a la evaluaci\'on exacta basada en
un total conocimiento de $P(x, y)$. Digamos que la evaluaci\'on
tentativa es $\rho(x, y)$. Entonces la funci\'on $\rho(x, y)$ se
acerca (como $T \rightarrow \infty$) a una constante para la mayor\'ia
(x,y) los cuales est\'an en la regi\'on altamente probable
correspondiente al sistema:
\begin{equation} \rho(x, y) \rightarrow v(P(x, y)) \end{equation}
y tambi\'en podemos escribir
\begin{equation} \rho(x, y) \rightarrow \int \int P(x,y)\rho(x, y) \diff x  \diff y \end{equation}
debido a que
\begin{equation} \int \int P(x, y) \diff x  \diff y = 1 \end{equation}
Esto establece el resultado deseado. La funci\'on $\rho(x, y)$ tiene
la naturaleza general de una "distancia" entre $x$ y $y^{9}$. Mide que
tan indeseable es (de acuerdo a nuestro criterio de fidelidad) recibir
$y$ mientras $x$ es transmitido. El resultado general dado
anteriormente puede ser expresado como sigue: Cualquier evaluaci\'on
razonable puede ser representada como un promedio de una funci\'on de
distancia sobre el conjunto de mensajes y mensajes recuperados $x$ y
$y$ ponderados de acuerdo a la probabilidad $P(x, y)$ de obtener el
par en cuesti\'on, siempre que la duraci\'on $T$ de los mensajes sea
suficientemente larga \footnote{No es ``m\'etrica'' en el sentido
estricto, ya que en general no satisface uno u otro ya sea: $\rho(x,y)
= \rho(y,x)$ o: $\rho(x,y) + \rho(y,z) \geq \rho(x,x)$.}.

\begin{enumerate}
\item{Criterio R.M.S.
\begin{equation} v = (x(t) - y(t))^{2} \end{equation}
En esta medida de fidelidad muy com\'unmente usada, la funci\'on de
distancia $\rho(x,y)$ es (aparte de un factor constante) el cuadrado
de la distancia Euclidiana ordinaria entre los puntos $x$ y $y$ en la
funci\'on espacio asociada.
\begin{equation} \rho(x, y) = \frac{1}{T} \int_0^T [x(t) - y(t)]^{2} \diff t \end{equation}}
\item{Criterio R.M.S. con frecuencia ponderada. M\'as generalmente uno puede
aplicar diferentes ponderaciones a los diferentes componentes de frecuencia
antes de usar una medici\'on de fidelidad R.M.S. Esto es el equivalente a pasar
la diferencia $x(t) - y(t)$ a trav\'es de un filtro de conformaci\'on y entonces 
determinar la potencia promedio en la salida.
As\'i, sea 
\begin{equation} e(t) = x(t) - y(t) \end{equation}
y
\begin{equation} f(t) = \int_{-\infty}^{\infty}  \epsilon(\tau)k(t - \tau) 
\diff \tau \end{equation}
entonces
\begin{equation} 
\rho (x, y) = \frac{1}{T} \int_0^T f(t)^{2} \diff t 
\end{equation}}
\item{Criterio del error absoluto
\begin{equation} 
\rho(x,y) = \frac{1}{T} \int_0^T | x(t) - y(t)| \diff t . 
\end{equation}}
\item{La estructura de la oreja y el cerebro determina
 impl\'icitamente una evaluaci\'on, o m\'as bien un n\'umero de
 evaluaciones, apropiado en el caso de transmisi\'on de m\'usica o
 habla. Hay, por ejemplo, un criterio de ``inteligibilidad'' en el cual
 $\rho(x,y)$ es equivalente a la frecuencia relativa de palabras
 incorrectamente interpretadas cuando el mensaje $x(t)$ es recibido
 como $y(t)$. Aunque no podemos dar una representaci\'on expl\'icita
 de $\rho(x,y)$, en esos casos podr\'ia, en principio, ser
 determinada por suficiente experimentaci\'on. Algunas de sus
 propiedades hacen seguimiento a buenos experimentos conocidos sobre
 el o\'ido, por ejemplo, la oreja es relativamente insensible a la
 fase y la sensibilidad de amplitud y frecuencia es aproximadamente
 logar\'itmica.}
\item{El caso discreto puede ser considerado como una
 especializaci\'on en la cual hemos asumido t\'acitamente una
 evaluaci\'on basada en la frecuencia de los errores. La funci\'on
 $\rho(x,y)$ es entonces definida como el n\'umero de s\'imbolos en
 la secuencia y que difieren de s\'imbolos correspondientes en $x$
 dividido por el total de n\'umero de s\'imbolos en $x$.}
\end{enumerate}

\clearpage

\chapter{La tasa para una fuente relativa a una evaluaci\'on de fidelidad}
\label{sec:28}

Estamos ahora en una posici\'on de definir la tasa de generaci\'on de
informaci\'on para una fuente continua. Se nos da $P(x)$ para la
fuente y una evaluaci\'on v determinada por una funci\'on de distancia
$\rho(x,y)$ la cual se asumir\'a continua en ambos $x$ y $y$. Con un
sistema particular $P(x,y)$ la calidad es medida por
\begin{equation} 
v = \int \int \rho(x,y) P(x,y) \diff x  \diff y 
\end{equation}
M\'as a\'un, la tasa de flujo de d\'igitos binarios correspondientes a
$P(x,y)$ es
\begin{equation} 
R = \int \int P(x,y) \log \frac{P(x,y)}{P(x)P(y)} \diff x  \diff y \end{equation}
Definimos que la tasa $R_{1}$ de generaci\'on de informaci\'on para
una calidad de reproducci\'on dada v1 sea el m\'inimo R cuando
mantenemos v fija en $v_{1}$ y $P_{x}(y)$ variable. Esto es:
\begin{equation} R_{1} = \min_{P_{x}(y)} \int \int P(x,y) \log \frac{P(x,y)}{P(x)P(y)} \diff x  \diff y \end{equation}
sujeto a la restricci\'on:
\begin{equation} v_{1} = \int \int P(x,y) \rho(x,y) \diff x  \diff y . \end{equation}
Esto significa que consideramos, en efecto, todos los sistemas de
comunicaci\'on que pueden ser usados y que transmiten con la fidelidad
requerida. La tasa de transmisi\'on en bits por segundo es calculada
para cada uno y escogemos el que tiene la menor tasa. \'Esta ultima
tasa es la tasa que asignamos a la fuente para la fidelidad en
cuesti\'on.

La justificaci\'on de esta definici\'on est\'a en el siguiente resultado:

\begin{theorem}
Si una fuente tiene una tasa $R_{1}$ para una valuaci\'on $v_{1}$, es
posible codificar la salida de la fuente y transmitirla sobre un canal
de capacidad $C$ con fidelidad tan cercana a $v_{1}$ como se desee,
siempre que $R_{1} \leq C$. Esto no es posible si $R_{1} > C$ .
\end{theorem}

El \'ultimo enunciado del teorema sigue inmediatamente de la
definici\'on de $R_{1}$ y resultados previos. Si esto no fuera cierto
podr\'iamos transmitir m\'as de C bits por segundos sobre un canal de
capacidad C. La primera parte del teorema es comprobada por un
m\'etodo an\'alogo al que fue usado en el Teorema \ref{FALTA PONER
 ETIQUETA}. Podemos, en primer lugar, dividir el espacio $(x,y)$ en
un gran n\'umero de peque\~{n}as celdas y representar la situaci\'on
en un caso discreto. Esto no va a cambiar la funci\'on de evaluaci\'on
por m\'as que una peque\~{n}a cantidad arbitraria (cuando las celdas
son muy peque\~{n}as) debido a la continuidad asumida para
$\rho(x,y)$. Suponga que $P_1(x,y)$ es el sistema particular el cual
minimiza la tasa y da $R_{1}$. Escogemos desde las $y$'s de alta
probabilidad, un conjunto al azar que contenga
\begin{equation} 2^{(R1 + E)T} \end{equation}
miembros donde $E \rightarrow 0$ como $T \rightarrow \infty$. Con una
$T$ grande, cada punto escogido ser\'a conectado por una linea de alta
probabilidad (como en la figura \ref{FALTA LA ETIQUETA}) a un conjunto
de $x$'s. Un c\'alculo similar al usado para comprobar el Teorema
\ref{FALTA LA ETIQUETA} muestra que con una $T$ grande la mayor\'ia de
las $x$'s son cubiertas por los $fans$ de los puntos y escogidos, para
la mayor\'ia de las elecciones de $y$'s. El sistema de comunicaci\'on
a ser usado opera como sigue: Los puntos seleccionados son n\'umeros
binarios asignados. Cuando un mensaje $x$ es originado se encontrara
dentro de al menos uno de los $fans$(con probabilidad acerc\'andose
uno ya que $T \rightarrow \infty$). El n\'umero binario
correspondiente es transmitido(o uno de ellos es escogido
arbitrariamente si existen m\'ultiples) sobre el canal por modos de
codificaci\'on adecuados para dar una pequeña probabilidad de
error. Ya que $R_{1} \leq C$, esto es posible. En el punto de
recepci\'on la y correspondiente es reconstruida y usada como el
mensaje de recuperaci\'on.

La evaluaci\'on $v'_{1}$ para este sistema se puede hacer
arbitrariamente cercana a $v_{1}$ tomando una T suficientemente
grande. Esto es debido a el hecho de que para cada muestra larga de un
mensaje x(t) y un mensaje de recuperaci\'on $y(t)$, la evaluaci\'on se
acerca a $v_{1}$ (con probabilidad 1). Es interesante notar que, en
este sistema, el ruido en el mensaje recuperado es en realidad
producido por un tipo de cuantificaci\'on general en el transmisor y
no producido por el ruido en el canal. Es m\'as o menos an\'alogo al
ruido de cuantificaci\'on en PCM.

\clearpage

\chapter{El c\'alculo de las tasas}
\label{sec:29}

La definici\'on de la tasa es similar en muchos aspectos a la
definici\'on de capacidad de canal. En la primera:
\begin{equation} 
R = \min_{P_{x}(y)} \int \int P(x,y) \log \frac{P(x,y)}{P(x)P(y)}
\diff x \diff y 
\end{equation}
con $P(x)$ y $v_{1} = \int \int P(x,y) \rho(x,y) \diff x \diff y $
fija. En la segunda:
\begin{equation} 
C = \max_{P(x)} \int \int P(x,y) \log \frac{P(x,y)}{P(x)P(y)} 
\diff x \diff y 
\end{equation} 
con $P_{x}(y)$ fija y posibilidad de uno o
m\'as restricciones (por ejemplo, una limitaci\'on de potencia
promedio) de la forma $K = \int \int P(x,y) \lambda(x,y) \diff x \diff
y $. Una soluci\'on parcial del problema de maximizaci\'on general
para determinar la tasa de una fuente se puede dar.

Usando el m\'etodo de Lagrange, consideramos:
\begin{equation} \int \int [ P(x,y) \log \frac{P(x,y)}{P(x)P(y)} + \mu P(x,y)\rho(x,y) + v(x)P(x,y)] \diff x  \diff y \end{equation}
La equaci\'on variacional (cuando tomamos la primera variaci\'on de $P(x,y)$) lleva a:
\begin{equation} Py(x) = B(x)\epsilon^{-\lambda \rho(x,y)} \end{equation}
donde $\lambda$ es determinada para dar la fidelidad requerida y $B(x)$ es elegida para
satisfacer:
\begin{equation} \int B(x) \epsilon^{\lambda \rho(x,y)} \diff x  = 1. \end{equation}
			
Esto muestra que, con la mejor codificaci\'on, la probabilidad
condicional de una cierta causa de variaci\'on recibida $y$,
$P_{y}(x)$ estar\'a en decline exponencialmente con la funci\'on de
distancia $\rho(x,y)$ entre el $x$ y $y$ en cuesti\'on. En el caso
epecial donde la funci\'on de distancia $\rho(x,y)$ depende solo en la
diferencia (vector) entre $x$ y $y$,
\begin{equation} \rho(x,y) = \rho(x - y) \end{equation}
tenemos
\begin{equation} \int B(x) \epsilon^{-\lambda p(x-y)} \diff x = 1 \end{equation}
			
Por lo tanto $B(x)$ es contante, digamos alpha, y
\begin{equation} P_{y}(x) = \alpha \epsilon^{ -\lambda \rho(x - y)}. \end{equation}
			
Desafortunadamente etas soluciones formales son dif\'iciles de evaluar
en casos particulares y parece ser de poco valor. De hecho, el calculo
actual de las tasas ha sido llevado a cabo en solo alguno casos muy
simples. Si la funci\'on de ditancia p(x,y) es el cuadrado medio de
la discrepancia entre $x$ y $y$, y el mensaje conjunto es ruido
blanco, la tasa puede ser determinada. En ese cao tenemos
\begin{equation} R = \min[ H(x) - H_{y}(x)] = H(x) - \max H_{y}(x) 
\end{equation}
con $N = \overline{(x-y)^{2})}$. Pero el $\max H_{y}(x)$ ocurre cuando
$y - x$ es un ruido blanco, y es equivalente a $W_{1} \log 2 \pi
\epsilon N$ donde $W_{1}$ es el ancho de banda del mensaje conjunto.
Por lo tanto
\begin{equation} R = W_{1} \log 2 \pi \epsilon Q - W_{1} \log 2 \pi \epsilon N \end{equation}
\begin{equation} = W_{1} \log \frac{Q}{N} \end{equation}
donde $Q$ es la potencia promedio del mensaje. Esto comprueba lo
siguiente:

\begin{theorem}
La tasa para la medici\'on de fidelidad de una fuente de ruido blanco
de potencia $Q$ y banda $W_{1}$ relativa a un R.M.S. es:
\begin{equation} R = W_{1} \log \frac{Q}{N} \end{equation}
donde $N$ es es el cuadradio medio del error permitido entre el
mensaje original y el recuperado.
\end{theorem}

M\'as generalmente, con cualquier fuente de mensaje podemos obtener
desigualdades delimitando la tasa a un criterio de cuadrado medio del
error.

\begin{theorem}
La tasa para cualquier fuente de banda $W_{1}$ es delimitada por:
\begin{equation} W_{1} \log Q1/N <= R <= W_{1} \log Q/N \end{equation}
donde $Q$ es la potencia promedio de la fuente, $Q_{1}$ la energ\'ia
de entrop\'ia y $N$ el cuadrado medio del error permitido.
\end{theorem}

El limite inferior sigue el hecho de que la $\max H_{y}(x)$ para un
$\overline{(x - y)^{2}} = N$ dado ocurre en el caso de ruido
blanco. El limite superior resulta si colocamos puntos (usados en la
comprobaci\'on del Teorema \ref{FALTA LA ETIQUETA}) no en la mejor
forma sino al azar en una esfera de radio $\sqrt{(Q - N)}$.

\clearpage

\chapter*{Agradecimientos}
\addcontentsline{toc}{chapter}{Agradecimientos}

El escritor est\'{a} en deuda con sus colegas en los laboratorio,
particularmente al Dr.\ H.\ W.\ Bode, Dr.\ J.\ R.\ Pierce,
Dr.\ B.\ McMillan y al Dr.\ B.\ M.\ Oliver, por muchas sugerencias y
criticismos \'utiles durante el curso de su trabajo. Cr\'{e}dito debe
tambi\'en ser otorgado al Profesor N.\ Wiener, cuya soluci\'on
elegante al problema de filtraci\'on y predicci\'on de conjuntos
estacionarios ha influido considerablemente la forma de pensar del
escritor en este campo de estudio.

\clearpage

\begin{appendices}

\chapter{}
\label{a:5}

Sea $S_{1}$ cualquier subconjunto medible del conjunto $g$, y $S_{2}$
el subconjunto del conjunto $f$ el cual da $S_{1}$ bajo la operaci\'on
$T$. Entonces
\begin{equation} S_{1} = TS_{2}.\end{equation}

Sea $H^{\lambda}$ el operador que desplaza todas las funciones en un
conjunto con tiempo $\lambda$. Entonces
\begin{equation} 
H^{\lambda} S_{1} = H^{\lambda} T S_{2} = T H^{\lambda} S_{2}. 
\end{equation}
debido a que $T$ es invariante y por lo tanto conmuta con
$H^{\lambda}$. Por lo tanto, si $m[S]$ es la probabilidad de
medici\'on del conjunto $S$,
\begin{equation} 
\begin{array}{rcl}
m[H^{\lambda} S_{1}] &=& m[T H^{\lambda} S_{2}] = m[H^{\lambda} S_{2}] \\
&=& m[S_{2}] = m[S_{1}]
\end{array}
\end{equation}
donde la segunda igualdad es por definici\'on la medici\'on del
espacio $g$, el tercero ya que el conjunto $f$ es estacionario, y el
\'ultimo nuevamente por definici\'on de la medici\'on de $g$.

Para probar que la propiedad ergódica es preservada bajo operaciones
invariantes, sea $S_{1}$ un subconjunto del conjunto $g$, el cual es
invariante bajo $H^{\lambda}$, y sea $S_{2}$ el conjunto de todas las
funciones $f$ que se transforman en $S_{1}$. Entonces
\begin{equation} H^{\lambda} S_{1} = H^{\lambda} T S_{2} = T H^{\lambda} S_{2} = S_{1} \end{equation}
as\'{\i} que $H^{\lambda} S_{2}$ es incluida en $S_{2}$ para todas las
$\lambda$. Ahora, debido a que
\begin{equation} m[H^{\lambda} S_{2}] = m[S_{1}] \end{equation}
esto implica
\begin{equation} H^{\lambda} S_{2} = S_{2} \end{equation}
para todo $\lambda$ con $m[S_{2}] \neq 0, 1$. Esta contradicci\'on muestra que $S_{1}$ no existe.

\chapter{}
\label{a:6}

El l\'{\i}mite superior, $\overline{N_{3}} \leq N_{1} + N_{2}$, se
debe al hecho que la m\'axima entrop\'ia posible para la potencia
$N_{1} + N_{2}$ ocurre cuando tenemos ruido blanco de esta
potencia. En este caso la energ\'ia de entrop\'ia es $N_{1} + N_{2}$.

Para obtener el l\'imite inferior, suponga que tenemos dos
distribuciones en $n$ dimensiones $p(x_{i})$ y $q(x_{i})$ con
energ\'ias de entrop\'ia $\overline{N_{1}}$ y $\overline{N_{2}}$. Que
forma deber\'ia $p$ y $q$ tener para poder minimizar la energ\'ia de
entrop\'ia $\overline{N_{3}}$ de su convoluci\'on $r(x_{i})$:
\begin{equation} r(x_{i}) = \int p(y_{i}) q(x_{i} - y_{i}) \diff y _{i}.
\end{equation}
			
La entrop\'ia $H_{3}$ de $r$ es dada por:
\begin{equation} H_{3} = - \int r(x_{i}) \log r(x_{i}) \diff x _{i}. \end{equation}
			
Deseamos minimizar esto sujeto a las restricciones:
\begin{equation} H_{1} = - \int p(x_{i}) \log p(x_{i}) \diff x _{i} \end{equation}
\begin{equation} H_{2} = - \int q(x_{i}) \log q(x_{i}) \diff x _{i}. \end{equation}
				
Consideramos entonces
\begin{equation} U = - \int [r(x) \log r(x) + 
\lambda p(x) \log p(x) + \mu q(x) \log q(x)] \diff x \end{equation}
\begin{equation} \delta U = - \int [[1 + \log r(x)] 
\delta r(x) + \lambda [1 + \log p(x) \delta p(x) + 
\mu [1 + \log q(x)]\delta q(x)] \diff x \end{equation}

Si $p(x)$ es variado en un argumento particular $x_{i} = s_{i}$, la variaci\'on en $r(x)$ es
\begin{equation} \delta r(x) = q(x_{i} - s_{i}) \end{equation}
y
\begin{equation} 
\delta U = - \int q (x_{i} - si_{i}) \log r(x_{i}) \diff x _{i} - \lambda \log p(s_{i}) = 0 
\end{equation}
y similarmente cuando $q$ es variado. Entonces las condiciones para un m\'inimo son:
\begin{equation} \int q(x_{i} - s_{i}) \log r(x_{i}) \diff x _{i} 
= -\lambda \log p(s_{i}) \end{equation}
				
\begin{equation} \int p(x_{i} - s_{i}) \log r(x_{i}) \diff x _{i} = -\mu \log q(s_{i}) \end{equation}
				
Si multiplicamos el primero por $p(s_{i})$ y el segundo por $q(s_{i})$ e integramos con 
respecto a $s_{i}$, obtenemos:

\begin{equation} H_{3} = -\lambda H_{1} \end{equation}
\begin{equation} H_{3} = -\mu H_{2} \end{equation}
o resolviendo $\lambda$ y $\mu$, y reemplazando en las ecuaciones
\begin{equation} H_{1} \int q(x_{i} - s_{i}) \log r(x_{i}) \diff x _{i} 
= - H_{3} \log p(s_{i}) 
\end{equation}
\begin{equation} H_{2} \int p(x_{i} - s_{i}) \log r(x_{i}) \diff x _{i} 
= - H_{3} \log q(s_{i}) \end{equation}

Ahora supongamos que $p(x_{i})$ y $q(x_{i})$ son normales
\begin{equation} p(x_{i}) = \frac{|A_{ij}^{\frac{n}{2}}|}{(2\pi)^{\frac{n}{2}}} exp - \frac{1}{2} \Sigma(A_{ij} x_{i} x_{j}) \end{equation}
\begin{equation} q(x_{i}) = \frac{|B_{ij}^{\frac{n}{2}}|}{(2\pi)^{\frac{n}{2}}} exp - \frac{1}{2} \Sigma(B_{ij} x_{i} x_{j}) \end{equation}

Entonces $r(x_{i})$ puede tambi\'en ser normal con la forma
cuadr\'atica $C_{ij}$. Si las inversas de \'estas formas son $a_{ij}$,
$b_{ij}$ y $c_{ij}$, entonces
\begin{equation} c_{ij} = a_{ij} + b_{ij}. \end{equation}
				
Deseamos mostrar que estas funciones satisfacen las condiciones de
minimizaci\'on s\'i y solo si $a_{ij} = Kb_{ij}$ y por lo tanto da el
m\'inimo $H_{3}$ bajo las restricciones. Primero tenemos
\begin{equation} \log r(x_{i}) = 
\frac{n}{2} \log \frac{1}{2\pi} |C_{ij}| - \frac{1}{2} 
\Sigma(C_{ij} x_{i} x_{j}) \end{equation}
			
\begin{equation} 
\int q(x_{i} - s_{i}) \log r(x_{i}) \diff x _{i} = \frac{n}{2} \log
\frac{1}{2\pi} |C_{ij}| - \frac{1}{2} \Sigma(C_{ij}S_{i}S_{j}) -
\frac{1}{2} \Sigma(C_{ij} B_{ij}) \end{equation}
		
Esto deberia ser equivalente a
\begin{equation} \frac{H_{3}}{H_{1}} [ \frac{n}{2} \log \frac{1}{2\pi} 
| A_{ij} | - \frac{1}{2} \Sigma(A_{ij} S_{i}S_{j})] \end{equation}
lo cual requiere $A_{ij} = H_{1}/H_{3} C_{ij}$. En este caso $A_{ij} = H_{1}/H_{2} B_{ij}$ y ambs ecuaciones
se reducen a identidades.

\chapter{}
\label{a:7}

Lo siguiente indicar\'a un acercamiento m\'as general y riguroso a las
definiciones centrales de teor\'ia de la comunicaci\'on. Consideremos
un espacio de medici\'on de probabilidad cuyos elementos est\'an
ordenados en pares $(x, y)$. Las variables $x$, y ser\'an
identificadas como de todos los puntos cuyos $x$ pertenecen al sub
conjunto Si las posibles señales transmitidas y recibidas en una
larga duraci\'on $T$. Llamaremos al conjunto de todos los puntos cuyas
$x$ pertenecen a un sub conjunt $S_{1}$ de puntos $x$: la tira sobre
$S_{1}$, y similarmente al conjunto cuyas y pertenecen a $S_{2}$, la
tira sobre $S_{2}$. Dividimos $x$ y $y$ en una colecci\'on de
subconjuntos medibles no superpuestos $X_{i}$ y $Y_{i}$, aproximado a
la tasa de transmisi\'on $R$ por

\begin{equation} R_{1} = \frac{1}{T} \displaystyle\sum_{i}(P(X_{i},Y_{i})
\log \frac{P(X_{i}, Y_{i})}{P(x_{i})P(Y_{i})} \end{equation}
donde
\begin{itemize}
\item $P(x_{i})$ es la probabilidad de medici\'on de la tira sobre
 $X_{i}$
\item $P(Y_{i})$ es la probabilidad de medici\'on de la tira sobre
 $Y_{i}$
\item $P(X_{i},Y_{i})$ es la probabilidad de medici\'on dela
 interseccion de las tiras.
\end{itemize}
		
Una subdivisi\'on adicional no puede disminuir $R_{1}$ nunca. Dejemos
que $X_{1}$ sea dividido en $X_{1} = X_{1}' + X_{1}''$ y sea
\begin{equation} 
\begin{array}{rclrcl}
P(Y_{1}) &=& a &
P(X_{1}) &=& b + c \\
P(X_{1}') &=& b &
P(X_{1}', Y_{1}) &=& d \\
P(X_{1}'') &=& c &
P(X_{1}'', Y_{1}) &=& e \\
P(X_{1}, Y_{1}) &=& d + e &&&
\end{array}
\end{equation}
					
Entonces en la suma hemos reemplazado (para la intersecci\'on $X_{1}, Y_{1}$)
\begin{equation} (d + e) \log 
\frac{(d + e)}{a(b + c)}  por d \log \frac{d}{ab} 
+ e \log \frac{e}{ac}. \end{equation}
Es f\'acilmente mostrado que con la limitaci\'on que tenemos en $b, c,
d, e$,
\begin{equation} 
[\frac{d + e}{b + c}]^{d+e} \leq \frac{d{d} e{e}}{b^{d} c^{e}} \end{equation}
y consecuentemente la suma es incrementada. Por lo tanto las varias formas posibles
de subdivisi\'on formanun conjunto dirigido, con $R$ incrementandose monot\'onicamente
con el refinamiento de la subdivisi\'on. Podemos definir $R$ sin ambigüedad como el
menor l\'imite superior para $R_{1}$ y escribirlo
\begin{equation} R = \frac{1}{T} \int \int P(x,y) \log 
\frac{P(x,y)}{P(x) P(y)} \diff x  \diff y \end{equation}
			
La integral, entendida en el sentido anterior, incluye ambos los casos
continus y discretos, y por supuesto, muchos otros que no pueden ser
representados en cualquiera de las formas. Es trivial en esta
formulaci\'on que si $x$ y $u$ est\'an en correspondencia uno a uno,
la tasa de $u$ a $y$ es equivalente a aquella entre $x$ y $y$. Si $v$
es cualquier funci\'on de $y$ (no necesariamente con una inversa)
entonces la tasa desde $x$ a $y$ es mayor o igual a aquella entre $x$
a $v$ debido a que, en el c\'alculo de las aproximaciones, las
subdivisiones de $y$ son escencialmente subdivisiones m\'as finas que
aquellas para $v$. M\'as generalmente si $y$ y $v$ est\'an
relacionadas, no funcionalmente pero estad\'isticamente, por ejemplo
si tenemos un espacio de medida de probabilidad $(y, v)$, entonces
$R(x,v) \leq R(x,y)$. Esto significa que cualquier operaci\'on
aplicada a la señal recibida, aunque involucre elementos
estad\'isticos, no incrementa $R$.

Otra noci\'on que debe ser definida precisamente en una formulaci\'on
abstracta de la teor\'ia, es "la tasa de dimensi\'on", que es el
n\'umero promedio de dimensiones por segundo requeridas para
especificar a un miembro de un conjunto. En el caso de una banda
limitada con $2W$ n\'umeros por segundo son suficientes. Una
definici\'on general puede ser enmarcada como sigue. Sea
$f_{\alpha}(t)$ un conjunto de funciones y sea $\rho_{\tau} \left
[f_{\alpha}(t),f_{\beta}(t) \right ]$ una m\'etrica midiendo la forma
de "distancia" desde $f_{\alpha}$ hasta $f_{\beta}$ sobre el tiempo
$T$(por ejemplo la discrepancia R.M.S. sobre \'este intervalo). Sea
$N(\varepsilon, \delta, \tau)$ el menor n\'umero de elementos $f$
que pueden ser elegidos, as\'i que todos los elementos del conjunto
adem\'as de un conjunto de medici\'on $\delta$, est\'an dentro de
distancia $\varepsilon$ de por lo menos uno de los escogidos.

Por lo tanto, estamos cubriendo el espacio dentro de $\varepsilon$
separado de un conjunto de poca medida $\delta$. Definimos la tasa de
dimensi\'on $\lambda$ para el conjunto, por el triple l\'imite
\begin{equation} 
\lambda = 
\lim_{\delta \to \infty} 
\lim_{\varepsilon \to \infty} 
\lim_{\tau \to \infty} 
\frac{\log N(\varepsilon, \delta, \tau)}{\tau \log \varepsilon} 
\end{equation} 
Esta es una generalizaci\'on de las definiciones de tipo de medida de
la dimensi\'on en la topolog\'ia, y est\'a de acuerdo con la tasa de
dimensi\'on intuitiva para conjuntos simples donde los resultados
deseados son obvios.

\end{appendices}
