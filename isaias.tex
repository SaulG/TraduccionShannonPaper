4. Graphical Representation of a Markoff Process

Los procesos estocasticos del tipo descrito arriba son matematicamente conocidos como Procesos Markovianos Discretos y han sido estudiados extensivamente en la literatura.^{6} El caso general puede ser descrito de la siguiente manera: Existe un numero finito de posibles "estados" de un sistema; S_{1}, S_{2}, ...,S_{n}. Adem\'{a}s existen un conjunto de probabilidades de transicion; p_{i}(j) la probabilidad que si el sistema esta en estado S_{i} entonces enseguida vaya al estado S_{j}. Para realizar este proceso Markoviano en una fuente de información solo necesitamos asumir que una letra es producida para cada transicion desde un estado a otro. Los estados corresponderán al "residuo de influencia" de letras precedentes.

\begin{equation}
b = a - 2
\end{equation}
