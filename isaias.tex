\section{Representacion grafica de procesos Markovianos}

Los procesos estocasticos del tipo descrito arriba son matematicamente
conocidos como Procesos Markovianos Discretos y han sido estudiados
extensivamente en la literatura \footnote{AQUI}. El caso general puede
ser descrito de la siguiente manera: Existe un numero finito de
posibles "estados" de un sistema; $S_{1}, S_{2}, \ldots,
S_{n}$. Adem\'{a}s existen un conjunto de probabilidades de
transicion; $p_{i}(j)$ la probabilidad que si el sistema esta en
estado $S_{i}$ entonces enseguida vaya al estado $S_{j}$. Para
realizar este proceso Markoviano en una fuente de informaci\'{o}n solo
necesitamos asumir que una letra es producida para cada transicion
desde un estado a otro. Los estados corresponder\'{a}n al ``residuo de
influencia'' de letras precedentes.

La situacion puede ser representada graficamente como se muestra en
las figuras 3, 4 y 5. Los ``estados'' son los puntos de union

% aqui va figura 3

en la grafica y las probabilidades y letras son producidas para una
transicion son dadas ademas de la linea correspondiente. La figura 3
es para el ejemplo B en la seccion 2, mientras que la figura 4
corresponde al ejemplo C. En la figura 3

%figura 4

solamente hay un estado ya que letras sucesivas son independientes. En
la figura 4 hay tantos estados como letras. \newline Si un ejemplo de
un triagrama fuera construido, habr\'{i}a por maximo $n^{2}$ estados
correspondiendo a los posibles pares de letras precediendo a uno que
haya sido elegido. La figura 5 es un grafo para el caso de estructura
de palabras en el ejemplo D. Aqui $S$ corresponde a el simbolo
``espacio''. \newline

\section{Fuentes erg\'{o}dicas y mixtas}

Como se ha indicado anteriormente, una fuente discreta para nuestros
propositos puede ser considerada representada por un proceso
Markoviano. Entre los posibles procesos discretos Markovianos existe
un grupo con propiedades especiales con importancia en la teoria de la
comunicacion. Esta clase especial consiste en los procesos
``ergodicos'' y deberiamos de llamar a las fuentes correspondientes,
fuentes ergodicas. Aunque una definicion rigurosa de los procesos
ergodicos es algo complicada, la idea general es simple. En un proceso
ergodico cada secuencia producida por el proceso permanece igual en
sus propiedades estadisticas. Por lo tanto las frecuencias de letras,
las frecuencias de bigramas, etc., obtenidos de una secuencia en
particulas, se acercaran a un limite definido conforme la longitud de
las secuencua aumenta, independientemente de la secuencia en
particular. En realidad esto no es meramente cierto para cada
secuencia pero el grupo para el cual esto es falso tiene una
probabilidad de cero. Practicamente, la propiedad ergodica significa
homogeneidad estadistica.

Todos los ejemplos de lenguaje artificial dados anteriormente son
ergodicos. Esta propiedad est\'{a} relacionada a la estructura de los
grafos correspondientes. Si el grafo tiene las siguientes dos
propiedades el proceso correspondiente ser\'{a} erg\'{o}dico:

\begin{enumerate}
   \item El grafo no consiste de dos partes aisladas $A$ y $B$ dado que es
   imposible ir desde los puntos de union en la parte $A$ a los puntos
   de union en la parte $B$ a traves de las lineas del grafo en la
   direccion de las flechas y tambien es imposible ir desde las
   uniones en la parte $B$ a las uniones en la parte $A$.  \item
\end{enumerate}

\begin{equation}
b = a - 2 \text{ (ejemplo)}
\end{equation}
