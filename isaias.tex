\section{Representacion grafica de procesos Markovianos}

Los procesos estocasticos del tipo descrito arriba son matematicamente
conocidos como Procesos Markovianos Discretos y han sido estudiados
extensivamente en la literatura \footnote{Para una explicaci\'{o}n detallada
ver M. Fr\'{e}chet, M\'{e}thode des fonctions arbitraires. Th\'{e}orie des \'{e}v\'{e}nements
en chaine dans le cas d'un nombre fini d'\'{e}tats possibles. Paris, Gauthier-Villars, 1938.}.
El caso general puede ser descrito de la siguiente manera: Existe un numero finito de
posibles ``estados'' de un sistema; $S_{1}, S_{2}, \ldots,
S_{n}$. Adem\'{a}s existen un conjunto de probabilidades de
transicion; $p_{i}(j)$ la probabilidad que si el sistema esta en
estado $S_{i}$ entonces enseguida vaya al estado $S_{j}$. Para
realizar este proceso Markoviano en una fuente de informaci\'{o}n solo
necesitamos asumir que una letra es producida para cada transicion
desde un estado a otro. Los estados corresponder\'{a}n al ``residuo de
influencia'' de letras precedentes.

La situacion puede ser representada graficamente como se muestra en
las figuras 3, 4 y 5. Los ``estados'' son los puntos de union

% aqui va figura 3

en la grafica y las probabilidades y letras son producidas para una
transicion son dadas ademas de la linea correspondiente. La figura 3
es para el ejemplo B en la seccion 2, mientras que la figura 4
corresponde al ejemplo C. En la figura 3

%figura 4

solamente hay un estado ya que letras sucesivas son independientes. En
la figura 4 hay tantos estados como letras. 

Si un ejemplo de un triagrama fuera construido, habr\'{i}a por maximo
$n^{2}$ estados correspondiendo a los posibles pares de letras
precediendo a uno que haya sido elegido. La figura 5 es un grafo para
el caso de estructura de palabras en el ejemplo D. Aqui $S$
corresponde a el simbolo ``espacio''. 

\section{Fuentes erg\'{o}dicas y mixtas}

Como se ha indicado anteriormente, una fuente discreta para nuestros
propositos puede ser considerada representada por un proceso
Markoviano. Entre los posibles procesos discretos Markovianos existe
un grupo con propiedades especiales con importancia en la teoria de la
comunicacion. Esta clase especial consiste en los procesos
``ergodicos'' y deberiamos de llamar a las fuentes correspondientes,
fuentes ergodicas. Aunque una definicion rigurosa de los procesos
ergodicos es algo complicada, la idea general es simple. En un proceso
ergodico cada secuencia producida por el proceso permanece igual en
sus propiedades estadisticas. Por lo tanto las frecuencias de letras,
las frecuencias de bigramas, etc., obtenidos de una secuencia en
particulas, se acercaran a un limite definido conforme la longitud de
las secuencua aumenta, independientemente de la secuencia en
particular. En realidad esto no es meramente cierto para cada
secuencia pero el grupo para el cual esto es falso tiene una
probabilidad de cero. Practicamente, la propiedad ergodica significa
homogeneidad estadistica.

Todos los ejemplos de lenguaje artificial dados anteriormente son
ergodicos. Esta propiedad est\'{a} relacionada a la estructura de los
grafos correspondientes. Si el grafo tiene las siguientes dos
propiedades \footnote{Estas son reformulaciones en t'{e}rminos de la gr\'{a}fica de las condiciones dadas en Fr\'{e}chet.} el proceso correspondiente ser\'{a} erg\'{o}dico:

\begin{enumerate}
  \item El grafo no consiste de dos partes aisladas $A$ y $B$ dado que es
   imposible ir desde los puntos de union en la parte $A$ a los puntos
   de union en la parte $B$ a traves de las lineas del grafo en la
   direccion de las flechas y tambien es imposible ir desde las
   uniones en la parte $B$ a las uniones en la parte $A$.
 \item Una serie de lineas cerradas en un grafo con todas sus flechas en las lineas
   apuntando en la misma direccion son llamados ``circuitos''.
   La ``longitud'' de un circuito es el numero de lineas en el.
   Por lo tanto figura 5, la serie BEBES es un circuito de longitud 5.
   La segunda propiedad requerida es que el maximo comun divisor
   de la longitud de todos los circuitos en el grafo sea igual a uno.
\end{enumerate}

%figura 5

Si la primera condicion es satisfecha pero la segunda no por tener
un maximo comun divisor igual a $d > 1$, las secuencias tiene algun tipo
de estructura periodica. Las diferentes secuencias caen dentro $d$ clases diferentes
que son estadisticamente las mismas partiendo desde un cambio
del origen (por ejemplo, que letra en la secuencia es llamada letra 1). Por un cambio
de desde $0$ hasta $d - 1$ cualquier secuencia puede ser
estadisticamente equivalente a cualquier otra. Un ejempl simple
con $d = 2$ es el siguiente: Existen tres posibles letras $a, b , c$.
La letra $a$ es seguida con $b$ \'{o} $c$ con
probabilidades $\frac{1}{3}$ y $\frac{2}{3}$ respectivamente.
Tanto $b$ como $c$ son siempre seguidas por una letra $a$. Por lo tanto una
secuencia tipica ser\'{i}a

\begin{equation}
a b a c a c a c a b a c a b a b a c a c
\end{equation}

Este tipo de situacion no es de mucha importancia para nuestro
trabajo.  Si la primera condici\'{o}n no se cumple, el grafo puede ser
separado en un conjunto de subgrafos que satisfagan cada uno la
primera condici\'{o}n.  Nosotros asumiremos que la segunda condicion
es tambien satisfecha para cada subgrafo.  Tenemos en este caso lo que
se le llama una fuente ``mezclada'' hecha de un numero de componentes
puros. Estos componentes corresponden a los diversos subgrafos. Si
$L_{1}, L_{2}, L_{3}, \cdots$ son los componentes fuente entonces
podemos escribir
\begin{equation}
L = p_{1}L_{1} + p_{2}L_{2} + p_{3}L_{3} + \ldots,
\end{equation}
donde $p_{i}$ es la probabilidad del componente fuente $L_{i}$.
Fisicamente la situacion representada es esta: Hay diferentes fuentes
$L_{1}, L_{2}, L_{3}, \cdots$ que son cada uno de una estructura
estadistica homogenea (por ejemplo, son erg\'{o}dicos). No sabemos
\textit{a priori} cual será utilizada, pero una vez que la secuencia
empieza en un componente puro dado $L_{i}$, este continua
indefinidamente de acuerdo a la estructura estadistica de ese
componente.  Como un ejemplo, uno puede tomar dos de estos procesos
definidos arriba y asumir $p_{1} = .2$ y $p_{2} = .8$. Una secuencia
de la fuente mezclada

\begin{equation}
L = .2L_{1} + .8L_{2}
\end{equation}

ser\'{i}a obtenida escogiendo primero $L_{1}$ o $L_{2}$ con
probabilidades .2 y .8 y despues de esta elecci\'{o}n generando una
secuenca de lo que sea haya sido elegido.  Excepto cuando se exprese
lo contrario nosotros debemos asumir que una fuente es
erg\'{o}dica. Esta asunci\'{o}n permite a uno identificar los
promedios a lo largo de la secuencia con promedios encima del conjunto
de posibles secuencias (la probabilidad de una discrepancia sea
cero). Por ejemplo la frecuencia relativa de la letra A en una
secuencia infinita particular ser\'{i}a, con la probabilidad uno,
igual a su frecuencia relativa en el conjunto de secuencias.  Si
$P_{i}$ es la probabilidad de un estado $i$ y $p_{i}(j)$ la
probabilidad de transicion de un estado $j$, entonces para que el
proceso sea estacionario es claro que $P_{i}$ debe satisfacer
condiciones de equilibrio:

\begin{equation}
P_{j} = \sum P_{i}p_{i}(j).
\end{equation}

En el caso erg\'{o}dico este puede ser demostrado que con cualquier condici\'{o}n inicial las
probabilidades $P_{j}(N)$ de estar en un estado $j$ despues de $N$ simbolos, se aproximan al
equilibrio de valores conforme $N \rightarrow \infty$.

\section{Selecci\'{o}n, incertidumbre y entropia.}

Hemos presenteado una fuente de informaci\'{o}n discreta como un
proceso Markoviano. Podremos definir una cantidad que mida, en algun
sentido, que tanta informaci\'{o}n es ``producida'' por estos
procesos, o mejor aun, a que tasa de informaci\'{o}n es producida?
Suponiendo que tenemos un conjunto de posibles eventos cuya
probabilidad de ocurrir es $p_{1}, p_{2}, \ldots, p_{n}$. Estas
probabilidades son conocidas pero eso es todo lo que conocemos en
cuanto a lo que concierne a que evento ocurrir\'{a}. ?'Podremos
encontrar una medidad de cuanta ``opcion'' est\'{a} involucrada en la
seleccion de un evento o de que tan incierto pudiera ser la salida?
En caso de que exista dicha medida, $H(p_{1}, p_{2}, \ldots, p_{n})$,
es razonable el requerir de esta las siguientes propiedades:

\begin{enumerate}
\item $H$ debe ser continuo en $p_{i}$.
\item Si todas las $p_{i}$ son iguales, $p_{i} = \frac{1}{n}$, entonces $H$ debe ser una funcion de incremento monotona de n. Con eventos igualmente probables hay mas opciones, o incertidumbre, cuando hay mas eventos posibles.
\item Si una opcion se desglosara en dos opciones, la H original deber\'{i}a ser la suma ponderada de los valores individuales de H. El significado de esto es ilustrado en la figura 6. A la izquierda tenemos

%Figura 6 aqui

tres posibilidades $p_{1} = \frac{1}{2}, p_{2} = \frac{1}{3}, p_{3} = \frac{1}{6}$. A la derecha primero se escoge enntre las dos posibilidades cada una con probabilidad $\frac{1}{2}$, y si la segunda ocurra hacer otra seleccion con probabilidades $\frac{2}{3}, \frac{1}{3}$. Los resultados finales tienen las mismas probabilidades que antes. Requerimos, en este caso en especial, que

\begin{equation}
H(\frac{1}{2},\frac{1}{3},\frac{1}{6}) = H(\frac{1}{2},\frac{1}{2}) + \frac{1}{2}H(\frac{2}{3},\frac{1}{3})
\end{equation}

El coeficiente $\frac{1}{2}$ es porque esta segunda eleccion solo ocurre la mitad del tiempo.
En el apendice 2, el siguiente resultado es establecido:
\textit{Theorem 2}: La unica $H$ que satisface las tres suposiciones de arriba tiene la forma:

\begin{equation}
H = -K \sum{i=1}^{n} p_{i} \log p_{i}
\end{equation}
donde K es una constante positiva.
\end{enumerate}

Este teorema, y las suposiciones requeridas para su demostracion,
no son en ninguna manera necesarias para la teoria de la
que se est\'{a} hablando.
Se da principalmente para dar verosimilitud a algunas
definiciones posteriores. La justificacion real de estas
definiciones, sin embargo, residira en sus implicaciones.
Cantidades de la forma $H = -\sum p_{i} \log p_{i}$ (la constante
viene a ser una opcion de una unidad de medida) juegan un papel
principal en la teoria de informacion como medidas de
informacion, opciones e incertidumbre. La forma de $H$ ser\'{a}
reconocida como de entrop\'{i}a como se define en ciertas formulaciones
de estadistica mec\'{a}nica\footnote{Ver, por ejemplo, R. C. Tolman, \textit{Principles of Statistical Mechanics}, Oxford, Clarendon, 1938}
donde $p_{i}$ es la probabilidad de un sistema de estar en
una celda $i$ de su espacio de fase. $H$ es entonces, por ejemplo,
la $H$ en el famoso teorema $H$ de Boltzmann. Debemos llamar
$H = -\sum p_{i} \log{p} p_{i}$ la entrop\'{i}a del conjunto de
probabilidades $p_{1}, \cdots,P_{n}$. Si $x$ es una variable
de probabilidad escribiriamos $H(x)$ para su entrop\'{i}a; por
lo tanto $x$ no es un argumento de una funci\'{o}n sino
una manera de representar un numero, para diferenciar de
$H(y)$ se dir\'{i}a la entrop\'{i}a de la variable de probabilidad $y$.
La entropia en este caso de dos posibilidades con probabilidades $p$
y $q = 1 - p$, es decir

\begin{equation}
H = -(p \log p + q \log q)
\end{equation}

es graficado en la figura 7 como una funci\'{o}n de p.

% figura 7 aqui

La cantidad $H$ tiene un numero interesante de propiedades que
adem\'{a}s se sustentan como una medida razonable de opcion o informaci\'{o}n.
\begin{enumerate}
\item $H = 0$ si y solo si todas las $p_{i}$ excepto una son cero, esta teniendo el valor unitario. As\'{i} solamente cuando estemos seguros $H$ desaparece. De lo contrario $H$ es positivo.
\item Dado un $n$, $H$ es un valor maximo igual a $\log n$ cuando todos los $p_{i}$ son iguales (por ejemplo, $\frac{1}{n}$). Esto tambien es intuitivamente la situaci\'{o}n mas incierta.
\item Suponiendo que hay dos eventos, $x$ y $y$, en cuestion con $m$ posibilidades para el primero y $n$ para el segundo. Sea $p(i,j)$ la probabilidad de la union de que ocurra $i$ para el primero y $j$ para el segundo. La entrop\'{i}a de la union de los eventos es
\begin{equation}
H(x,y) = -\sum{i,j}{}p(i,j)\log p(i,j)
\end{equation}
mientras
\begin{equation}
H(x,y) = -\sum{i,j}{}p(i,j)\log\sum{j}{}p(i,j)
\end{equation}
\begin{equation}
H(x,y) = -\sum{i,j}{}p(i,j)\log\sum{i}{}p(i,j)
\end{equation}
Es facilmente demostrable que
\begin{equation}
H(x,y) \leq H(x) + H(y)
\end{equation}
con igualdad solo si los eventos son independientes (por ejemplo, p(i)p(j)). La incertidumbre de la union de eventos es menor que o igual a la suma de las incertidumbres individuales.
\item Cualquier cambio hacia la ecualizacion de las probabilidades  $p_{1}, p_{2}, \ldots, p_{n}$ incrementa $H$. Por lo tanto, si $p_{1} < p_{2}$ y incrementamos $p_{1}$, decrementando $p_{2}$ una cantidad igual de manera que $p_{1}$ y $p_{2}$ sean mas cercanos, entonces $H$ incrementa. De una manera mas general, si ejecutamos de nuevo alguna operacion ``promediante'' en $p_{i}$ de la forma
\begin{equation}
p^{'}_{i} = \sum{j}{} a_{ij}p_{j}
\end{equation}
donde $\sum_{i} a_{ij} = \sum_{j} a_{ij} = 1$, y todo $a_{ij} \leq 0$, entonces H incrementa (excepto in el caso especial donde esta transformacion de cantidades a no mas de $p_{j}$ permutaciones con $H$ permaneciendo igual).
\item Suponiendo que hay dos oportunidades de eventos $x$ y $y$ como en 3, no necesariamente independiente. Para cada valor particular $i$ que $x$ puede tener hay una probabilidad condicional $p_{i}(j)$ que $y$ tenga el valor $j$. Esto est\'{a} dado por
\begin{equation}
p_{i}(j) = \frac{p(i,j)}{\sum_{j}p(i,j)}
\end{equation}
Definimos la \textit{entrop\'{i}a condicional} de $y$, $H_{x}(y)$ como el promedio de entropia de $y$ para cada valor de $x$, ponderado de acuerdo a la probabilidad de obtener ese $x$ en particular. Esto es
\begin{equation}
H_{x}(y) = -\sum{i,j}{}p(i,j)logp_{i}(j)
\end{equation}
Esta cantidad mide que tan incierto se est\'{a} de $y$ en promedio cuando se conoce $x$. Substituyendo el valor de $p_{i}(j)$ obtenemos
\begin{equation}
H_{x}(y) = -\sum{i,j}{}p(i,j) \log p(i,j) + \sum{i,j}{}p(i,j)\log\sum{j}{}p(i,j) = H(x,y) - H(x)
\end{equation}
o tambien
\begin{equation}
H(x,y) = H(x) + H_{x}(y)
\end{equation}
La incertidumbre (o entropia) del evento conjunto $x,y$ es la incertidumbre de $x$ mas la incertidumbre de $y$ cuando $x$ es conocido.
\item Desde 3 y 5 tenemos
\begin{equation}
H(x) + H(y) \leq H(x,y) \ H(x) + H_{x}(y)
\end{equation}
Por lo tanto
\begin{equation}
H(y) \leq H_{x}(y)
\end{equation}
La incertidumbre de $y$ nunca es incrementada conociendo $x$. Esta ser\'{a} decrementada a menos que $x$ y $y$ sean eventos independientes, en cuyo caso no se cambia.
\end{enumerate}

\section{La entrop\'{i}a de una fuente de inormaci\'{o}n}

Considere una fuente discreta del tipo estado finito
considerado arriba. Para cada posible estado $i$ existir\'{a} un
conjunto de probabilidades $p_{i}(j$ resultado de producir
los diferentes simbolos $j$. Por lo tanto existe una entrop\'{i}a $H_{i}$
para cada estado. La entropia de la fuente se definir\'{a} como
el promedio de estas $H_{i}$ ponderadas de acuerdo con la probabilidad
de ocurrencia de los estados en cuesti\'{o}n:

\begin{equation}
H = \sum{i}{} P_{i}H_{i}
  = - \sum{i,j}{} P_{i}p_{i}(j) \log p_{i}(j)
\end{equation}

Esta es la entrop\'{i}a de la fuente por simbolo de texto. Si el proceso
Markoviano est\'{a} procediendo a una taza de tiempo definido hay tambien
una entrop\'{i}a por segundo.

\begin{equation}
H' = \sum{i}{} f_{i}H_{i}
\end{equation}

donde $f_{i}$ es la frecuencia promedio (sucesos por segundo) del estado $i$. Claramente

\begin{equation}
H' = mH
\end{equation}

donde $m$ es el numero promedio de simbolos producidos por
segundo. $H$ o $H'$ mide la cantidad de informaci\'{o}n generada por
la fuente por simbolo o por segundo. Si la base logaritmica es 2,
representar\'{a}n bits por simbolo o por segundo.  Si simbolos
sucesivos son independientes entonces $H$ es simplemente $-\sum p_{i}
\log p_{i}$ donde $p_{i}$ es la probabilidad de simbolo
$i$. Suponiendo que en este caso se considere un mensaje largo de $N$
simbolos. Con mucha probabilidad este contendr\'{i}a alrededor de
$p_{i}N$ ocurrencias del primer simbolo, $p_{2}N$ ocurrencias del
segundo, etc. Por lo tanto la probabilidad en particular de este
mensaje ser\'{i}a aproximadamente
\begin{equation}
p = p_{1}^{ p_{1}^{N}} p_{2}^{ p_{2}^{N}} \ldots p_{n}^{ p_{n}^{N}}
\end{equation}
o
\begin{equation}
\log p \doteq N \sum{i}{} p_{i} \log p_{i}
\end{equation}
\begin{equation}
\log p \doteq -NH
\end{equation}
\begin{equation}
\log p \doteq \frac{\log 1/p}{N}
\end{equation}

Entonces $H$ es aproximadamente el logaritmo del reciproco de la
probabilidad de una longitud de secuencia tipica dividido por el
numero de simbolos en la secuencia. El mismo resultado permanece para
cualquier fuente. Dicho de otra forma mas precisa tenemos (ver
ap\'{e}ndice \ref{a3}): 

\begin{theorem}
Dado cualquier $\epsilon > 0$ y $\delta > 0$, podemos encontrar un
$N_{0}$ de tal manera que las secuencias de cualquier longitud $N \geq
N_{0}$ caen dentro de dos clasificaciones:
\begin{enumerate}
\item Un conjunto cuya probabilidad total es menor que $\epsilon$.
\item El residuo, todos aquellos miembros cuyas probabilidades satisfacen la inequalidad
\begin{equation}
|\frac{\log p^{-1}}{N} - H| < \delta
\end{equation}
\end{enumerate}
\end{theorem}

En otras palabras estamos casi seguros de tener $\frac{log p^{-1}}{N}$
muy cerca a $H$ cuando $N$ es grande.  Un resultado estrechamente
relacionado trata con el numero de secuencias de varias
probabilidades. Considerando de nuvo el numero de secuencias de
longitud $N$ y dejando que sean acomodados en order decremental de
probabilidad. Definimos $n(q)$ para ser el numero que debemos de tomar
de este conjunto empezando con el mas probable a fin de acumular una
probabilidad total q para aquellos tomados.

\begin{theorem}
\begin{equation}
\lim_{N \rightarrow \infty} \frac{\log n(q)}{N} = H
\end{equation}
cuando $q$ no es igual a 0 o 1.
\end{theorem}

Se puede interpretar $\log n(q)$ como el numero de bits requeridos
para especificar la secuencia cuando consideramos solo la secuencia
mas probable con una probabilidad $q$. Entonces $\frac{\log n(q)}{N}$
es el numero de bits por simbolo para la especificaci\'{o}n. Este
teorema dice que para un $N$ grande este ser\'{a} independiente de $q$
e igual a $H$. La taza de crecimiento del logaritmo del numero de
secuencias rasonablemente probables est\'{a} dado por $H$,
independientemente de nuestra interpretacion de ``razonalmente
probable''. Debido a estos resultados, que son probados en el apendice
3, es posible para la mayoria de nuestros propositos el tratar a las
secuencias largas como si solo fueran $2^{HN}$ de si mismas, cada una
con una probabilidad $2^{-HN}$.  Los siguientes dos teoremas muestran
que $H$ y $H'$ pueden ser determinados al limitar las operaciones
directamente de las estadisticas de las secuencias de mensajes, sin
referencia a los estados y probabilidades de transicion entre los
estados.

\begin{theorem}
Siendo $p(B_{i})$ la probabilidad de una secuencia $B_{i}$ de simbolos
de una fuente. Sea
\begin{equation}
G_{N} = - \frac{1}{N} \sum{i}{} p(B_{i}) \log p(B_{i}),
\end{equation}
donde la suma est\'{a} por encima de todas las secuencias $B_{i}$ que
contienen $N$ simbolos. Entonces $G_{N}$ es una funcion de decremento
monotono de $N$ y
\begin{equation}
\lim_{N \rightarrow \infty} G_{N} = H.
\end{equation}
\end{theorem}

\begin{theorem}
Sea $p(B_{i},S_{j})$ la probabilidad de secuencia $B_{i}$ seguida por
el simbolo $S_{j}$ y $p_{B_{j}}(S_{j}) = p(B_{i},S_{j}/p(B_{i}))$ sea
la probabilidad condicional de $S_{j}$ despues de $B_{i}$. Sea
\begin{equation}
F_{N} = - \sum{i,j}{} p(B_{i},S_{j}) \log p_{B_{i}}(S_{j}),
\end{equation}
donde la suma est\'{a} por encima todos los bloques $B_{i}$ de $N-1$ simbolos y sobre
todos los simbolos $S_{j}$. Entonces $F_{N}$ es una funcion mon\'{o}tona decreciente de $N$,
\begin{equation}
\begin{array}{rcl}
F_{N} &=& NG_{N} - (N - 1)G_{N-1}, \\
G_{N} &=& \frac{1}{N} \sum{n=1}{N} F_{n}, \\
F_{N} &\leq& G_{N},
\end{array}
\end{equation}
y $\lim_{N \rightarrow \infty} F_{N} = H$.  
\end{theorem}

Estos resultados se derivan del ap\'{e}ndice \ref{a3}. Muestran que
unas series de aproximaciones a H pueden ser obtenidas considerando
solo la estructura estadistica de las secuencias sobre $1, 2, \cdots,
N$ simbolos. $F_{N}$ es la mejor aproximaci\'{o}n. De hecho $F_{N}$ es
la entropia de la aproximacion de $N^{th}$ orden a la fuente del tipo
discutido arriba. Si no existen influencias estadisticas se extienden
sobre mas $N$ simbolos, eso si la probabilidad condicional de el
siguiente simbolo sabiendo el $(N-1)$ anterior no es cambiado por
conocimiento de cualquiera anterior a el, entonces $F_{N} =
H$. $F_{N}$ por supuesto es la entropia condicional del siguiente
simbolo cuando los $(N-1)$ anteriores son conocidos, mientras $G_{N}$
es la entropia por simbolo de bloques de $N$ simbolos.  El radio de
entrop\'{i}a de una fuente a el valor m\'{a}ximo valor que puede tener
mientras aun est\'{a} restringido a los mismos simbolos ser\'{a}
llamado su \textit{entropia} relativa. Este es la compresion
m\'{a}xima posible cuando codificamos dentro del mismo alfabeto. Uno
menos la entrop\'{i}a relativa es la \textit{redundancia}. La
redundancia de el idioma ingl\'{e}s ordinario, no teniendo en cuenta a
las estructuras estadisticas sobre distancias mayores a ocho letras,
es aproximadamente 50\%. Esto significa que cuando se escribe en
ingles la mitad de lo que se escribe es determinado por la estructura
del lenguaje y la mitad se escoge libremente. La cifra de 50\% se
encontr\'{o} por diversos metodos independientes que todos arrojaron
resultados similares. Uno es por calculo de la entrop\'{i}a de las
aproximaciones al ingl\'{e}s. Un segundo metodo trata sobre borrar una
cierta fraccion de letras de un ejemplo de texto en ingl\'{e}s y
despues permitir a alguien el tratar de restaurarlo. Si puede ser
restaurado cuando 50\% son borrados por redundancia. Un tercer metodo
depende de ciertos resultados conocidos en criptografia.  Dos extremos
de la redundancia en prosa inglesa son representados por ingl\'{e}s
b\'{a}sico y por el libro de James Joyc ``Finnegans Wake''. El
vocabulario del ingl\'{e}s basico est\'{a} limitado a 850 palabras y
la redundancia es muy alta. Esto se refleja en la expansion que ocurre
cuando un texto es traducido a ingl\'{e}s basico. Por otro lado Joyce
agranda el vocabulario y alega lograr una compresi\'{o}n del contenido
sem\'{a}ntico.  La redundancia de un lenguaje est\'{a} relacionada a
la existencia de crucigramas. Si la redundancia es cero, cualquier
secuencia de letras es un texto razonable en el lenguaje y cualquier
arreglo bidimensional de letras forma un crucigrama. Si la redundancia
es muy alta, el lenguaje impone demasiadas restricciones para que los
crucigramas grandes existan. Un an\'{a}lisis mas detallado muestra que
si se asume que las restricciones impuestas por el lenguaje son de una
naturaleza ca\'{o}tica y aleatoria, crucigramas grandes son posibles
de realizar cuando la redundancia es 50\%. Si la redundancia es 33\%,
crucigramas tridimensionales deber\'{i}an ser realizables.

% aqui termina pagina 14
