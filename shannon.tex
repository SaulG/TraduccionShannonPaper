\documentclass{report}
\usepackage[spanish]{babel}
\usepackage{url}
\usepackage[titletoc,toc,title]{appendix}
\usepackage{amsmath}
\usepackage{amsthm}
\usepackage[letterpaper, top=30mm, bottom=30mm, left=30mm, right=30mm]{geometry}
\usepackage[T1]{fontenc}
\usepackage[sort]{natbib}
\usepackage{libertine}
\renewcommand*\familydefault{\sfdefault} 
\usepackage{hyperref}
\usepackage{floatflt}
\usepackage{graphicx}

\setlength{\parindent}{0pt}
\setlength{\parskip}{1em}

\makeatletter
\providecommand*{\diff}%
        {\@ifnextchar^{\DIfF}{\DIfF^{}}}
\def\DIfF^#1{%
        \mathop{\mathrm{\mathstrut d}}%
                \nolimits^{#1}\gobblespace
}
\def\gobblespace{%
        \futurelet\diffarg\opspace}
\def\opspace{%
        \let\DiffSpace\!%
        \ifx\diffarg(%
                \let\DiffSpace\relax
        \else
                \ifx\diffarg\[%
                        \let\DiffSpace\relax
                \else
                        \ifx\diffarg\{%
                                \let\DiffSpace\relax
                        \fi\fi\fi\DiffSpace} 

\author{Claude E. Shannon}
\title{Una teor\'{\i}a matem\'{a}tica de la comunicaci\'{o}n}
\date{1948}
\begin{document}
\maketitle

\newtheorem{theorem}{Teorema}[section]
\newtheorem{definition}{Definici\'{o}n}[section]

\begin{abstract}
  Esta es una traducci\'{o}n al espa\~{n}ol del art\'{\i}culo
  publicado por Shannon en {\em The Bell System Technical Journal},
  realizado a base del PDF disponible en
  \url{http://cm.bell-labs.com/cm/ms/what/shannonday/shannon1948.pdf}
  como un esfuerzo colectivo de estudiantes de octavo semestre del ITS
  de la FIME de la UANL por puntos extra en la unidad de apxrendizaje
  {\em Teor\'{\i}a de la informaci\'{o}n y m\'{e}todos de
    codificaci\'{o}n}, impartida por Dra.\ Elisa Schaeffer en la
  primavera del 2013.
\end{abstract}

Hello world!
8 \leq 10 \approx True \pi \Lambda
 % pp. 1--3 recientemente confirmados
En caso de que existan restricciones en secuencias permitidas
todav\'ia se puede obtener una ecuaci\'on diferencial de este tipo y
encontrar C de la ecuaci\'on caracter\'istica. En el caso mencionado
anteriormente de la telegraf\'ia
\begin{equation}
N(t) = N(t-2)+N(t-4)+N(t-5)+N(t-7)+N(t-8)+N(t-10),
\end{equation}
como se ve por el c\'alculo de secuencia de los s\'imbolos de acuerdo
al \'ultimo o al siguiente \'ultimo s\'imbolo ocurriendo.  Por lo
tanto $-\log\mu_{0}$ donde $\mu_{0}$ es la ra\'iz positiva de $1
= \mu^{2}+\mu^{4}+\mu^{5}+\mu^{7}+\mu^{8}+\mu^{10}$. Resoviendo eso
encontramos que $C = 0.539$.

Una restricci\'on general se puede colocar en secuencias como la
siguiente: Imagenemos un numero de estados posibles $a_{1},
a_{2},\ldots,{m}$. Por cada estado solo ciertos s\'imbolos del
conjunto $S_{1},\ldots,S_{n}$ pueden ser transmitidos (diferentes
subconjuntos por cada estado). Cuando uno de esos han sido
transmitidos el estado cambia a un nuevo estado dependiendo de el
viejo estado y del s\'imbolo en particular transmitido. Un ejemplo
simple de esto es el caso del tel\'egrafo. Hay dos estados en
funci\'on de si o no un espacio era el \'ultimo s\'imbolo transmtido.
Si es as\'i, entonces s\'olo un punto o una raya puede ser enviado al
lado y el estado cambia siempre. Si no, cualquier s\'imbolo puede ser
transmitido y los cambios de estado si un espacio es enviado, de lo
contrario sigue siendo el mismo.  Las condiciones pueden ser indicadas
en una gr\'afica lineal como se ve en la figura \ref{fig:2}. 

\begin{figure}[!ht]
\centerline{\includegraphics[width=80mm]{ejemplo.png}}
\caption{Una representaci\'{o}n gr\'{a}fica de las restricciones 
en los s\'{\i}mbolos telegr\'{a}ficos.}
\label{fig:2}
\end{figure}

Los puntos de uni\'on correspoden a los estados y las l\'ineas
indicando los s\'imbolos posibles en un estado y el estado resultante.
En el Anexo \ref{a1} se muestran que si las condiciones en las
secuencias permitidas puede ser descrita en la forma $C$ puede existir
y se puede calcular de acuerdo con el siguiente resultado:
\begin{theorem}
 Siendo $b_{ij}^{(s)}$ la duraci\'on del $s$\'{e}simo s\'{\i}mbolo, el
cual es permisible en el estado $i$ y conduce al estado $j$. Despu\'es la
capacidad del canal $C$ es igual a $\log W$ donde $W$ es la ra\'iz real
m\'as grande de la ecuaci\'on:
\begin{equation}
\left|\sum_{s}W^{-b_{ij}^{(s)}}-\delta_{ij}\right|=0,
\end{equation}
donde $\delta_{ij}=1$ si $i=j$ y es cero en cualquier otro caso
\end{theorem}
Por ejemplo en el caso del tel\'egrafo el determinante es:
\begin{equation}
\begin{vmatrix}
-1&(W^{-2}+W^{-2}) \\ 
 (W^{-3}+W^{-6})&(W^{-2}+W^{-2}-1) 
\end{vmatrix}=0
\end{equation}
Esta expansi\'on conduce a la ecuaci\'on dada anteriormente para este
caso.

\clearpage

\chapter{La fuente discreta de la informaci\'on}
\label{sec:2}

Hemos visto que bajo condiciones muy generales el logaritmo del
n\'umero de se\~nales posibles en un canal discreto aumenta
linealmente con el tiempo. La capacidad de transmisi\'on de
informaci\'on se puede especificar dando a esta tasa de aumento, el
n\'umero de bits por segundo necesarios para especificar la se\~nal
particular utilizada.  Consideremos ahora la fuente de
informaci\'on. {\textquestiondown}C\'omo es una fuente de
informaci\'on a ser descrita matem\'aticamente, y la cantidad de
informaci\'on en bits por segundo se produce en una fuente dada? El
principal punto en cuesti\'on es la efecto de los datos estad\'isticos
sobre la fuente en la reducci\'on de la capacidad requerida de la
canal, por el uso de una correcta codificaci\'on de la
informaci\'on. En la telegraf\'ia, por ejemplo, los mensajes para ser
transmitidos consisten en secuencias de letras. Estas secuencias, sin
embargo, no son completamente aleatorias.

En general, las oraciones tienen una estructura estad\'istica de, por
ejemplo, ingl\'es. La letra $E$ se produce con m\'as frecuencia que
$Q$, la secuencia $TH$ con m\'as frecuencia que $XP$, etc. La
existencia de esta estructura permite hacer un ahorro en el tiempo (o
capacidad de canal) para codificar correctamente las secuencias de
mensajes en una secuencia de se\~nales. Esto ya se hace en una medida
limitada en telegraf\'ia utilizando el s\'imbolo de canal m\'as corto,
un punto, por la m\'as letra E que es la m\'as com\'un en ingl\'es,
mientras que las letras infrecuentes, $Q$, $X$, $Z$ est\'an
representados por secuencias m\'as largas de puntos y rayas.

Esta idea se lleva a\'un m\'as lejos en ciertos c\'odigos comerciales
donde las palabras y frases comunes est\'an representados por cuatro o
cinco grupos de c\'odigo de letra con un considerable ahorro de tiempo
promedio.  Los telegramas est\'andar de saludo y aniversario se han
usado tanto hasta el punto de que se codifican una o dos frases en una
secuencia relativamente corta de n\'umeros.

Podemos pensar en una fuente discreta de generar el mensaje, s\'imbolo
por s\'imbolo. Se elegir\'a una sucesi\'on de s\'imbolos de acuerdo
con ciertas probabilidades dependiendo, en general, de las opciones
anteriores, as\'i como los s\'imbolos en cuesti\'on. Un sistema
f\'isico, o un modelo matem\'atico de un sistema que produce una
secuencia de s\'imbolos gobernadas por un conjunto de probabilidades,
se le conoce como un proceso estoc\'astico.  Podemos considerar por lo
tanto que una fuente discreta puede ser representada por un proceso
estoc\'astico. A la inversa, cualquier proceso estoc\'astico que
produce una secuencia discreta de s\'imbolos elegidos a partir de un
conjunto finito puede ser considerado una fuente discreta. Esto
incluye casos como:

\begin{enumerate}
\item{Idiomas naturales escritos como el ingl\'es, alem\'an y chino.}

\item{Fuentes de informaci\'on continuos que se han vuelto discreta por
alg\'un proceso de cuantificaci\'on. Por ejemplo, la se\~nal de voz
cuantizada desde un transmisor PCM, o una se\~nal de televisi\'on
cuantizada.}

\item{Casos matem\'aticos en los que simplemente se definen abstractamente
procesos estoc\'asticos que generan una secuencia de s\'imbolos. Los
siguientes son ejemplos de este \'ultimo tipo de fuente.}
\end{enumerate}

\begin{exmp}
Supongamos que tenemos cinco letras $A$, $B$, $C$, $D$, $E$, que se
eligen cada una con probabilidad $0.2$, elecciones sucesivas siendo
independientes. Esto dar\'ia lugar a una secuencia, la siguiente es un
t\'ipico ejemplo:
$$B D C B C E C C C A D C B D D A A E C E E A,$$
$$A B B A E D E C A C E E E E B A C B C E A D.$$
Este fue construido con el uso de una tabla de azar.
\label{ej:a}
\end{exmp}

\begin{exmp}
Utilizando las mismas cinco letras y siendo las probabilidades $0.4$,
$0.1$, $0.2$, $0.2$, $0.1$, respectivamente, con decisiones independientes
sucesivas. Un mensaje t\'ipico de esta fuente es entonces:
$$A A A C D C B D C E A A D A D A C E D A,$$
$$E A D C A B E D A D E D C C A A A A A D.$$
\label{ej:b}
\end{exmp}

\begin{exmp}
Una estructura m\'as complicada se obtiene si los s\'imbolos no son
elegidos de manera independiente pero sus probabilidades dependen de
las letras anteriores. En el caso m\'as simple este tipo de elecci\'on
depende solamente de la letra anterior y no de las que estan antes. La
estructura estad\'istica puede entonces ser descrita por un conjunto
de probabilidades de transici\'on $p_{i}(j)$, la probabilidad de que
la letra $i$ es seguida por la letra $j$. Los \'indices de $i$ y $j$
se extienden sobre todos los s\'imbolos posibles. Una segunda manera
equivalente de especificar la estructura es dar el ``diagrama'' de
probabilidades $p(i,j)$, la frecuencia relativa de el diagrama $i j$.
La frecuencia de letras $p(i)$, (la probabilidad de la letra i), la
transici\'on de probabilidades $p_{i}(j)$ y el diagrama de
probabilidades $p(i,j)$ son descritos en las siguientes formulas:
\begin{equation}
p(i)=\sum_{j}p(i,j)=\sum_{j}p(j,i)=\sum_{j}p(j)p_{j}(i) 
\end{equation}

falta terminar ejemplo C.
\label{ej:c}
\end{exmp}


 % pp. 4--5 recientemente asignados
\section{Representacion grafica de procesos Markovianos}

Los procesos estocasticos del tipo descrito arriba son matematicamente
conocidos como Procesos Markovianos Discretos y han sido estudiados
extensivamente en la literatura \footnote{AQUI}. El caso general puede
ser descrito de la siguiente manera: Existe un numero finito de
posibles "estados" de un sistema; $S_{1}, S_{2}, \ldots,
S_{n}$. Adem\'{a}s existen un conjunto de probabilidades de
transicion; $p_{i}(j)$ la probabilidad que si el sistema esta en
estado $S_{i}$ entonces enseguida vaya al estado $S_{j}$. Para
realizar este proceso Markoviano en una fuente de informaci\'{o}n solo
necesitamos asumir que una letra es producida para cada transicion
desde un estado a otro. Los estados corresponder\'{a}n al ``residuo de
influencia'' de letras precedentes.

La situacion puede ser representada graficamente como se muestra en
las figuras 3, 4 y 5. Los ``estados'' son los puntos de union
% aqui va figura 3
en la grafica y las probabilidades y letras son producidas para una
transicion son dadas ademas de la linea correspondiente. La figura 3
es para el ejemplo B en la seccion 2, mientras que la figura 4
corresponde al ejemplo C. En la figura 3
%figura 4
solamente hay un estado ya que letras sucesivas son independientes. En
la figura 4 hay tantos estados como letras. 

Si un ejemplo de un triagrama fuera construido, habr\'{i}a por maximo
$n^{2}$ estados correspondiendo a los posibles pares de letras
precediendo a uno que haya sido elegido. La figura 5 es un grafo para
el caso de estructura de palabras en el ejemplo D. Aqui $S$
corresponde a el simbolo ``espacio''. 

\section{Fuentes erg\'{o}dicas y mixtas}

Como se ha indicado anteriormente, una fuente discreta para nuestros
propositos puede ser considerada representada por un proceso
Markoviano. Entre los posibles procesos discretos Markovianos existe
un grupo con propiedades especiales con importancia en la teoria de la
comunicacion. Esta clase especial consiste en los procesos
``ergodicos'' y deberiamos de llamar a las fuentes correspondientes,
fuentes ergodicas. Aunque una definicion rigurosa de los procesos
ergodicos es algo complicada, la idea general es simple. En un proceso
ergodico cada secuencia producida por el proceso permanece igual en
sus propiedades estadisticas. Por lo tanto las frecuencias de letras,
las frecuencias de bigramas, etc., obtenidos de una secuencia en
particulas, se acercaran a un limite definido conforme la longitud de
las secuencua aumenta, independientemente de la secuencia en
particular. En realidad esto no es meramente cierto para cada
secuencia pero el grupo para el cual esto es falso tiene una
probabilidad de cero. Practicamente, la propiedad ergodica significa
homogeneidad estadistica.

Todos los ejemplos de lenguaje artificial dados anteriormente son
ergodicos. Esta propiedad est\'{a} relacionada a la estructura de los
grafos correspondientes. Si el grafo tiene las siguientes dos
propiedades el proceso correspondiente ser\'{a} erg\'{o}dico:

\begin{enumerate}
   \item El grafo no consiste de dos partes aisladas $A$ y $B$ dado que es
   imposible ir desde los puntos de union en la parte $A$ a los puntos
   de union en la parte $B$ a traves de las lineas del grafo en la
   direccion de las flechas y tambien es imposible ir desde las
   uniones en la parte $B$ a las uniones en la parte $A$.  \item
\end{enumerate}

\begin{equation}
b = a - 2 \text{ (ejemplo)}
\end{equation}
 % pp. 7--18
FALTA LO DE ARRIBA EN PAGINA 19

\part{El canal discreto con ruido}
\label{part:2}

\chapter{Representaci\'{o}n de un canal discreto con ruido}
\label{sec:11}

FALTA

\clearpage

\chapter{Equivocaci\'{o}n y capacidad de canal}
\label{sec:12}

FALTA

\begin{theorem}
\label{th:10}
Si el canal de correcci\'{o}n...
\end{theorem}

FALTA

\begin{figure}[!ht]
\centerline{\includegraphics[width=120mm]{Imagenes/Pagina21-Figura8.png}}
\caption{Un diagrama esquem\'{a}tico de un sistema de correcci\'{o}n.}
\label{fig:8}
\end{figure}

FALTA

\begin{exmp}
Suponga que los errores suceden al azar...
\end{exmp}

FALTA

\begin{theorem}
\label{th:11}
Que un canal discreto tenga
\end{theorem}

FALTA

\clearpage

\chapter{El teorema fundamental para un canal discreto con ruido}
\label{sec:13}

\begin{figure}[!ht]
\centerline{\includegraphics[width=100mm]{Imagenes/Pagina22-Figura9.png}}
\caption{La equivocaci\'{o}n posible para una entropia de entrada dada
  a un canal.}
\label{fig:9}
\end{figure}

FALTA

\begin{figure}[!ht]
\centerline{\includegraphics[width=140mm]{Imagenes/Pagina23-Figura10.png}}
\caption{Una representaci\'{o}n esquem\'{a}tica de las relaciones
  entre las entradas y salidas en un canal.}
\label{fig:10}
\end{figure}

FALTA

\clearpage

\chapter{Discussion}
\label{sec:14}

FALTA

 % pp. 19--25 recientemente confirmados
En el caso sin ruido un retraso generalmente requiere para la codificacion\'on ideal ahora 
esta tiene una funci\'on adicional que permite una muestra larga de ruido para afectar la señal antes de un juicio, se hace 
en el punto de resepci\'on  para el mensaje original.Incrementando el tamaño del ejemplo siempre agudiza las posibles afirmaciones estadisticas.

El contenido del teorema 11 y su prueba se pueden formular de una manera diferente que muestra 
una conexi\'on sin ruido de una manera mas clara.Considere las posibles duraciones de las señales T y supongamos 
un subconjutno de ellos se seleccionan para ser usados. Los que esten en el subconjunto se utilizen todos con igual probabilidad, y suponiendo 
que el resector es construido para seleccionar, como la señal original. la causa mas probable de subconjutno cuando 
una señal perturbada es recivida.Nosotros definimos \begin{em}N(T,q)\end{em} siendo el numero maximo de señales que podemos elegir para el subconjunto tal que la probabilidad de una interpretaci\on incorrecta sea menor o igual a q.

\begin{em}Teorema 12:\end{em} \displaystyle\lim_{t \to{}\infty}\frac{\log{N}(T,q)}{T} = C \begin{em}, Donde C es la capacidad de canal,a condici\'on que q no sea igual a 0 o 1\end{em}

En otras palabras, no importa lo que nos propusimos de limites de confiabilidad, podemos distinguir de forma fiable en el tiempo T 
suficientes mensajes para corresponder a CT bits, cuando T es suficientemente grande. En el teorema 12 podemos comparar 
la capacidad de un canal sin ruido en la secci\'on uno.

\begin{center}
EJEMPLO DE UN CANAL DISCRETO Y SU CAPACIDAD
\end{center} 

Un ejemplo simple de un canal discreto se indica en la figura 11.Hay tres posibles simbolos. EL primero 
nunca se vera afectado por el ruido.El segundo y la tercera tiene cada una probabilidad p de que viene a trav\'es impertubable q de ser cambiado en el otro elemto par.Nosotros tenemos (dejar =  \alpha = -[plogp + q\log{q}] y p y q son probabilidades de estar usando los simbolos primero y segundo).

\begin{centerMe gusta ·  · Compartir · Hace 9 minutos · }
H(x) = -P\log{P} - 2Q\log{Q}
H_y(x) = 2Q\alpha
\end{center}


Nosotros deseamos elegir P y Q, de tal manera que se maximice H(x) - H_y(x) sujeto a la restricci\'on P + 2Q = 1. 

Por lo tanto concideremos:

\begin{center}
U = -P\log{P} - 2Q\log{Q} -2Q\alfa + \lambda(P+2Q)

\frac{{\partial U}}{{\partial P}} = -1 - \log{P} + \lambda = 0

\frac{{\partial U}}{{\partial P}} = -2 - 2\log{Q} -2\alpha + 2\lambda = 0
\end{center}


Eliminando \lamda

\begin{center}
\log{P} = \log{Q} + \alpha
P = Q e^\alpha = Q\beta 
\end{center}

\begin{center}
P = \frac{\beta}{\beta + 2}   Q = \frac{1}{\beta + 2}
\end{center}

La capacidad del canal es de:

\begin{center}
C = \log{\frac{\beta + 2}{\beta}}
\end{center}

Notese como esto comprueba los valores obvios en los casos: p = 1 y p = \frac{1}{2}. En primero, \beta = 1 y C = \log{3}, el cual es correcto debido a que el canal es entonces sin ruido con tres posibles s\'imbolos. Si p = \frac{1}{2}, \beta = 2 y C = \log{2}.Aqu\'i el segundo y el tercer s\'imbolos, no se pueden distinguir en absoluto y act\'uan conjuntamente como un solo s\'imbolo. El primer simbolo se utiliza con una probabilidad P = \frac{1}{2}  y el segundo junto al tercero con probabilidad \frac{1}{2}.Esto puede ser distribuido entre ellos de cualquier modo deseado y todav\'ia alcanzar la m\'axima capacidad.
Para los valores intermedios de la capacidad del canal p estara entre \log{2} y \log{3}.Esta distinci\'on 
entre el segundo y tercer s\'imbolo transmite alguna informaci\'on, pero no tanto como en el caso sin ruido.

El primer s\'imbolo se utiliza tanto mas frecuentemente que los otros dos, debido a su ausencia de ruido.

\begin{center}
16. LA CAPACIDAD DE CANALES EN CIERTOS CASOS ESPECIALES
\end{center}

Si el ruido afecta símbolos sucesivos de canal de forma independiente pueden ser descritos por un conjunto de transici\'on de probabilidades p_{i,j}.Esta es la probabilidad, si el simbolo i es enviado, que j sera recibido.La tasa de canal maximo viene dado por el m\'aximo de

\begin{center}
- \sum_{i,j}P_i p_{i,j} \log{\sum{P_i p_{i,j}}} + \sum_{i,j}P_i p_{i,j}\log{p_{i,j}}
\end{center}
 
Multiplicar por P_s y sumando en s muestra que \mu = C. Vamos a la inversa de p_{sj} (si existe) en h_{st}  de modo que \sum_{s} h_{st} p_{sj} = \delta_{tj}.Entonces: 

\begin{center}
\sum_{s,j}h_{st} p_{s,j} \log{p_{s.j}} - \log{\sum_{i}P_i p_{i,t}} = C \sum_{s} h_{s,t}
\end{center}
Por lo tanto:

\begin{center}
\sum_{i} P_i p_{i,t} = exp[- C \sum_{s} h_{s,t}+ \sum_{s,j} h_{s,t} p_{s,j} \log{p_{s,j}}]
\end{center}

o:  
\begin{center}
P_i = \sum_{t} h_{i,t} exp[ - C \sum_{s} h_{s,t}+ \sum_{s,j} h_{s,t} p_{s,j} \log{p_{s,j}} ]
\end{center}


Este es el sistema de ecuaciones para determinar el valor maximo de P, con C se determina de manera que \sum P_i = 1. Cuando esto esta hecho, C sera la capacidad del canal y P_i las probabilidades para los simbolos de canal para lograr esta capacidado.
Si cada s\'imbolo de entrada tiene el mismo conjunto de probabilidades en las l\'ineas que emergen de ella y lo mismo sucede a cada s\'imbolo de salida, la capacidad puede ser calculada f\'acilmente. Los ejemplos se muestran en la figura 12. En tal caso H_x (y) es independiente de la destribuci\'on de probabilidades de los s\'imbolos de entrada, y esta dad por -\sum p_i \log{p_i}. Cuando p_i son los valores de probabilidad de transici\'on de cualquier s\'imbolo de entrada.La capacidad del canal es:

\begin{center}
MAX [H(y) - H_x(y)] = MAX H(y) + \sum p_i \log{p_i}.
\end{center}
  
El valor m\'aximo de H(y) esta claramente  \log{m} donde m es el numero de simbolos de salida, ya que es posible para hacer todos igualmente probables haciendo los simbolos de entradas igualmente probables. La capacidad del canal es por lo tanto:

\begin{center}
C = \log{m} + \sum p_i \log{p_i}
\end{center}

En la figura 12a ser\'ia 

\begin{center}
C = = \log{4} - \log{2} 0 \log{2}.
\end{center}

Esto se podria lograr mediante el uso solo the la 1a y 3d simbolo. En la Figura b:

\begin{center}
C = \log{4} - \frac{2}{3}\log{3} - \frac{1}{3}\log{6}
= \log{4} - \log{3} - \frac{1}{3}\log{2}
= \log{\frac{1}{3}} 2^{\frac{5}{3}}
\end{center}

En la figura 12c nosotros tenemos:

\begin{center}
C = \log{3} - \frac{1}{2}\log{2} - \frac{1}{3}\log{3} - \frac{1}{6} \log{6}

= \log{\frac{3}{ 2^{\frac{1}{2} 3^{\frac{1}{3} 6^{\frac{1}{6} }}
\end{center}

Supongamos que los s\'imbolos se dividen en varios grupos tal que el ruido causa a un s\'imbolo en un grupo a ser confundido con un s\'imbolo de otro grupo.Deja la capacidad de un grupo n-\'esimo ser C_n
(en bits por segundo)donde solo utilizamos los s\'imbolos de este grupo. Entonces es facil desmotrar  de todo el conjunto, la probabilidad P_n para todos los s\'imbolos del grupo n-esimo deber\'ia ser:

\begin{center}
P_n = \frac{2^{C_n}}{\sum 2^{C_n}}
\end{center}

En un grupo la probabilidad se distribuye tal como seria si estos eran los unicos simbolos que se utilizan. La capacidad del canal es:

\begin{center}
C = \log{\sum 2^{C_n}}
\end{center}

\begin{center}
17. UN EJEMPLO DE CODIFICACI\'ON EFICIENTE
\end{center}

El siguiente ejemplo, aunque es un poco realista, es un caso en que la coincidencia exacta para un canal con ruido, es posible. Hay dos s\'imbolos de canal 0 y 1, y el ruido les afecta en bloques de siete s\'imbolos. Un bloque de siete o se transmite sin error, o exactamente un s\'imbolo de los siete es incorrecta.Estas ocho posibilidades son igualmente probables. Nosotros tenemos:

\begin{center}
C = MAX[ H(y) - H_x(y) ]
= \frac{1}{7}[7 + \frac{8}{8}\log{\frac{1}{8}}]
= \frac{4}{7}bits/simbolos
\end{center}
 
Un c\'odigo eficiente, permite la correci\'on de todos los errores y transmitir a la tasa C, es el siguiente(encontrado por un metodo de R.Hamming):
 



 % 25--27 recientemente asignados

Deje un bloque de siete s\'imbolos que $X_{1}, X_{2},...,X_{7}$. De
estos $X_{3}, X_{5}, X_{6} y X_{7}$ son los mensajes de s\'{\i}mbolos
y eligidos arbitrariamente por la fuente. Los otros tres son
redundantes y se calculan como sigue:
\begin{center}
\begin{tabular}{c c c c}
$X_{4}$ & es elegido para hacer & $\alpha=X_{4}+X_{5}+X_{6}+X_{7}$ & par, \\
$X_{2}$ & `` `` `` & $\beta=X_{2}+X_{3}+X_{6}+X_{7}$ & `` `` ``, \\
$X_{1}$ & `` `` ``& $\gamma=X_{1}+X_{3}+X_{5}+X_{7}$ & `` `` ``.
\end{tabular}
\end{center}
Cuando un bloque de siete es recibido $\alpha, \beta$ y $\gamma$ son
calculados y si incluso llam\'o a cero, si un extra\~no llamado. El
n\'umero binario $\alpha \beta \gamma$ entonces da el sub\'indice de
la $X_{i}$ que no es correcto (si es 0 no hay error).

\clearpage

\begin{appendices}

\chapter{El crecimiento del n\'umero de bloques de s\'imbolos con una
condici\'on de estado finito}
\label{aini:1}

Siendo $N_{i}(L)$ el n\'umero de bloques de s\'imbolos de largo L
terminando en estado $i$.  Entonces tenemos
\begin{equation}
N_{j}(L)=\sum_{i,s} N_{i}(L-b_{ij}^{(s)})
\end{equation}
donde $b_{ij}^{1},b_{ij}^{2},...,b_{ij}^{m}$ el largo de los s\'imbolos los cuales pueden 
ser elegidos en estado $i$ y pasar a estado $j$. Esas son ecuaciones diferenciales lineales 
y el comportamiento como $L\rightarrow \infty $ debe ser de tipo
\begin{equation}
N_{j}=A_{j}W^{L}.
\end{equation}
Sustituyendo en la ecuaci\'on diferencial 
\begin{equation}
A_{j}W^{L} = \sum_{i,s}A_{i}W^{L-b_{ij}^{(s)}}
\end{equation}
o
\begin{equation}
\sum_{i} \left(\sum_{s}W^{b_{ij}^{(s)}}-\delta_{ij}\right)A_{i}=0.
\end{equation}
Para hacer posible esto el determinante
\begin{equation}
D(W)=\left| a_{ij}\right| =\left| \sum_{s}W^{-b_{ij}^{(s)}}-\delta_{ij}\right|
\end{equation}
debe desaparecer y esto determina $W$, que es, por supuesto, la mayor raíz real de $D = 0$.
La cantidad C est\'a dada entonces por
\begin{equation}
C = \lim_{L\rightarrow \infty}\frac{\log \sum A_{j}W^{L}}{L}=\log W
\end{equation}
y observamos tambi\'en que las mismas propiedades de crecimiento
resultan si nosotros requierimos que todos los bloques comiencen en el
mismo (elegido arbitrariamente) estado.

\clearpage

\chapter{Derivaci\'on de $H=-\sum p_{i}\log  p_{i}$}
\label{aini:2}

Siendo $H(\frac{1}{n},\frac{1}{n},...,\frac{1}{n})=A(n)$. Desde la
condici\'on (3) se puede descomponer una elecci\'on de $s^{m}$ con las
mismas probabilidades en una serie de $m$ elecciones de $s$ de
posiblemente igualmente probabilidades y obtener

\begin{equation}
A(s_{m})=mA(s).
\end{equation}

Igualmente

\begin{equation}
A(t^{n})=nA(t).
\end{equation}

Podemos elegir $n$ arbitrariamente grande y encontrar una $m$ para satisfacer

\begin{equation}
s^{m} \leq t^{n} < s^{(m+1)}.
\end{equation}

As\'i, tomando logaritmos y dividiendo en $n\log s$,

\begin{equation}
\frac{m}{n}\leq \frac{\log  t}{\log  s}\leq \frac{m}{n}+\frac{1}{n} \ o \ 
\left|\frac{m}{n}-\frac{\log  t}{\log  s}\right| < \epsilon 
\end{equation}

donde $\epsilon$ es arbitrariamente peque\~na.Ahora de la propiedad 
monot\'onica de $A(n)$,
\begin{equation}
A(s^{m})\leq A(t^{n})\leq A(s^{m+1})
\end{equation}
\begin{equation}
mA(s)\leq nA(t) \leq (m+1)A(s)
\end{equation}

Por lo tanto

\begin{equation}
H=K\left[\sum p_{i}\log \sum n_{i}-\sum p_{i}\log n_{i}\right]
\end{equation}
\begin{equation}
=-K\sum p_{i}\log \frac{n_{i}}{\sum n_{i}}=-K\sum p_{i}\log p_{i}.
\end{equation}

Si el $p_{i}$ es inconmensurables, que se puede aproximar por racionales y la misma 
expresi\'on debe contener por nuestra suposici\'on de continuidad. As\'i, la expresi\'on 
se mantiene en general. La elecci\'on del coeficiente $K$ es un asunto de conveniencia y 
asciende a la elecci\'on de una unidad de medida.

\clearpage

\chapter{Teoremas sobre fuentes erg\'odicas}
\label{aini:3}

Si es posible pasar de un estado con $P>0$ a cualquier otro a lo largo
de un camino de probabilidad $p>0$, el sistema es erg\'odico y una
fuerte ley de numeros largos es aplicada.  As\'i, el n\'umero de veces
que un camino dado $p_{ij}$ en una red se desplaza en una larga
secuencia de longitud $N$ es aproximadamente proporcional a la
probabilidad de estar en $i$, dice $P_{i}$, y escoge la ruta,
$P_{i}p_{ij}N$. Si $N$ es lo suficientemente larga la probabilidad de
porcentaje de error $\pm\delta$ es esto es menor que $\epsilon$ as\'i
que para todos, pero un conjunto de baja probabilidad de los n\'umeros
reales se encuentran dentro de los l\'imites
\begin{equation}
(P_{i}p_{ij}\pm \delta)N.
\end{equation}
Por lo tanto casi todas las secuencias tienen una probabilidad $p$ dada
por
\begin{equation}
p=\prod p_{ij}^{(P_{i}p_{ij}\pm \delta)N}
\end{equation}
y $\frac{\log  p}{N}$ es limitada por
\begin{equation}
\frac{\log  p}{N} = \sum (P_{i}p_{ij}\pm \delta)\log  p_{ij}
\end{equation}
o
\begin{equation}
\left|\frac{\log  p}{N} = \sum (P_{i}p_{ij}\pm \delta)\log p_{ij}\right|<\eta
\end{equation}

Esto demuestra el teorema \ref{th:3}. Teorema \ref{th:4} sigue
inmediatamente de esto en el c\'alculo de los l\'imites superior e
inferior para $n(q)$ basado en el rango posible de valores de $p$ en
el Teorema \ref{th:3}.  En el mixto (no erg\'{o}dico) caso si
\begin{equation}
L = \sum p_{i}L_{i}
\end{equation}
y las entropias de los componentes son $H_{1}\leq H_{2}\leq ... \leq
H_{n}$ tenemos :
\begin{theorem}
$\lim_{N\rightarrow \infty }\frac{\log n(q)}{N}=\varphi (q)$ es una
funci\'on de paso decreciente $\varphi (q)=H_{s}$ en el intervalo
$\sum_{1}^{s-1}\alpha_{i}<q<\sum_{1}^{s}\alpha_{i}$.
\label{nuevo}
\end{theorem}

Para demostrar teoremas \ref{th:5} y \ref{th:6}, primero note que
$F_{N}$ es mon\'otona decreciente porque el aumento de $N$ agrega un
sub\'indice a la entrop\'ia condicional. Una sencilla sustituci\'{o}n
de $p_{B_{i}}(S_{j})$ en la definici\'on de $F_{N}$ muestra que
\begin{equation}
F_{N} = NG_{N}-(N-1)G_{N}-1,
\end{equation}
y sumando esto por todas las $N$ dadas $G_{N}=\frac{1}{N}\sum
F{n}$. Por lo tanto $G_{N}\leq F_{N}$ y $G_{N}$ son mon\'{o}tonas
decrecientes. Tambi\'en se debe aproximar al mismo l\'imite. Usando el
teorema \ref{th:3} se observa que $\lim_{N\rightarrow \infty } G_{N} =
H$.

\clearpage

\chapter{Maximizar la tasa para un sistema de restricciones}
\label{a4}

Supongamos que tenemos un conjunto de restricciones sobre secuencias
de s\'imbolos que es del tipo de estado finito y puede ser
representado por una gr\'afica lineal. Siendo $\l_{ij}^{(s)}$ el largo
de varios s\'imbolos que pueden pasar de estado $i$ a estado $j$.  ?`
Qu\'e distribuci\'on de probabildades $P_{i}$ para los estados
diferentes y $p_{ij}^{(s)}$ para elegir un s\'imbolo $s$ en estado $i$
e ir a estado $j$ maximiza la tasa de generaci\'on de informaci\'on en
virtud de estas limitaciones? Las limitaciones defininen un canal
discreto y la velocidad m\'aximo debe ser menor o igual a la capacidad
$C$ de este canal, ya que si todos los bloques de longitud grande eran
igualmente probables, esta tasa dar\'ia lugar, y si es posible que
esto ser\'ia mejor. Se mostrar\'a que esta tasa se puede lograr
mediante la elecci\'on adecuada de $P_{i}$ y $p_{ij}^{(s)}$.  La tasa
es:
\begin{equation}
\frac{-\sum P_{i}p_{ij}^{(s)}\log p_{ij}^{(s)}}{\sum P_{i}p_{ij}^{(s)}l _{ij}^{(s)}} = \frac{N}{M}.
\end{equation}

Siendo $l_{ij}=\sum_{s}l _{ij}^{(s)}$. Evidentemente para un m\'aximo de 
$p_{ij}^{(s)}=k exp l_{ij}^{(s)}$. Las restricciones a la maximizaci\'on son 
$\sum P_{i} = 1$, $\sum P_{i}(p_{ij}-\delta_{ij})=0$. Por lo tanto podemos maximizar 
\begin{equation}
U=\frac{-\sum P_{i}p_{ij}\log p_{ij}}{\sum P_{i}p_{ij}l_{ij}}+\lambda \sum _{i}P_{i}+\sum\mu _{i}p_{ij}+\sum\eta _{j}P_{i}(p_{ij}-\delta_{ij})
\end{equation}

\begin{equation}
U=\frac{-\sum P_{i}p_{ij}\log p_{ij}}{\sum P_{i}p_{ij}l_{ij}}+\lambda \sum _{i}P_{i}+\sum\mu _{i}p_{ij}+\sum\eta _{j}P_{i}(p_{ij}-\delta_{ij})
\end{equation}

\begin{equation}
\frac {\partial U}{\partial p_{ij}}=-\frac{MP{i}(1+\log p_{ij})+NP_{i}l_{ij}}{M^{2}}+\lambda+\mu_{I}+\eta _{i}P_{i}=0.
\end{equation}

Resolviendo para $p_{ij}$
\begin{equation}
p_{ij}=A_{i}B_{j}D^{-l_{ij}}
\end{equation}

Desde

\begin{equation}
\sum _{j}p_{ij}=1,\ A_{i}^{-1}=\sum_{j}B_{j}D^{-l_{ij}}
\end{equation}
\begin{equation}
p_{ij}=\frac{B_{j}D^{-l_{ij}}}{\sum_{s}B_{s}D^{-l_{is}}}
\end{equation}

para luego

\begin{equation}
p_{ij}=\frac{B_{j}}{B_{i}}C^{-l_{ij}}
\end{equation}
\begin{equation}
\sum p_{i}\frac{B_{j}}{B_{i}}C^{-l_{ij}}=P_{j}
\end{equation}

o

\begin{equation}
\sum \frac{P_{i}}{B_{i}}C^{-l_{ij}}=\frac {P_{j}}{B_{j}}.
\end{equation}

Entonces si $\lambda_{i}$ satisface

\begin{equation}
\sum\gamma _{i}C^{-lij}=\gamma_{j}
\end{equation}
\begin{equation}
P_{i}=B_{i}\gamma _{i}.
\end{equation}

Tanto los conjuntos de ecuaciones para $B_{i}$ and $\gamma_{i}$ puede ser satisfecha ya 
que C es
\begin{equation}
\left | C^{-l_{ij}} - \delta _{ij}\right |=0.
\end{equation}

En este caso la tasa es
\begin{equation}
\sum \frac{P_{i}p_{ij}\log \frac{B_{j}}{B_{i}}C^{-l_{ij}}}{\sum P_{i}p_{ij}l_{ij}}=C-\frac{P_{i}p_{ij}\log \frac{B_{j}}{B_{i}}}{\sum P_{i}p_{ij}l_{ij}}
\end{equation}

pero

\begin{equation}
\sum P_{i}p_{ij}(\log B_{j}-\log B_{i})=\sum_{j}P_{j}\log B_{j}-\sum P_{i}\log B_{i}=0
\end{equation}

Por lo tanto la tasa es $C$ y como esto nunca se podr\'ia superar este es el 
m\'aximo, lo que justifica la soluci\'on supuesta.

\end{appendices}
 % 28--31 recientemente asignados
\chapter{Preliminares matem\'aticos}

En esta entrega final del documento se aborda el caso donde las
se\~nales o los mensajes, o ambos, son variables continuas, en
contraste con la naturaleza discreta asumida hasta ahora. En gran
medida, el caso continuo puede obtenerse a trav\'es de un proceso
limitado del caso discreto dividiendo la continuidad de mensajes y
se\~nales en un n\'umero elevado pero finito de peque\~nas regiones y
calculando los diferentes par\'ametros que intervienen en una base
discreta. A medida que el tama\~no de las regiones se disminuye, desde
el enfoque general, estos par\'ametros limitan los valores adecuados
para el caso continuo. Sin embargo, hay algunos efectos nuevos que
aparecen y tambi\'en un cambio general del \'enfasis en la direcci\'on
de la especializaci\'on de los resultados generales a casos
particulares.

No vamos a intentar, en el caso continuo, obtener nuestros resultados
con la mayor generalidad, o con el rigor extremo de la matem\'atica
pura, ya que esto implicar\'ia una gran cantidad de teor\'ia de la
medida abstracta y oscurecer\'ia el hilo principal del an\'alisis. Sin
embargo, un estudio preliminar indica que la teor\'ia puede ser
formulada de una manera completamente axiom\'atica y rigurosa que
incluye tanto los casos continuos y discretos, y muchos otros.

\clearpage

\section{Conjuntos y familias de funciones}

Tendremos que hacer frente en el caso continuo con conjuntos de
funciones y familias de funciones. Un conjunto de funciones, como el
nombre implica, es simplemente una clase o colecci\'on de funciones,
generalmente de una variable, el tiempo. Puede ser especificado dando
una representaci\'on expl\'icita de las diversas funciones en el
conjunto, o impl\'icitamente, dando una propiedad cuya funci\'on en el
conjunto poseen y otros no lo hacen. Algunos ejemplos son:
\begin{enumerate}
  \item El conjunto de funciones:
  \begin{equation}
    f_{\theta}(t) = \sen(t+\theta).
  \end{equation}
  Cada valor particular de $\theta$ determina una funci\'on particular
  en el conjunto.

  \item El conjunto de todas las funciones de tiempo no conteniendo
  frecuencias sobre $W$ ciclos por segundo.

  \item El conjunto de todas las funciones limitadas en banda a $W$ y
  amplitud en $A$.

  \item El conjunto de todas las se\~nales de habla inglesa como
  funciones de tiempo.
\end{enumerate}

Una familia de funciones es un conjunto de funciones junto con una
medida de probabilidad mediante el cual se puede determinar la
probabilidad de una funci\'on en el conjunto que tiene ciertas
propiedades.\footnote[1]{En terminolog\'ia matem\'atica, las funciones
pertenecen a un espacio de medida cuya medida total es la unidad.} Por
ejemplo con el conjunto,
\begin{equation}
  f_{\theta}(t) = \sen(t+\theta),
\end{equation}
podemos tener una distribuci\'on de probabilidad para $\theta$, $P(\theta)$.
El conjunto se convierte en una familia. Algunos otros ejemplos de familias de
funciones son:
\begin{enumerate}
  \item Un conjunto finito de funciones $f_k(t)(k=1, 2, \ldots, n)$
  con la probabilidad de $f_k$ siendo $p_k$.

  \item Una familia de dimensi\'on finita de funciones
  \begin{equation}
    f(\alpha_1, \alpha_2, \ldots, \alpha_n; t)
  \end{equation}
  con una distribuci\'on de probabilidad sobre los par\'ametros $\alpha_i$:
  \begin{equation}
    p(\alpha_1, \ldots, \alpha_n).
  \end{equation}
  Por ejemplo podemos considerar la familia definida por
  \begin{equation}
    f(a_1, \ldots, a_n, \theta_1, \ldots, \theta_n; t) = \sum_{i=1}^{n} a_i
    \sen i(\omega t + \theta_i)
  \end{equation}
  con las amplitudes $a_i$ distribuidas normalmente e independientemente,
  y las fases $\theta_i$ distribuidas uniformemente (desde 0 a $2\pi$) e
  independientemente.

  \item La familia
  \begin{equation}
    f(a_i, t) = \sum_{n=-\infty}^{+\infty} a_n \frac{\sen\pi(2Wt-n)}{\pi(2Wt-n)}
  \end{equation}
  con la $a_i$ normal e independiente todas con la misma desviaci\'on
  est\'andar $\sqrt{N}$. Esta es una representaci\'on de ruido ``blanco'',
  banda limitada a la banda de 0 a $W$ ciclos por segundo y con potencia
  media de $N$.\footnote[2]{Esta representaci\'on puede ser utilizada
  como una definici\'on de banda de ruido blanco limitada. Esto tiene
  ciertas ventajas que implican un menor n\'umero de operaciones
  limitantes que usan definiciones que se han utilizado en el
  pasado. El nombre de ``ruido blanco'', ya firmemente arraigada en la
  literatura, es tal vez un poco desafortunado. En \'optica, luz
  blanca significa cualquier espectro continuo en contraste con un
  espectro de punto, o un espectro que es plano con una {\em longitud
  de onda} (que no es el mismo que un espectro plano con frecuencia).}

  \item Los puntos se distribuir\'an en el eje $t$ de acuerdo con una
  distribuci\'on de Poisson. En cada punto seleccionado la funci\'on
  $f(t)$ es colocada y las diferentes funciones agregadas, dando la
  familia
  \begin{equation}
    \sum_{k=-\infty}^{\infty} f(t+t_k)
  \end{equation}
  donde los $t_k$ son los puntos de la distribuci\'on Poisson. Esta
  familia puede ser considerada como un tipo de impulso o disparo de
  ruido donde todos los impulsos son id\'enticos.

  \item El conjunto de todas las funciones de habla inglesa con la medida de
  probabilidad dada por la frecuencia de ocurrencia en el uso ordinario.
\end{enumerate}

Una {\em familia} de funciones $f_{\alpha}(t)$ es {\em estacionaria}
si la misma familia resulta cuando todas las funciones son desplazadas
una cantidad fija de tiempo. La familia
\begin{equation}
  f_{\theta}(t) = \sen(t+\theta)
\end{equation}
es estacionario si $\theta$ es distribuido uniformemente desde 0 a $2\pi$. Si
desplazamos cada funci\'on en $t_1$ obtenemos
\begin{equation}
  f_{\theta}(t+t_1) = \sen(t+t_1+\theta)
\end{equation}
\begin{equation}
  f_{\theta}(t+t_1) = \sen(t+\varphi)
\end{equation}
con $\varphi$ distribuida uniformemente desde 0 a $2\pi$. Cada funci\'on ha
cambiado, pero la familia como un todo es invariante por el desplazamiento.
Los otros ejemplos dados anteriormente son tambi\'en estacionarios.

Una familia es {\em erg\'odica} si es estacionaria, y no existe un subconjunto
de las funciones en el conjunto con una probabilidad distinta de 0 y 1 que es
estacionaria. La familia
\begin{equation}
  \sen(t+\theta)
\end{equation}
es erg\'odica. Ning\'un subconjunto de estas funciones de probabilidad
$\neq0,1$ se transforma en s\'i mismo bajo todos los desplazamientos en el
tiempo. Por otra parte la familia
\begin{equation}
  a \sen(t+\theta)
\end{equation}
con $a$ distribuida normalmente y $\theta$ uniformemente, es estacionaria pero
no erg\'odica. El subconjunto de estas funciones con $a$ entre 0 y 1, por
ejemplo, es estacionario.

De los ejemplos dados, el 3 y 4 son erg\'odicos, y el 5 puede quiz\'as
ser considerado as\'i. Si una familia es erg\'odica, podemos decir que
aproximadamente cada funci\'on en el conjunto es t\'ipico de la
familia.

M\'as precisamente, se sabe que con un conjunto erg\'odico un promedio
de cualquier estad\'istica sobre el conjunto es igual (con una
probabilidad de 1) a un promedio sobre los desplazamientos de tiempo
de una funci\'on particular del conjunto.\footnote[3]{Este es el
famoso teorema erg\'odico o m\'as bien un aspecto de este teorema que
fue demostrado en formulaciones algo diferentes por Birkoff, von
Neumann, y Koopman, y posteriormente generalizada por Wiener, Hopf,
Hurewicz y otros. La literatura sobre la teor\'ia erg\'odica es
bastante extensa y se remite al lector a los trabajos de estos autores
para formulaciones precisas y generales; por ejemplo, E. Hopf,
``Ergodentheorie,'' {\em Ergebnisse der Mathematik und ihrer
Grenzgebiete}, v. 5; ``On Causality Statistics and Probability,'' {\em
Journal of Mathematics and Physics}, v. XIII, No. 1, 1934; N. Wiener,
``The Ergodic Theorem,'' {\em Duke Mathematical Journal}, v. 5, 1939.}
En t\'erminos generales, en cada funci\'on se puede esperar que, a
medida que avanza el tiempo, pase con la frecuencia adecuada todas las
convoluciones de cualquiera de las funciones en el conjunto.

Del mismo modo que se pueden realizar diversas operaciones sobre los
n\'umeros o funciones para obtener nuevos n\'umeros o funciones,
podemos realizar operaciones sobre familias para obtener nuevas
familias. Supongamos por ejemplo que tenemos una familia de funciones
$f_{\alpha}(t)$ y un operador $T$ que nos da por cada funci\'on
$f_{\alpha}(t)$ una funci\'on resultante $g_{\alpha}(t)$:
\begin{equation}
  g_{\alpha}(t) = Tf_{\alpha}(t).
\end{equation}

La medida de probabilidad se define por el conjunto de $g_{\alpha}(t)$
por medio del conjunto $f_{\alpha}(t)$. La probabilidad de un cierto
subconjunto de las funciones $g_{\alpha}(t)$ es igual a la del
subconjunto de las funciones $f_{\alpha}(t)$ que producen los miembros
del subconjunto dado de funciones $g$ bajo la operaci\'on
$T$. F\'isicamente esto corresponde al pasar el conjunto a trav\'es de
alg\'un dispositivo, por ejemplo, un filtro, un rectificador o un
modulador. Las funciones de salida del dispositivo forman la familia
$g_{\alpha}(t)$.

Un dispositivo u operador $T$ se llama invariante si desplazando la
entrada simplemente se desplaza la salida, es decir, si
\begin{equation}
  g_{\alpha}(t) = Tf_{\alpha}(t)
\end{equation}
implica
\begin{equation}
  g_{\alpha}(t+t_1) = Tf_{\alpha}(t+t_1)
\end{equation}
para toda $f_{\alpha}(t)$ y toda $t_1$. Esto es f\'acilmente
demostrado (ver Ap\'endice 5) que si $T$ es invariante y la familia de
entrada es estacionaria, luego la familia de salida es
estacionaria. Del mismo modo, si la entrada es erg\'odica, la salida
tambi\'en ser\'a erg\'odica.

Un filtro o un rectificador es invariante bajo todos los
desplazamientos de tiempo. La operaci\'on de modulaci\'on no es desde
la fase portadora que proporciona una estructura de tiempo
determinado. Sin embargo, la modulaci\'on es invariante bajo todos los
desplazamientos que son m\'ultiplos del per\'iodo del portador.

Wiener ha se\~nalado la \'intima relaci\'on entre la invariancia de
dispositivos f\'isicos en desplazamientos en tiempo y la teor\'ia de
Fourier.\footnote[4]{La teor\'ia de la comunicaci\'on est\'a muy en
deuda con Wiener por gran parte de su filosof\'ia y teor\'ia. Su
art\'iculo cl\'asico NDRC, {\em The Interpolation, Extrapolation and
Smoothing of Stationary Time Series} (Wiley, 1949), contiene la
primera formulaci\'on clara de la teor\'ia de la comunicaci\'on como
un problema estad\'istico, el estudio de las operaciones en series de
tiempo. Este trabajo, aunque ocupa principalmente de la predicci\'on
lineal y el problema de filtraci\'on, es una referencia colateral
importante en relaci\'on con el presente documento. Tambi\'en podemos
referirnos aqu\'i a {\em Wiener's Cybernetics} (Wiley, 1948), que
trata de los problemas generales de comunicaci\'on y control.} De
hecho, \'el ha demostrado que si un dispositivo es lineal, as\'i como
el invariante an\'alisis de Fourier, es entonces la herramienta
matem\'atica adecuada para tratar con el problema.

Una familia de funciones es la representaci\'on matem\'atica adecuada
de los mensajes producidos por una fuente continua (por ejemplo, el
habla), de las se\~nales producidas por un transmisor, y el del ruido
perturbador. La teor\'ia de la comunicaci\'on se interesa
espec\'ificamente, como se ha enfatizado por Wiener, no con las
operaciones en funciones particulares, pero con las operaciones sobre
familias de funciones. Un sistema de comunicaci\'on no est\'a
dise\~nado para una funci\'on de habla particular y menos a\'un para
una onda sinusoidal, pero si para la familia de las funciones del
habla.

\clearpage

\section{Funciones de familias con banda limitada}

Si la funci\'on de tiempo $f(t)$ es limitada a la banda de 0 a $W$
ciclos por segundo, este es completamente determinado dando sus
ordenadas en una serie de puntos discretos espaciados $\frac{1}{2W}$
segundos aparte de la manera indicada por el siguiente
resultado.\footnote[5]{Para una demostraci\'on de este teorema y
discusi\'on adicional, v\'ease el art\'iculo del autor ``Communication
in the Presence of Noise'' publicado en {\em Proceedings of the
Institute of Radio Engineers}, v. 37, No. 1, Enero, 1949, pp. 10-21.}

\begin{theorem}
  No dejar que $f(t)$ contenga frecuencias por encima de $W$. Entonces
  \begin{equation}
    f(t) = \sum_{-\infty}^{\infty} X_n \frac{\sen \pi(2Wt-n)}{\pi(2Wt-n)}
  \end{equation}
  donde
  \begin{equation}
    X_n = f\left(\frac{n}{2W} \right).
  \end{equation}
\end{theorem}

En esta expansi\'on $f(t)$ es representada como una suma de funciones
ortogonales. El coeficiente $X_n$ de los diversos t\'erminos puede
considerar como coordenadas en una dimensi\'on infinita ``espacio
funcional''. En este espacio cada funci\'on corresponde precisamente
un punto y cada punto a una funci\'on.

Una funci\'on puede ser considerada para estar sustancialmente
limitada a un tiempo $T$ si todas las ordenadas $X_n$ fuera de este
intervalo de tiempo es cero. En este caso todo pero $2TW$ de las
coordenadas ser\'ian cero. As\'i funciones limitadas a una banda $W$ y
duraci\'on $T$ corresponden a puntos en un espacio de $2TW$
dimensiones.

Un subconjunto de funciones de banda $W$ y duraci\'on $T$ corresponde
a una regi\'on en este espacio. Por ejemplo, las funciones cuya
energ\'ia total es menor que o igual a $E$ corresponden a puntos en
una esfera dimensional $2WT$ con radio $r=\sqrt{2WE}$.

Una familia de funciones de duraci\'on limitada y la banda
representada por una distribuci\'on de probabilidad $p(x_1, \ldots
x_n)$ en el correspondiente espacio $n$ dimensional. Si la familia no
est\'a limitada en el tiempo se pueden considerar las coordenadas
$2TW$ en un intervalo $T$ para representar sustancialmente la parte de
la funci\'on en el intervalo $T$ y la distribuci\'on de probabilidad
$p(x_1, \ldots, x_n)$ para dar la estructura estad\'istica de la familia
para intervalos de esa duraci\'on.

\clearpage

\section{Entrop\'ia de una distribuci\'on continua}

La entrop\'ia de un conjunto discreto de probabilidades $p_1, \ldots,
p_n$ ha sido definido como:
\begin{equation}
  H = -\sum p_i \log p_i.
\end{equation}
De manera an\'aloga se define la entrop\'ia de una distribuci\'on continua
con la funci\'on de la distribuci\'on de densidad $p(x)$ por:
\begin{equation}
  H = -\int_{-\infty}^{\infty} p(x) \log p(x) \diff x.
\end{equation}
Con una distribuci\'on $n$ dimensional $p(x_1, \ldots, x_n)$ tenemos
\begin{equation}
  H = -\int \cdots \int p(x_1, \ldots, x_n) \log p(x_1, \ldots,
  x_n) \diff x _1 \cdots \diff x _n.
\end{equation}
Si tenemos dos argumentos $x$ y $y$ (que pueden ser ellos mismos
multidimensionales) las entrop\'ias conjuntas y condicionales de $p(x, y)$
est\'an dadas por
\begin{equation}
  H(x, y) = -\iint p(x, y) \log p(x, y) \diff x \diff y 
\end{equation}
y
\begin{equation}
  H_x(y) = -\iint p(x, y) \log \frac{p(x, y)}{p(x)} \diff x \diff y 
\end{equation}
\begin{equation}
  H_y(x) = -\iint p(x, y) \log \frac{p(x, y)}{p(y)} \diff x \diff y 
\end{equation}
donde
\begin{equation}
  p(x) = \int p(x, y) \diff y 
\end{equation}
\begin{equation}
  p(y) = \int p(x, y) \diff x.
\end{equation}
Las entrop\'ias de distribuciones continuas tienen la mayor\'ia (pero
no todos) las propiedades del caso discreto. En particular, tenemos lo
siguiente:
\begin{enumerate}
\item{Si $x$ es limitado a un cierto volumen $v$ en su espacio,
  entonces $H(x)$ es un m\'aximo e igual a $\log v$ cuando $p(x)$ es
  constante $(1/v)$ en el volumen.}

 % pp. 32--35 terminados
Hola Mundo!

Agrego todas las ecuaciones de las paginas 36 - 40, y una imagen tabla.eps:

\begin{equation}
H\left ( x,y \right )\leq H\left ( x \right )+H\left ( y \right )
\end{equation}

\\

\begin{equation}
{p}'\left ( y \right )=\int a\left ( x,y \right )p\left ( x \right )dx
\end{equation}

\\

\begin{equation}
\int a\left ( x,y \right )dx=\int a\left ( x,y \right )dy=1
\:\:\:\:\:\:\:,\:\:\:\:\:\:\:
a\left ( x,y \right )\geq 0
\end{equation}

\\

\begin{equation}
H\left ( x,y \right )=H\left ( x \right )+H_{x}\left ( y \right )=H\left ( y \right )+H_{y}\left ( x \right )
\end{equation}

\\

\begin{equation}
H_x{}\left ( y \right )\leq H\left ( y \right )
\end{equation}

\\

\begin{equation}
H\left ( x \right )=-\int p\left ( x \right )\log p\left ( x \right )dx
\end{equation}

\\

\begin{equation}
\sigma^{2}=\int p\left ( x \right )x^{2}dx \:\:\:\:\: \textup{y} \:\:\:\:\: 1=\int p\left ( x \right )dx
\end{equation}

\\

\begin{equation}
\int \left [ -p\left ( x \right ) \log p\left ( x \right ) + \lambda p\left ( x \right ) x^{2} + \mu p\left ( x \right )\right ]dx
\end{equation}

\\

\begin{equation}
-1-\log p\left ( x \right )+\lambda x^{2}+\mu =0
\end{equation}

\\

\begin{equation}
p\left ( x \right )=\frac{1}{\sqrt{2\pi \sigma }}e^{-\left ( x^{2}/2\sigma ^{2} \right )}
\end{equation}

\\

\begin{equation}
A_{ij}=\int \cdots \int x_{i}x_{j}p\left ( x_{1},...,x_{n} \right )dx_{1}\cdots dx_{n}
\end{equation}

\\

\begin{equation}
H\left ( x \right )=\log \sqrt{2\pi e\sigma }
\end{equation}

\\

\begin{equation}
p\left ( x \right )=\frac{1}{\sqrt{2\pi \sigma }}e^{-\left ( x^{2}/2\sigma ^{2} \right )}
\newline
-\log p\left ( x \right )=\log \sqrt{2\pi \sigma }+\frac{x^{2}}{2\sigma ^{2}}
\newline
H\left ( x \right )=-\int p\left ( x \right )\log p\left ( x \right )dx
\newline
\hspace*{9.8mm} =\int p\left ( x \right )\log \sqrt{2\pi \sigma }dx+\int p\left ( x \right )\frac{x^{2}}{2\sigma ^{2}}dx
\newline
\hspace*{9.8mm} =\log \sqrt{2\pi \sigma }+\frac{\sigma }{2\sigma ^{2}}
\newline
\hspace*{9.8mm} =\log \sqrt{2\pi \sigma }+\log \sqrt{e}
\newline
\hspace*{9.8mm} =\log \sqrt{2\pi e \sigma }
\end{equation}

\\

\begin{equation}
p\left ( x_{1},...,x_{n} \right )=\frac{\left | a_{ij} \right |^{\frac{1}{2}}}{\left ( 2\pi  \right )^{n/2}}\exp \left ( -\frac{1}{2} \sum a_{ij}x_{i}x_{j} \right)
\end{equation}

\\

\begin{equation}
H=\log \left ( 2\pi e \right )^{n/2}\left | a_{ij} \right |^{-\frac{1}{2}}
\end{equation}

\\

\begin{equation}
a=\int_{0}^{\infty}p\left ( x \right )xdx
\end{equation}

\\

\begin{equation}
p\left ( x \right )=\frac{1}{a}e^{-\left ( x/a \right )}
\end{equation}

\\

\begin{equation}
H\left ( y \right )=\int \cdots \int p\left ( x_{1},...,x_{n} \right )J\left ( \frac{x}{y} \right )\log p\left ( x_{1},...,x_{n} \right )J\left ( \frac{x}{y} \right )dy_{1}\cdots dy_{n}
\end{equation}

\\

\begin{equation}
H\left ( y \right )=H\left ( x \right )-\int \cdots \int p\left ( x_{1},...,x_{n} \right )\log J \left ( \frac{x}{y} \right )dx_{1}\cdots dx_{n}
\end{equation}

\\

\begin{equation}
y_{j}=\sum_{i}^{\:}a_{ij}x_{i}
\end{equation}

\\

\begin{equation}
H\left ( y \right )=H\left ( x \right )+\log\left | a_{ij} \right |
\end{equation}

\\

\begin{equation}
p\left ( x_{1},...,x_{n}\right )
\end{equation}

\\

\begin{equation}
{H}'=-\lim_{n\rightarrow \infty }\frac{1}{n}\int \cdots \int p\left ( x_{1},...,x_{n}\right ) \log p\left ( x_{1},...,x_{n}\right )dx_{1}...dx_{n}
\end{equation}

\\

\begin{equation}
{H}'=\log \sqrt{2\pi eN}
\newline
H=W\log 2\pi eN
\end{equation}

\\

\begin{equation}
\left | \frac{\log p}{n} - {H}'\right |< \varepsilon 
\end{equation}

\\

\begin{equation}
\lim_{n\rightarrow \infty }\frac{\log V_{n}\left ( q \right )}{n}={H}'
\end{equation}

\\

\begin{equation}
p\left ( x_{1},...,x_{n}\right )=\frac{1}{\left ( 2\pi N \right )^{n/2}}\exp -\frac{1}{2N}\sum x_{i}^{2}
\end{equation}

\\

\begin{equation}
N_{1}=\frac{1}{2\pi e}\exp 2{H}'
\end{equation}

\\

\begin{equation}
H_{2}=H_{1}+\frac{1}{W}\int_{W}^{\:}\log \left | Y\left ( f \right ) \right |^{2}df
\end{equation}

\\

\begin{equation}
J=\prod_{i=1}^{n}\left | Y\left ( f_{i} \right )  \right |^{2}
\end{equation}

\\

\begin{equation}
\exp \frac{1}{W}\int_{W}^{\:}\log \left | Y\left ( f \right ) \right |^{2}df
\end{equation}

\\

\begin{equation}

\end{equation}

\\


 % pp. 36--40 recientemente confirmados
{\textexclamdown}Hola, mundo!
		
\begin{equation}
  x^{2}+y^{2}=h^2
\end{equation} 
 % pp. 41--45 recientemente confirmados
\begin{theorem}
  La capacidad del canal $C$ para una banda $W$ perturbada por el
  ruido t\'ermico blanco de potencia $N$ est\'a limitada por
  \begin{equation}
    C \ge W \log \dfrac{2}{\pi e^3} \dfrac{S}{N},
  \end{equation}
  donde $S$ es el pico permitido por el transmisor de potencia. Para
  $\dfrac{S}{N}$ suficientemente grande
  \begin{equation}
    C \le W \log \dfrac{\dfrac{2}{\pi e} S + N}{N} (1 + \epsilon)
  \end{equation}
  donde $\epsilon$ es arbitrariamente peque\~no. Como
  $\dfrac{S}{N} \to 0$ (y siempre que la banda $W$ inicia en 0)
  \begin{equation}
    \frac{C}{W \log \left(1 + \dfrac{S}{N} \right)} \to 1.
  \end{equation}
\end{theorem}

Queremos maximizar la entrop\'ia de la se\~nal recibida. Si
$\dfrac{S}{N}$ es grande, esto ocurrir\'a muy pronto cuando
maximizamos la entrop\'ia de la familia transmitida.

La parte superior asint\'otica se obtiene mediante la relajaci\'on de
las condiciones en la familia. Supongamos que el poder se limita a
$S$ no en cada instante de tiempo, pero s\'olo en los puntos de
muestreo. La entrop\'ia m\'axima de la familia transmitida en estas
condiciones debilitadas, es ciertamente mayor que o igual a la que en
las condiciones originales. Este problema alterado se puede resolver
f\'acilmente. La entrop\'ia m\'axima se produce si las diferentes
muestras son independientes y tienen una funci\'on de distribuci\'on
que es constante a partir de $-\sqrt{S}$ a $+\sqrt{S}$. La entrop\'ia
se puede calcular como
\begin{equation}
  W \log 4S.
\end{equation}
La se\~nal recibida tendr\'a entonces una entrop\'ia menor que
\begin{equation}
  W \log (4S + 2 \pi eN)(1 + \epsilon)
\end{equation}
con $\epsilon \to 0$ y $\dfrac{S}{N} \to \infty$, la capacidad del
canal se obtiene restando la entrop\'ia del ruido blanco,
$W \log 2 \pi eN$:
\begin{equation}
  W \log (4S + 2 \pi eN)(1 + \epsilon) - W \log (2 \pi eN) =
  W \log \dfrac{\dfrac{2}{\pi e} S + N}{N} (1 + \epsilon).
\end{equation}
Este es el l\'imite superior deseado unido a la capacidad del canal.

Para obtener un l\'imite inferior consideramos la misma familia de
funciones. Permitir que estas funciones pasen a trav\'es de un filtro
ideal con una caracter\'istica de transferencia triangular. La
ganancia es igual a la unidad en la frecuencia 0 y disminuyendo
linealmente hasta obtener 0 en la frecuencia $W$. En primer lugar,
demuestra que las funciones de salida de los filtros tienen una
limitaci\'on de potencia pico $S$ en todo momento (no s\'olo los
puntos de muestreo). En primer lugar observamos que un pulso
$\dfrac{\sen 2 \pi Wt}{2 \pi Wt}$ dentro del filtro produce
\begin{equation}
  \dfrac{1}{2} \dfrac{\sen^2 \pi Wt}{(\pi Wt)^2}
\end{equation}
en la salida. Esta funci\'on nunca es negativa. La funci\'on de
entrada (en el caso general) se puede considerar como la suma de una
serie de funciones desplazadas
\begin{equation}
  a \dfrac{\sen 2 \pi Wt}{2 \pi Wt}
\end{equation}
donde $a$, la amplitud de la muestra, no es mayor que $\sqrt{S}$. Por
lo tanto, la salida es la suma de funciones desplazados de la forma
no negativa, anteriormente con los mismos coeficientes. Estas
funciones son no negativas, el mayor valor positivo para cualquier
$t$ se obtiene cuando todos los coeficientes $a$ tienen sus valores
positivos m\'aximos, es decir, $\sqrt{S}$. En este caso la funci\'on
de entrada es una constante de amplitud $\sqrt{S}$ y ya que el filtro
tiene una unidad de ganancia para D.C., la salida es la misma. Por lo
tanto el conjunto de salida tiene una potencia pico $S$.

La entrop\'ia de la familia de salida puede ser calculada a partir
de la familia de entrada usando el teorema a tratar con dicha
situaci\'on. La entrop\'ia de salida es igual a la entrop\'ia de
entrada m\'as la ganancia media geom\'etrica del filtro:
\begin{equation}
  \int_{0}^{W} \log G^2 \diff f = \int_{0}^{W} \log \left(
  \dfrac{W - f}{W} \right)^2 \diff f = -2W.
\end{equation}
Por lo tanto la entrop\'ia de salida es
\begin{equation}
  W \log 4S - 2W = W \log \dfrac{4S}{e^2}
\end{equation}
y la capacidad del canal es mayor que
\begin{equation}
  W \log \dfrac{2}{\pi e^3} \dfrac{S}{N}.
\end{equation}
Ahora queremos demostrar que, para peque\~nos $\dfrac{S}{N}$
(potencia pico de la se\~nal a trav\'es de la potencia media de
ruido blanco), el canal de capacidad es aproximadamente
\begin{equation}
  C = W \log \left(1 + \dfrac{S}{N} \right)
\end{equation}
M\'as precisamente
$\dfrac{C}{W \log \left(1 + \dfrac{S}{N} \right)} \to 1$ como
$\dfrac{S}{N} \to 0$. Puesto que la se\~nal de potencia media $P$ es
menor o igual a el pico $S$, se deduce que para todos $\dfrac{S}{N}$
\begin{equation}
  C \le W \log \left(1 + \dfrac{P}{N} \right)
  \le W \log \left(1 + \dfrac{S}{N} \right).
\end{equation}
Por lo tanto, si podemos encontrar una familia de funciones tal que
correspondan a la tasa cerca de
$W \log \left(1 + \dfrac{S}{N} \right)$ y se limitan a la banda $W$
y pico $S$ el resultado ser\'a demostrado. Consid\'erese la familia
de funciones del siguiente tipo. Una serie de muestras $t$ tienen el
mismo valor, ya sea $+\sqrt{S}$ o $-\sqrt{S}$, entonces las
siguientes muestras $t$ tienen el mismo valor, etc. El valor de una
serie se elige al azar, probabilidad $\dfrac{1}{2}$ para $+\sqrt{S}$
y $\dfrac{1}{2}$ para $-\sqrt{S}$. Si esta familia se pasa a trav\'es
de un filtro con caracter\'istica de ganancia triangular (unidad de
ganancia en D.C.), la salida est\'a limitada al pico $\pm S$.
Adem\'as la potencia media es casi $S$ y se puede hacer para
acercarse a esto tomando $t$ suficientemente grande. La entrop\'ia
de la suma de este y el ruido t\'ermico se encuentra aplicando el
teorema de la suma de un ruido y una peque\~na se\~nal. Este teorema
se aplicar\'a si
\begin{equation}
  \sqrt{t} \dfrac{S}{N}
\end{equation}
es suficientemente peque\~no. Esto puede garantizarse tomando
$\dfrac{S}{N}$ suficientemente peque\~no (despu\'es de que se elige
$t$). La energ\'ia de la entrop\'ia ser\'a $S + N$ para acercarse a
una aproximaci\'on como se desee, y por lo tanto la tasa de
transmisi\'on tan cerca como queremos
\begin{equation}
  W \log \left(\dfrac{S + N}{N} \right).
\end{equation}

\chapter{La tasa para una fuente continua}

\section{Funciones para una evaluaci\'on de fidelidad}

En el caso de una fuente discreta de informaci\'on que fueron capaces
de determinar una tasa definida de generaci\'on de informaci\'on, es
decir, la entrop\'ia del proceso estoc\'astico subyacente. Con una
fuente continua, la situaci\'on es considerablemente m\'as
complicada. En primer lugar, una cantidad de variaci\'on continua
puede asumir un n\'umero infinito de valores y requiere, por lo
tanto, un n\'umero infinito de d\'igitos binarios para la
especificaci\'on exacta. Esto significa que para transmitir la salida
de una fuente continua con {\em recuperaci\'on exacta} en el punto
de recepci\'on requiere,
 % pp. 46--47 recientemente asignados
\chapter{La tasa para una fuente continua}

En el caso de una fuente continua de informaci\'on nos fue posible
determinar una definida tasa de generaci\'on de informaci\'on, esta es
la entrop\'ia del proceso estoc\'astico subyacente. Con una continua
fuente, la situaci\'on es m\'as complicada. En primer lugar, una
cantidad continuamente variable puede ser asumida como un n\'umero
infinito de valores y por lo tanto requiere un n\'umero infinito de
d\'igitos binarios para su especificaci\'on exacta. Esto significa que
para transmitir la salida de una fuente continua con una {\em
 recuperaci\'on exacta} en el punto de recepci\'on, requiere
generalmente un canal de capacidad infinita (en bits por
segundo). Debido a que, ordinariamente, los canales tienen una cierta
cantidad de ruido, y por lo tanto una capacidad finita, la
transmisi\'on exacta es imposible.

Esto, aun as\'i, evade el problema real. De forma pr\'actica, nosotros
no estamos interesados en transmisi\'on exacta cuando tenemos una
fuente continua, sino solamente en la transmisi\'on dentro de una
cierta tolerancia. La cuesti\'on es si podemos asignar una tasa
definida a una fuente continua cuando requerimos solamente una cierta
fidelidad de recuperaci\'on, medida en una forma adecuada. Claro, a
como los requerimientos de fidelidad sean incrementados la tasa se
incrementar\'a de igual manera. Ser\'a mostrado que podemos, en casos
muy generales, definir tal tasa, teniendo la propiedad de que es
posible, propiamente mediante la codificaci\'on de la informaci\'on
para transmitirla a otro canal cuya capacidad sea igual a la tasa en
cuesti\'on, y as\'i satisfacer los requerimientos de fidelidad. Un
canal de menor capacidad es insuficiente.

Primero es necesario dar la formulaci\'on matem\'atica general de la
idea de fidelidad de transmisi\'on. Considera el conjunto de mensajes
de larga duraci\'on, digamos $T$ segundos. La fuente es descrita dando
la densidad de probabilidad en el espacio asociado, as\'i que la
fuente seleccione el mensaje en cuesti\'on $P(x)$. Un cierto sistema
de comunicaci\'on es descrito(desde el punto de vista externo) dando
la probabilidad condicional $P_{x}(y)$ as\'i que si el mensaje $x$ es
producido por la fuente, el mensaje recuperado en el punto de
recepci\'on ser\'a $y$. El sistema como un todo(incluyendo la fuente y
el sistema de transmisi\'on) es descrito por la funci\'on de
probabilidad $P(x, y)$, probabilidad de tener mensaje $x$ y salida
final $y$. Si esta funci\'on es conocida, las caracter\'isticas
completas del sistema desde el punto de vista de fidelidad son
conocidas. Cualquier evaluaci\'on de fidelidad debe corresponder
matem\'aticamente a una operaci\'on aplicada a $P(x, y)$. Esta
operaci\'on debe tener por lo menos las propiedades de un simple
ordenamiento de sistemas, por ejemplo, debe ser posible decir que dos
sistemas representados por $P_{1}(x, y)$ Y $P_{2}(x, y)$ que, de
acuerdo a nuestro criterio de fidelidad cumpla con ya sea (1) el
primero tiene una fidelidad m\'as alta, (2) el segundo tiene una
fidelidad m\'as alta, o (3) cuentan con una fidelidad
equivalente. Esto significa que el criterio de fidelidad puede ser
representado mediante una funci\'on n\'umericamente valuada.
\begin{equation} v(P(x,y)) \end{equation}
cuyos argumentos van m\'as all\'a de las posibles funciones de
probabilidad $P(x,y)$. Ahora mostraremos que bajo suposiciones muy
generales y razonables, la funci\'on $v(P(x,y))$ puede ser escrita en
una forma aparentemente mucho m\'as especializada, esta siendo un
promedio de una funci\'on $\rho(x,y)$ sobre el conjunto de valores
posibles de $x$ y $y$:
\begin{equation} 
v(P(x,y)) = \int \int P(x,y) \rho(x,y) \diff x  \diff y . 
\end{equation}
Para obtener esto necesitamos solamente asumir (1) que la fuente y el
sistema son erg\'odicos as\'i que una muestra muy larga ser\'a,
probablemente cercana a 1, t\'ipicamente del conjunto, y (2) que la
evaluaci\'on es razonable en el sentido que es posible, mediante la
observaci\'on de una tipica entrada y salida $x_{1}$ y $y_{1}$, formar
la evaluaci\'on tentativa en la base de esas muestra; y si estas
muestras son incrementadas en duraci\'on la evaluaci\'on tentativa,
con probabilidad 1, se acercar\'a a la evaluaci\'on exacta basada en
un total conocimiento de $P(x, y)$. Digamos que la evaluaci\'on
tentativa es $\rho(x, y)$. Entonces la funci\'on $\rho(x, y)$ se
acerca (como $T \rightarrow \infty$) a una constante para la mayor\'ia
(x,y) los cuales est\'an en la regi\'on altamente probable
correspondiente al sistema:
\begin{equation} \rho(x, y) \rightarrow v(P(x, y)) \end{equation}
y tambi\'en podemos escribir
\begin{equation} \rho(x, y) \rightarrow \int \int P(x,y)\rho(x, y) \diff x  \diff y \end{equation}
debido a que
\begin{equation} \int \int P(x, y) \diff x  \diff y = 1 \end{equation}
Esto establece el resultado deseado. La funci\'on $\rho(x, y)$ tiene
la naturaleza general de una "distancia" entre $x$ y $y^{9}$. Mide
que tan indeseable es (de acuerdo a nuestro criterio de fidelidad)
recibir $y$ mientras $x$ es transmitido. El resultado general dado
anteriormente puede ser expresado como sigue: Cualquier evaluaci\'on
razonable puede ser representada como un promedio de una funci\'on de
distancia sobre el conjunto de mensajes y mensajes recuperados $x$ y
$y$ ponderados de acuerdo a la probabilidad $P(x, y)$ de obtener el
par en cuesti\'on, siempre que la duraci\'on $T$ de los mensajes sea
suficientemente larga. \footnote{No es ``m\'etrica'' en el sentido
 estricto, ya que en general no satisface uno u otro ya sea:
 $\rho(x,y) = \rho(y,x)$ o: $\rho(x,y) + \rho(y,z) \geq \rho(x,x)$.}

\begin{enumerate}
\item{Criterio R.M.S. 
\begin{equation} v = (x(t) - y(t))^{2} \end{equation}
En esta medida de fidelidad muy com\'unmente usada, la funci\'on de
distancia $\rho(x,y)$ es (aparte de un factor constante) el cuadrado
de la distancia Euclidiana ordinaria entre los puntos $x$ y $y$ en la
funci\'on espacio asociada.
\begin{equation} \rho(x, y) = \frac{1}{T} \int_0^T [x(t) - y(t)]^{2} \diff t \end{equation}}
\item{Criterio R.M.S. con frecuencia ponderada. M\'as generalmente uno puede
aplicar diferentes ponderaciones a los diferentes componentes de frecuencia
antes de usar una medici\'on de fidelidad R.M.S. Esto es el equivalente a pasar
la diferencia $x(t) - y(t)$ a trav\'es de un filtro de conformaci\'on y entonces 
determinar la potencia promedio en la salida.
As\'i, sea 
\begin{equation} e(t) = x(t) - y(t) \end{equation}
y
\begin{equation} f(t) = \int_{-\infty}^{\infty}  \epsilon(\tau)k(t - \tau) 
\diff \tau \end{equation}
entonces
\begin{equation} 
\rho (x, y) = \frac{1}{T} \int_0^T f(t)^{2} \diff t 
\end{equation}}
\item{Criterio del error absoluto
\begin{equation} 
\rho(x,y) = \frac{1}{T} \int_0^T | x(t) - y(t)| \diff t . 
\end{equation}}
\item{La estructura de la oreja y el cerebro determina
 impl\'icitamente una evaluaci\'on, o m\'as bien un n\'umero de
 evaluaciones, apropiado en el caso de transmisi\'on de m\'usica o
 habla. Hay, por ejemplo, un criterio de ``inteligibilidad'' en el cual
 $\rho(x,y)$ es equivalente a la frecuencia relativa de palabras
 incorrectamente interpretadas cuando el mensaje $x(t)$ es recibido
 como $y(t)$. Aunque no podemos dar una representaci\'on expl\'icita
 de $\rho(x,y)$, en esos casos podr\'ia, en principio, ser
 determinada por suficiente experimentaci\'on. Algunas de sus
 propiedades hacen seguimiento a buenos experimentos conocidos sobre
 el o\'ido, por ejemplo, la oreja es relativamente insensible a la
 fase y la sensibilidad de amplitud y frecuencia es aproximadamente
 logar\'itmica.}
\item{El caso discreto puede ser considerado como una
 especializaci\'on en la cual hemos asumido t\'acitamente una
 evaluaci\'on basada en la frecuencia de los errores. La funci\'on
 $\rho(x,y)$ es entonces definida como el n\'umero de s\'imbolos en
 la secuencia y que difieren de s\'imbolos correspondientes en $x$
 dividido por el total de n\'umero de s\'imbolos en $x$.}
\end{enumerate}

\clearpage

\section{La tasa para una fuente relativa a una evaluaci\'on de fidelidad}

Estamos ahora en una posici\'on de definir la tasa de generaci\'on de
informaci\'on para una fuente continua. Se nos da $P(x)$ para la
fuente y una evaluaci\'on v determinada por una funci\'on de distancia
$\rho(x,y)$ la cual se asumir\'a continua en ambos $x$ y $y$. Con un
sistema particular $P(x,y)$ la calidad es medida por
\begin{equation} 
v = \int \int \rho(x,y) P(x,y) \diff x  \diff y 
\end{equation}
M\'as a\'un, la tasa de flujo de d\'igitos binarios correspondientes a
$P(x,y)$ es
\begin{equation} 
R = \int \int P(x,y) \log \frac{P(x,y)}{P(x)P(y)} \diff x  \diff y \end{equation}
Definimos que la tasa $R_{1}$ de generaci\'on de informaci\'on para
una calidad de reproducci\'on dada v1 sea el m\'inimo R cuando
mantenemos v fija en $v_{1}$ y $P_{x}(y)$ variable. Esto es:
\begin{equation} R_{1} = \min_{P_{x}(y)} \int \int P(x,y) \log \frac{P(x,y)}{P(x)P(y)} \diff x  \diff y \end{equation}
sujeto a la restricci\'on:
\begin{equation} v_{1} = \int \int P(x,y) \rho(x,y) \diff x  \diff y . \end{equation}
Esto significa que consideramos, en efecto, todos los sistemas de
comunicaci\'on que pueden ser usados y que transmiten con la fidelidad
requerida. La tasa de transmisi\'on en bits por segundo es calculada
para cada uno y escogemos el que tiene la menor tasa. \'Esta ultima
tasa es la tasa que asignamos a la fuente para la fidelidad en
cuesti\'on.

La justificaci\'on de esta definici\'on est\'a en el siguiente resultado:

\begin{theorem}
Si una fuente tiene una tasa $R_{1}$ para una valuaci\'on $v_{1}$, es
posible codificar la salida de la fuente y transmitirla sobre un canal
de capacidad $C$ con fidelidad tan cercana a $v_{1}$ como se desee,
siempre que $R_{1} \leq C$. Esto no es posible si $R_{1} > C$ .
\end{theorem}

El \'ultimo enunciado del teorema sigue inmediatamente de la
definici\'on de $R_{1}$ y resultados previos. Si esto no fuera cierto
podr\'iamos transmitir m\'as de C bits por segundos sobre un canal de
capacidad C. La primera parte del teorema es comprobada por un
m\'etodo an\'alogo al que fue usado en el Teorema \ref{FALTA PONER
 ETIQUETA}. Podemos, en primer lugar, dividir el espacio $(x,y)$ en
un gran n\'umero de peque\~{n}as celdas y representar la situaci\'on
en un caso discreto. Esto no va a cambiar la funci\'on de evaluaci\'on
por m\'as que una peque\~{n}a cantidad arbitraria (cuando las celdas
son muy peque\~{n}as) debido a la continuidad asumida para
$\rho(x,y)$. Suponga que $P_1(x,y)$ es el sistema particular el cual
minimiza la tasa y da $R_{1}$. Escogemos desde las $y$'s de alta
probabilidad, un conjunto al azar que contenga
\begin{equation} 2^{(R1 + E)T} \end{equation}
miembros donde $E \rightarrow 0$ como $T \rightarrow \infty$. Con una
$T$ grande, cada punto escogido ser\'a conectado por una linea de alta
probabilidad (como en la figura \ref{FALTA LA ETIQUETA}) a un conjunto
de $x$'s. Un c\'alculo similar al usado para comprobar el Teorema
\ref{FALTA LA ETIQUETA} muestra que con una $T$ grande la mayor\'ia de
las $x$'s son cubiertas por los $fans$ de los puntos y escogidos, para
la mayor\'ia de las elecciones de $y$'s. El sistema de comunicaci\'on
a ser usado opera como sigue: Los puntos seleccionados son n\'umeros
binarios asignados. Cuando un mensaje $x$ es originado se encontrara
dentro de al menos uno de los $fans$(con probabilidad acerc\'andose
uno ya que $T \rightarrow \infty$). El n\'umero binario
correspondiente es transmitido(o uno de ellos es escogido
arbitrariamente si existen m\'ultiples) sobre el canal por modos de
codificaci\'on adecuados para dar una pequeña probabilidad de
error. Ya que $R_{1} \leq C$, esto es posible. En el punto de
recepci\'on la y correspondiente es reconstruida y usada como el
mensaje de recuperaci\'on.

La evaluaci\'on $v'_{1}$ para este sistema se puede hacer
arbitrariamente cercana a $v_{1}$ tomando una T suficientemente
grande. Esto es debido a el hecho de que para cada muestra larga de un
mensaje x(t) y un mensaje de recuperaci\'on $y(t)$, la evaluaci\'on se
acerca a $v_{1}$ (con probabilidad 1). Es interesante notar que, en
este sistema, el ruido en el mensaje recuperado es en realidad
producido por un tipo de cuantificaci\'on general en el transmisor y
no producido por el ruido en el canal. Es m\'as o menos an\'alogo al
ruido de cuantificaci\'on en PCM.

\clearpage

\section{El c\'alculo de las tasas}

La definici\'on de la tasa es similar en muchos aspectos a la
definici\'on de capacidad de canal. En la primera:
\begin{equation} 
R = \min_{P_{x}(y)} \int \int P(x,y) \log \frac{P(x,y)}{P(x)P(y)}
\diff x \diff y 
\end{equation}
con $P(x)$ y $v_{1} = \int \int P(x,y) \rho(x,y) \diff x \diff y $
fija. En la segunda:
\begin{equation} 
C = \max_{P(x)} \int \int P(x,y) \log \frac{P(x,y)}{P(x)P(y)} 
\diff x \diff y 
\end{equation} 
con $P_{x}(y)$ fija y posibilidad de uno o
m\'as restricciones (por ejemplo, una limitaci\'on de potencia
promedio) de la forma $K = \int \int P(x,y) \lambda(x,y) \diff x \diff
y $. Una soluci\'on parcial del problema de maximizaci\'on general
para determinar la tasa de una fuente se puede dar.

Usando el m\'etodo de Lagrange, consideramos:
\begin{equation} \int \int [ P(x,y) \log \frac{P(x,y)}{P(x)P(y)} + \mu P(x,y)\rho(x,y) + v(x)P(x,y)] \diff x  \diff y \end{equation}
La equaci\'on variacional (cuando tomamos la primera variaci\'on de $P(x,y)$) lleva a:
\begin{equation} Py(x) = B(x)\epsilon^{-\lambda \rho(x,y)} \end{equation}
donde $\lambda$ es determinada para dar la fidelidad requerida y $B(x)$ es elegida para
satisfacer:
\begin{equation} \int B(x) \epsilon^{\lambda \rho(x,y)} \diff x  = 1. \end{equation}
			
Esto muestra que, con la mejor codificaci\'on, la probabilidad
condicional de una cierta causa de variaci\'on recibida $y$,
$P_{y}(x)$ estar\'a en decline exponencialmente con la funci\'on de
distancia $\rho(x,y)$ entre el $x$ y $y$ en cuesti\'on. En el caso
epecial donde la funci\'on de distancia $\rho(x,y)$ depende solo en la
diferencia (vector) entre $x$ y $y$,
\begin{equation} \rho(x,y) = \rho(x - y) \end{equation}
tenemos
\begin{equation} \int B(x) \epsilon^{-\lambda p(x-y)} \diff x = 1 \end{equation}
			
Por lo tanto $B(x)$ es contante, digamos alpha, y
\begin{equation} P_{y}(x) = \alpha \epsilon^{ -\lambda \rho(x - y)}. \end{equation}
			
Desafortunadamente etas soluciones formales son dif\'iciles de evaluar
en casos particulares y parece ser de poco valor. De hecho, el calculo
actual de las tasas ha sido llevado a cabo en solo alguno casos muy
simples. Si la funci\'on de ditancia p(x,y) es el cuadrado medio de
la discrepancia entre $x$ y $y$, y el mensaje conjunto es ruido
blanco, la tasa puede ser determinada. En ese cao tenemos
\begin{equation} R = \min[ H(x) - H_{y}(x)] = H(x) - \max H_{y}(x) 
\end{equation}
con $N = \overline{(x-y)^{2})}$. Pero el $\max H_{y}(x)$ ocurre cuando
$y - x$ es un ruido blanco, y es equivalente a $W_{1} \log 2 \pi
\epsilon N$ donde $W_{1}$ es el ancho de banda del mensaje conjunto.
Por lo tanto
\begin{equation} R = W_{1} \log 2 \pi \epsilon Q - W_{1} \log 2 \pi \epsilon N \end{equation}
\begin{equation} = W_{1} \log \frac{Q}{N} \end{equation}
donde $Q$ es la potencia promedio del mensaje. Esto comprueba lo
siguiente:

\begin{theorem}
La tasa para la medici\'on de fidelidad de una fuente de ruido blanco
de potencia $Q$ y banda $W_{1}$ relativa a un R.M.S. es:
\begin{equation} R = W_{1} \log \frac{Q}{N} \end{equation}
donde $N$ es es el cuadradio medio del error permitido entre el
mensaje original y el recuperado.
\end{theorem}

M\'as generalmente, con cualquier fuente de mensaje podemos obtener
desigualdades delimitando la tasa a un criterio de cuadrado medio del
error.

\begin{theorem}
La tasa para cualquier fuente de banda $W_{1}$ es delimitada por:
\begin{equation} W_{1} \log Q1/N <= R <= W_{1} \log Q/N \end{equation}
donde $Q$ es la potencia promedio de la fuente, $Q_{1}$ la energ\'ia
de entrop\'ia y $N$ el cuadrado medio del error permitido.
\end{theorem}

El limite inferior sigue el hecho de que la $\max H_{y}(x)$ para un
$\overline{(x - y)^{2}} = N$ dado ocurre en el caso de ruido
blanco. El limite superior resulta si colocamos puntos (usados en la
comprobaci\'on del Teorema \ref{FALTA LA ETIQUETA}) no en la mejor
forma sino al azar en una esfera de radio $\sqrt{(Q - N)}$.

\clearpage

\section*{Reconocimientos}

El escritor est\'{a} en deuda con sus colegas en los laboratorio,
particularmente al Dr.\ H.\ W.\ Bode, Dr.\ J.\ R.\ Pierce,
Dr.\ B.\ McMillan y al Dr.\ B.\ M.\ Oliver, por muchas sugerencias y
criticismos \'utiles durante el curso de su trabajo. Cr\'{e}dito debe
tambi\'en ser otorgado al Profesor N.\ Wiener, cuya soluci\'on
elegante al problema de filtraci\'on y predicci\'on de conjuntos
estacionarios ha influido considerablemente la forma de pensar del
escritor en este campo de estudio.

\clearpage

\begin{appendices}

\chapter{}

Sea $S_{1}$ cualquier subconjunto medible del conjunto $g$, y $S_{2}$
el subconjunto del conjunto $f$ el cual da $S_{1}$ bajo la operaci\'on
$T$. Entonces
\begin{equation} S_{1} = TS_{2}.\end{equation}

Sea $H^{\lambda}$ el operador que desplaza todas las funciones en un
conjunto con tiempo $\lambda$. Entonces
\begin{equation} 
H^{\lambda} S_{1} = H^{\lambda} T S_{2} = T H^{\lambda} S_{2}. 
\end{equation}
debido a que $T$ es invariante y por lo tanto conmuta con
$H^{\lambda}$. Por lo tanto, si $m[S]$ es la probabilidad de
medici\'on del conjunto $S$,
\begin{equation} 
\begin{array}{rcl}
m[H^{\lambda} S_{1}] &=& m[T H^{\lambda} S_{2}] = m[H^{\lambda} S_{2}] \\
&=& m[S_{2}] = m[S_{1}]
\end{array}
\end{equation}
donde la segunda igualdad es por definici\'on la medici\'on del
espacio $g$, el tercero ya que el conjunto $f$ es estacionario, y el
\'ultimo nuevamente por definici\'on de la medici\'on de $g$.

Para probar que la propiedad ergódica es preservada bajo operaciones
invariantes, sea $S_{1}$ un subconjunto del conjunto $g$, el cual es
invariante bajo $H^{\lambda}$, y sea $S_{2}$ el conjunto de todas las
funciones $f$ que se transforman en $S_{1}$. Entonces
\begin{equation} H^{\lambda} S_{1} = H^{\lambda} T S_{2} = T H^{\lambda} S_{2} = S_{1} \end{equation}
as\'{\i} que $H^{\lambda} S_{2}$ es incluida en $S_{2}$ para todas las
$\lambda$. Ahora, debido a que
\begin{equation} m[H^{\lambda} S_{2}] = m[S_{1}] \end{equation}
esto implica
\begin{equation} H^{\lambda} S_{2} = S_{2} \end{equation}
para todo $\lambda$ con $m[S_{2}] \neq 0, 1$. Esta contradicci\'on muestra que $S_{1}$ no existe.

\chapter{}

El l\'{\i}mite superior, $\overline{N_{3}} \leq N_{1} + N_{2}$, se
debe al hecho que la m\'axima entrop\'ia posible para la potencia
$N_{1} + N_{2}$ ocurre cuando tenemos ruido blanco de esta
potencia. En este caso la energ\'ia de entrop\'ia es $N_{1} + N_{2}$.

Para obtener el l\'imite inferior, suponga que tenemos dos
distribuciones en $n$ dimensiones $p(x_{i})$ y $q(x_{i})$ con
energ\'ias de entrop\'ia $\overline{N_{1}}$ y $\overline{N_{2}}$. Que
forma deber\'ia $p$ y $q$ tener para poder minimizar la energ\'ia de
entrop\'ia $\overline{N_{3}}$ de su convoluci\'on $r(x_{i})$:
\begin{equation} r(x_{i}) = \int p(y_{i}) q(x_{i} - y_{i}) \diff y _{i}.
\end{equation}
			
La entrop\'ia $H_{3}$ de $r$ es dada por:
\begin{equation} H_{3} = - \int r(x_{i}) \log r(x_{i}) \diff x _{i}. \end{equation}
			
Deseamos minimizar esto sujeto a las restricciones:
\begin{equation} H_{1} = - \int p(x_{i}) \log p(x_{i}) \diff x _{i} \end{equation}
\begin{equation} H_{2} = - \int q(x_{i}) \log q(x_{i}) \diff x _{i}. \end{equation}
				
Consideramos entonces
\begin{equation} U = - \int [r(x) \log r(x) + 
\lambda p(x) \log p(x) + \mu q(x) \log q(x)] \diff x \end{equation}
\begin{equation} \delta U = - \int [[1 + \log r(x)] 
\delta r(x) + \lambda [1 + \log p(x) \delta p(x) + 
\mu [1 + \log q(x)]\delta q(x)] \diff x \end{equation}

Si $p(x)$ es variado en un argumento particular $x_{i} = s_{i}$, la variaci\'on en $r(x)$ es
\begin{equation} \delta r(x) = q(x_{i} - s_{i}) \end{equation}
y
\begin{equation} 
\delta U = - \int q (x_{i} - si_{i}) \log r(x_{i}) \diff x _{i} - \lambda \log p(s_{i}) = 0 
\end{equation}
y similarmente cuando $q$ es variado. Entonces las condiciones para un m\'inimo son:
\begin{equation} \int q(x_{i} - s_{i}) \log r(x_{i}) \diff x _{i} 
= -\lambda \log p(s_{i}) \end{equation}
				
\begin{equation} \int p(x_{i} - s_{i}) \log r(x_{i}) \diff x _{i} = -\mu \log q(s_{i}) \end{equation}
				
Si multiplicamos el primero por $p(s_{i})$ y el segundo por $q(s_{i})$ e integramos con 
respecto a $s_{i}$, obtenemos:

\begin{equation} H_{3} = -\lambda H_{1} \end{equation}
\begin{equation} H_{3} = -\mu H_{2} \end{equation}
o resolviendo $\lambda$ y $\mu$, y reemplazando en las ecuaciones
\begin{equation} H_{1} \int q(x_{i} - s_{i}) \log r(x_{i}) \diff x _{i} 
= - H_{3} \log p(s_{i}) 
\end{equation}
\begin{equation} H_{2} \int p(x_{i} - s_{i}) \log r(x_{i}) \diff x _{i} 
= - H_{3} \log q(s_{i}) \end{equation}

Ahora supongamos que $p(x_{i})$ y $q(x_{i})$ son normales
\begin{equation} p(x_{i}) = \frac{|A_{ij}^{\frac{n}{2}}|}{(2\pi)^{\frac{n}{2}}} exp - \frac{1}{2} \Sigma(A_{ij} x_{i} x_{j}) \end{equation}
\begin{equation} q(x_{i}) = \frac{|B_{ij}^{\frac{n}{2}}|}{(2\pi)^{\frac{n}{2}}} exp - \frac{1}{2} \Sigma(B_{ij} x_{i} x_{j}) \end{equation}

Entonces $r(x_{i})$ puede tambi\'en ser normal con la forma
cuadr\'atica $C_{ij}$. Si las inversas de \'estas formas son $a_{ij}$,
$b_{ij}$ y $c_{ij}$, entonces
\begin{equation} c_{ij} = a_{ij} + b_{ij}. \end{equation}
				
Deseamos mostrar que estas funciones satisfacen las condiciones de
minimizaci\'on s\'i y solo si $a_{ij} = Kb_{ij}$ y por lo tanto da el
m\'inimo $H_{3}$ bajo las restricciones. Primero tenemos
\begin{equation} \log r(x_{i}) = 
\frac{n}{2} \log \frac{1}{2\pi} |C_{ij}| - \frac{1}{2} 
\Sigma(C_{ij} x_{i} x_{j}) \end{equation}
			
\begin{equation} 
\int q(x_{i} - s_{i}) \log r(x_{i}) \diff x _{i} = \frac{n}{2} \log
\frac{1}{2\pi} |C_{ij}| - \frac{1}{2} \Sigma(C_{ij}S_{i}S_{j}) -
\frac{1}{2} \Sigma(C_{ij} B_{ij}) \end{equation}
		
Esto deberia ser equivalente a
\begin{equation} \frac{H_{3}}{H_{1}} [ \frac{n}{2} \log \frac{1}{2\pi} 
| A_{ij} | - \frac{1}{2} \Sigma(A_{ij} S_{i}S_{j})] \end{equation}
lo cual requiere $A_{ij} = H_{1}/H_{3} C_{ij}$. En este caso $A_{ij} = H_{1}/H_{2} B_{ij}$ y ambs ecuaciones
se reducen a identidades.

\chapter{}
\label{a3}

Lo siguiente indicar\'a un acercamiento m\'as general y riguroso a las
definiciones centrales de teor\'ia de la comunicaci\'on. Consideremos
un espacio de medici\'on de probabilidad cuyos elementos est\'an
ordenados en pares $(x, y)$. Las variables $x$, y ser\'an
identificadas como de todos los puntos cuyos $x$ pertenecen al sub
conjunto Si las posibles señales transmitidas y recibidas en una
larga duraci\'on $T$. Llamaremos al conjunto de todos los puntos cuyas
$x$ pertenecen a un sub conjunt $S_{1}$ de puntos $x$: la tira sobre
$S_{1}$, y similarmente al conjunto cuyas y pertenecen a $S_{2}$, la
tira sobre $S_{2}$. Dividimos $x$ y $y$ en una colecci\'on de
subconjuntos medibles no superpuestos $X_{i}$ y $Y_{i}$, aproximado a
la tasa de transmisi\'on $R$ por

\begin{equation} R_{1} = \frac{1}{T} \displaystyle\sum_{i}(P(X_{i},Y_{i})
\log \frac{P(X_{i}, Y_{i})}{P(x_{i})P(Y_{i})} \end{equation}
donde
\begin{itemize}
\item $P(x_{i})$ es la probabilidad de medici\'on de la tira sobre
 $X_{i}$
\item $P(Y_{i})$ es la probabilidad de medici\'on de la tira sobre
 $Y_{i}$
\item $P(X_{i},Y_{i})$ es la probabilidad de medici\'on dela
 interseccion de las tiras.
\end{itemize}
		
Una subdivisi\'on adicional no puede disminuir $R_{1}$ nunca. Dejemos
que $X_{1}$ sea dividido en $X_{1} = X_{1}' + X_{1}''$ y sea
\begin{equation} 
\begin{array}{rclrcl}
P(Y_{1}) &=& a &
P(X_{1}) &=& b + c \\
P(X_{1}') &=& b &
P(X_{1}', Y_{1}) &=& d \\
P(X_{1}'') &=& c &
P(X_{1}'', Y_{1}) &=& e \\
P(X_{1}, Y_{1}) &=& d + e &&&
\end{array}
\end{equation}
					
Entonces en la suma hemos reemplazado (para la intersecci\'on $X_{1}, Y_{1}$)
\begin{equation} (d + e) \log 
\frac{(d + e)}{a(b + c)}  por d \log \frac{d}{ab} 
+ e \log \frac{e}{ac}. \end{equation}
Es f\'acilmente mostrado que con la limitaci\'on que tenemos en $b, c,
d, e$,
\begin{equation} 
[\frac{d + e}{b + c}]^{d+e} \leq \frac{d{d} e{e}}{b^{d} c^{e}} \end{equation}
y consecuentemente la suma es incrementada. Por lo tanto las varias formas posibles
de subdivisi\'on formanun conjunto dirigido, con $R$ incrementandose monot\'onicamente
con el refinamiento de la subdivisi\'on. Podemos definir $R$ sin ambigüedad como el
menor l\'imite superior para $R_{1}$ y escribirlo
\begin{equation} R = \frac{1}{T} \int \int P(x,y) \log 
\frac{P(x,y)}{P(x) P(y)} \diff x  \diff y \end{equation}
			
La integral, entendida en el sentido anterior, incluye ambos los casos
continus y discretos, y por supuesto, muchos otros que no pueden ser
representados en cualquiera de las formas. Es trivial en esta
formulaci\'on que si $x$ y $u$ est\'an en correspondencia uno a uno,
la tasa de $u$ a $y$ es equivalente a aquella entre $x$ y $y$. Si $v$
es cualquier funci\'on de $y$ (no necesariamente con una inversa)
entonces la tasa desde $x$ a $y$ es mayor o igual a aquella entre $x$
a $v$ debido a que, en el c\'alculo de las aproximaciones, las
subdivisiones de $y$ son escencialmente subdivisiones m\'as finas que
aquellas para $v$. M\'as generalmente si $y$ y $v$ est\'an
relacionadas, no funcionalmente pero estad\'isticamente, por ejemplo
si tenemos un espacio de medida de probabilidad $(y, v)$, entonces
$R(x,v) \leq R(x,y)$. Esto significa que cualquier operaci\'on
aplicada a la señal recibida, aunque involucre elementos
estad\'isticos, no incrementa $R$.

Otra noci\'on que debe ser definida precisamente en una formulaci\'on
abstracta de la teor\'ia, es "la tasa de dimensi\'on", que es el
n\'umero promedio de dimensiones por segundo requeridas para
especificar a un miembro de un conjunto. En el caso de una banda
limitada con $2W$ n\'umeros por segundo son suficientes. Una
definici\'on general puede ser enmarcada como sigue. Sea
$f_{\alpha}(t)$ un conjunto de funciones y sea $\rho_{\tau} \left
[f_{\alpha}(t),f_{\beta}(t) \right ]$ una m\'etrica midiendo la forma
de "distancia" desde $f_{\alpha}$ hasta $f_{\beta}$ sobre el tiempo
$T$(por ejemplo la discrepancia R.M.S. sobre \'este intervalo). Sea
$N(\varepsilon, \delta, \tau)$ el menor n\'umero de elementos $f$
que pueden ser elegidos, as\'i que todos los elementos del conjunto
adem\'as de un conjunto de medici\'on $\delta$, est\'an dentro de
distancia $\varepsilon$ de por lo menos uno de los escogidos.

Por lo tanto, estamos cubriendo el espacio dentro de $\varepsilon$
separado de un conjunto de poca medida $\delta$. Definimos la tasa de
dimensi\'on $\lambda$ para el conjunto, por el triple l\'imite
\begin{equation} 
\lambda = 
\lim_{\delta \to \infty} 
\lim_{\varepsilon \to \infty} 
\lim_{\tau \to \infty} 
\frac{\log N(\varepsilon, \delta, \tau)}{\tau \log \varepsilon} 
\end{equation} 
Esta es una generalizaci\'on de las definiciones de tipo de medida de
la dimensi\'on en la topolog\'ia, y est\'a de acuerdo con la tasa de
dimensi\'on intuitiva para conjuntos simples donde los resultados
deseados son obvios.

\end{appendices}
 % pp. 47--55 terminados


\end{document}
