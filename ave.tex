En el caso sin ruido un retraso generalmente requiere para la
codificacion\'on ideal ahora esta tiene una funci\'on adicional que
permite una muestra larga de ruido para afectar la se\~{n}al antes de
un juicio, se hace en el punto de resepci\'on para el mensaje
original. Incrementando el tama\~{n}o del ejemplo siempre agudiza las
posibles afirmaciones estad\'isticas.

El contenido del teorema 11 PONER REF y su prueba se pueden formular
de una manera diferente que muestra una conexi\'on sin ruido de una
manera mas clara. Considere las posibles duraciones de las se\~{n}ales
T y supongamos un subconjutno de ellos se seleccionan para ser
usados. Los que esten en el subconjunto se utilizen todos con igual
probabilidad, y suponiendo que el resector es construido para
seleccionar, como la se\~{n}al original. la causa m\'as probable de
subconjutno cuando una se\~{n}al perturbada es recivida. Nosotros
definimos $N(T,q)$ siendo el numero m\'aximo de se\~{n}ales que podemos
elegir para el subconjunto tal que la probabilidad de una
interpretaci\'{o}n incorrecta sea menor o igual a $q$.

\begin{theorem}
\begin{equation}
\lim_{t \to{}\infty}\frac{\log{N}(T,q)}{T} = C,
\end{equation}
donde $C$ es la capacidad de canal,a condici\'on que q no sea igual a
0 o 1
\label{t12}
\end{theorem}

En otras palabras, no importa lo que nos propusimos de limites de
confiabilidad, podemos distinguir de forma fiable en el tiempo $T$
suficientes mensajes para corresponder a $CT$ bits, cuando $T$ es
suficientemente grande. En el teorema \ref{t12} podemos comparar la
capacidad de un canal sin ruido en la secci\'on uno PONER REF.

\section{Ejemplo de un canal discreto y su capacidad}

Un ejemplo simple de un canal discreto se indica en la figura 11 PONER
REF. Hay tres posibles s\'imbolos. El primero nunca se vera afectado por
el ruido. El segundo y la tercera tiene cada una probabilidad $p$ de que
viene a trav\'es impertubable q de ser cambiado en el otro elemento
par. Nosotros tenemos (dejar $\alpha = -[plogp + q\log{q}]$ y $p$ y $q$
son probabilidades de estar usando los s\'imbolos primero y segundo).

\begin{equation}
\begin{array}{rcl}
H(x) &=& -P\log{P} - 2Q\log{Q} \\
H_y(x) &=& 2Q\alpha
\end{array}
\end{equation}

Nosotros deseamos elegir $P$ y $Q$, de tal manera que se maximice
$H(x) - H_y(x)$ sujeto a la restricci\'on $P + 2Q = 1$.

Por lo tanto concideremos:
\begin{equation}
\begin{array}{rcl}
U &=& -P\log{P} - 2Q\log{Q} -2Q\alpha + \lambda(P+2Q) \\
\frac{{\partial U}}{{\partial P}} &=& -1 - \log{P} + \lambda = 0 \\
\frac{{\partial U}}{{\partial P}} &=& -2 - 2\log{Q} -2\alpha + 2\lambda = 0
\end{array}
\end{equation}

Eliminando $\lambda$
\begin{equation}
\begin{array}{rcl}
\log{P} &=& \log{Q} + \alpha \\
P &=& Q e^\alpha = Q\beta 
\end{array}
\end{equation}

\begin{equation}
	P = \frac{\beta}{\beta + 2}   Q = \frac{1}{\beta + 2}
\end{equation}

La capacidad del canal es de:

\begin{equation}
	C = \log{\frac{\beta + 2}{\beta}}
\end{equation}

Notese como esto comprueba los valores obvios en los casos: $p = 1$ y $p = \frac{1}{2}$. En primero, $\beta = 1$ y $C = \log{3}$, 
el cual es correcto debido a que el canal es entonces sin ruido con tres posibles s\'imbolos. Si $p = \frac{1}{2}$, $\beta = 2$ y 
$C = \log{2}$.Aqu\'i el segundo y el tercer s\'imbolos, no se pueden distinguir en absoluto y act\'uan conjuntamente como un solo 
s\'imbolo. El primer s\'imbolo se utiliza con una probabilidad $P = \frac{1}{2}$  y el segundo junto al tercero con probabilidad $\frac{1}{2}$.
Esto puede ser distribuido entre ellos de cualquier modo deseado y todav\'ia alcanzar la m\'axima capacidad.

Para los valores intermedios de la capacidad del canal $p$ estara entre $\log{2}$ y $\log{3}$.Esta distinci\'on 
entre el segundo y tercer s\'imbolo transmite alguna informaci\'on, pero no tanto como en el caso sin ruido.
El primer s\'imbolo se utiliza tanto mas frecuentemente que los otros dos, debido a su ausencia de ruido.

\section{16. LA CAPACIDAD DE CANALES EN CIERTOS CASOS ESPECIALES}


Si el ruido afecta s\'imbolos sucesivos de canal de forma independiente pueden ser descritos por un conjunto de transici\'on 
de probabilidades $p_{i,j}$.Esta es la probabilidad, si el simbolo i es enviado, que j sera recibido.La tasa de canal m\'aximo 
viene dado por el m\'aximo de:

\begin{equation}
	\sum_{i,j}P_i p_{i,j} \log{\sum{P_i p_{i,j}}} + \sum_{i,j}P_i p_{i,j}\log{p_{i,j}}
\end{equation}
 
Multiplicar por P_s y sumando en s muestra que $\mu = C$. Vamos a la inversa de $p_{sj}$ (si existe) en $h_{st}$  de modo que 
$\sum_{s}$ $h_{st}$ $p_{sj}$ = $\delta_{tj}$.Entonces: 

\begin{equation}
	\sum_{s,j}h_{st} p_{s,j} \log{p_{s.j}} - \log{\sum_{i}P_i p_{i,t}} = C \sum_{s} h_{s,t}
\end{equation}
Por lo tanto:

\begin{equation}
	\sum_{i} P_i p_{i,t} = exp[- C \sum_{s} h_{s,t}+ \sum_{s,j} h_{s,t} p_{s,j} \log{p_{s,j}}]
\end{equation}

o:  
\begin{equation}
	P_i = \sum_{t} h_{i,t} exp[ - C \sum_{s} h_{s,t}+ \sum_{s,j} h_{s,t} p_{s,j} \log{p_{s,j}} ]
\end{equation}


Este es el sistema de ecuaciones para determinar el valor m\'aximo de $P$, con $C$ se determina 
de manera que $\sum P_i = 1$. Cuando esto esta hecho, $C$ sera la capacidad del canal y $P_i$ las probabilidades 
para los s\'imbolos de canal para lograr esta capacidado.
Si cada s\'imbolo de entrada tiene el mismo conjunto de probabilidades en las l\'ineas que emergen de ella 
y lo mismo sucede a cada s\'imbolo de salida, la capacidad puede ser calculada f\'acilmente. Los ejemplos se muestran en la figura 12. 
En tal caso $H_xy$ es independiente de la destribuci\'on de probabilidades de los s\'imbolos de entrada, y esta dad por -\sum p_i \log{p_i}. 
Cuando $p_i$ son los valores de probabilidad de transici\'on de cualquier s\'imbolo de entrada.La capacidad del canal es:

\begin{equation}
	MAX [H(y) - H_x(y)] = MAX H(y) + \sum p_i \log{p_i}.
\end{equation}
  
El valor m\'aximo de $H(y)$ esta claramente  $\log{m}$ donde $m$ es el numero de s\'imbolos de salida, ya que es posible para hacer 
todos igualmente probables haciendo los s\'imbolos de entradas igualmente probables. La capacidad del canal es por lo tanto:

\begin{equation}
	C = \log{m} + \sum p_i \log{p_i}
\end{equation}

En la figura 12a ser\'ia 

\begin{equation}
	C = = \log{4} - \log{2} 0 \log{2}.
\end{equation}

Esto se podr\'ia lograr mediante el uso solo the la 1a y 3d simbolo. En la Figura b:

\begin{equation}
	C = \log{4} - \frac{2}{3}\log{3} - \frac{1}{3}\log{6}
	= \log{4} - \log{3} - \frac{1}{3}\log{2}
	= \log{\frac{1}{3}} 2^{\frac{5}{3}}
\end{equation}

En la figura 12c nosotros tenemos:

\begin{equation}
	C = \log{3} - \frac{1}{2}\log{2} - \frac{1}{3}\log{3} - \frac{1}{6} \log{6}
	= \log{\frac{3}{ 2^{\frac{1}{2} 3^{\frac{1}{3} 6^{\frac{1}{6} }}
\end{equation}

Supongamos que los s\'imbolos se dividen en varios grupos tal que el ruido causa a un s\'imbolo en un grupo a 
ser confundido con un s\'imbolo de otro grupo. Deja la capacidad de un grupo n-\'esimo ser $C_n$ (en bits por segundo)
donde solo utilizamos los s\'imbolos de este grupo. Entonces es f\'acil desmotrar  de todo el conjunto, 
la probabilidad $P_n$ para todos los s\'imbolos del grupo n-esimo deber\'ia ser:

\begin{equation}
	P_n = \frac{2^{C_n}}{\sum 2^{C_n}}
\end{equation}

En un grupo la probabilidad se distribuye tal como ser\'ia si estos eran los \'unicos s\'imbolos que se utilizan. 
La capacidad del canal es:

\begin{equation}
	C = \log{\sum 2^{C_n}}
\end{equation}

\secction{UN EJEMPLO DE CODIFICACI\'ON EFICIENTE}

El siguiente ejemplo, aunque es un poco realista, es un caso en que la coincidencia exacta para un canal con ruido, 
es posible. Hay dos s\'imbolos de canal 0 y 1, y el ruido les afecta en bloques de siete s\'imbolos. 
Un bloque de siete o se transmite sin error, o exactamente un s\'imbolo de los siete es incorrecta.Estas 
ocho posibilidades son igualmente probables. Nosotros tenemos:

\begin{equation}
C = MAX[ H(y) - H_x(y) ]
= \frac{1}{7}[7 + \frac{8}{8}\log{\frac{1}{8}}]
= \frac{4}{7}bits/s\'imbolos
\end{equation}
 
Un c\'odigo eficiente, permite la correci\'on de todos los errores y transmitir a la tasa C, es el 
siguiente(encontrado por un m\'etodo de R.Hamming):
